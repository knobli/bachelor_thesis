%%%%%%%%%%%%%%%%%%%%%%%%%%%%%%%%%%%%%%%%%%%%%%%%%%%%%%%%%%%%%%%%%
%
% Project     : Bachelorarbeit
% Title       : Machbarkeitsanalyse für eine ressourcenorientierte Schnittstelle zur Verarbeitung grundlegender Probleme der Informatik
% File        : doc.tex Rev. 01
% Date        : 01.03.2015
% Author      : Raffael Santschi
%
%%%%%%%%%%%%%%%%%%%%%%%%%%%%%%%%%%%%%%%%%%%%%%%%%%%%%%%%%%%%%%%%%

%%%%%%%%%%%%%%%%%%%%%%%%%%%%%%%%%%%%%%%%%%%%%%%%%%%%%%%%%%%%%%%%%
%
% Project     : Bachelorarbeit
% Title       : Machbarkeitsanalyse für eine ressourcenorientierte Schnittstelle zur Verarbeitung grundlegender Probleme der Informatik
% File        : header.tex Rev. 01
% Date        : 01.03.2015
% Author      : Raffael Santschi
%
%%%%%%%%%%%%%%%%%%%%%%%%%%%%%%%%%%%%%%%%%%%%%%%%%%%%%%%%%%%%%%%%%

\documentclass[10pt,parskip=half,ngerman,pointlessnumbers]{scrreprt}

%***********************************************************************
% include some libs
%***********************************************************************
\usepackage[utf8]{inputenc}
\usepackage{listings}
\usepackage{color}
\usepackage{fancyhdr}
\usepackage{rotating}
\usepackage{titlesec}
\usepackage{mathptmx}
% \usepackage{helvet}
\usepackage[scaled]{uarial}
\renewcommand*\familydefault{\sfdefault} %% Only if the base font of the document is to be sans serif
\usepackage[T1]{fontenc}
\usepackage{ngerman}
\usepackage{textcomp}
\usepackage[squaren]{SIunits}
\usepackage{graphicx}
\usepackage{subfigure}
\usepackage{url}
\usepackage{geometry}
\usepackage[absolute]{textpos}
\usepackage{makeidx}
\usepackage{colortbl}
\usepackage{pdflscape}
\usepackage{pdfpages}
\usepackage{tabularx}
\usepackage{lmodern}
\usepackage{longtable}
\usepackage{array}
\usepackage{float}
\usepackage{scrhack}
\usepackage[plainpages=false]{hyperref}
\usepackage{wallpaper} %\ThisTileWallPaper{}
\usepackage[super,square]{natbib} %für BibTeX Literaturverzeichnis
\usepackage{amsmath}
\usepackage[section]{placeins}
\usepackage{booktabs}
\usepackage{todonotes}
\usepackage{slashbox}



%***********************************************************************
% various styles
%***********************************************************************	

%create index
\makeindex

%define pagestyle
\pagestyle{fancy}

%use sans-serif font 
%\renewcommand{\familydefault}{\sfdefault}

%define page margin
\geometry{a4paper, top=30mm, left=30mm, right=30mm, bottom=30mm,headsep=10mm,footskip=10mm}

%textpos parameter
\setlength{\TPHorizModule}{30mm}
\setlength{\TPVertModule}{\TPHorizModule}
\textblockorigin{10mm}{10mm} % start everything near the top-left corner
\setlength{\parindent}{0pt}

%increase size of subusubseciont that is not equal to paragraph
\setkomafont{subsubsection}{\large}

%horizontal lines for titlepage 
\newcommand{\HRule}{\rule{\linewidth}{0.5mm}}

%reference to source items inlc source number
\newcommand{\srcref}[1]{\nameref{src:#1} \cite{#1}}

%glossarmark
\newcommand{\glossarmark}[1]{\textcolor{blue}{#1}}
\newcommand{\intromark}[1]{\textcolor{darkgreen}{#1}}
\newcommand{\resultAssignment}[1]{#1}

%header / footer 
\renewcommand{\headrulewidth}{0.3pt}
\renewcommand{\footrulewidth}{0.3pt}

\fancyhead[LO,RE]{} %clear headings for contents 
\fancyhead[RO,LE]{\nouppercase{\rightmark}} %right odd pages and left even pages
\fancyhead[LO,RE]{\MakeUppercase{\leftmark}} %left odd pages and right even pages
\fancyfoot[LE,RO]{\thepage} %page numbering
\fancyfoot[C]{} %clear centered page numbering 

%define some colors
\definecolor{darkgreen}{RGB}{31, 135, 68}
\definecolor{gray}{rgb}{0.85,0.85,0.85}
\definecolor{darkgray}{rgb}{0.4,0.4,0.4}
%listing colors
\definecolor{lgray}{RGB}{250,250,250}
\definecolor{lgreen}{RGB}{63,127,95}
\definecolor{lred}{RGB}{127,0,85}
\definecolor{lblue}{RGB}{42,0,255}

% footnote layout optimization
\deffootnote[1.25em]{1.25em}{1.25em}{\textsuperscript{\normalfont\thefootnotemark\,}}

%***********************************************************************
% listing
%***********************************************************************

\lstset{		
		basicstyle=\small\ttfamily,
		frame=single,
		numbers=left,	
		numberstyle=\tiny,
		%firstnumber=auto,
		numberblanklines=true,
		captionpos=b,
		extendedchars=true,
		float=ht,
		showtabs=false,
		tabsize=2,
		showspaces=false,
		showstringspaces=false,
		breaklines=true,
		%prebreak=\Righttorque,
		backgroundcolor=\color{lgray},
		keywordstyle=\color{lred}\bfseries, 
		commentstyle=\color{lgreen}\ttfamily,
%		morekeywords={printstr, printhexln},
		stringstyle=\color{lblue},
		xleftmargin=0.5cm,
		xrightmargin=0.5cm
}

\lstloadlanguages{Java}

%\lstdefinelanguage{xc}{
%     keywords={printstr, printhexln, attributes, class, classend, do, empty, endif, endwhile, fail, function, functionend, if, implements, in, inherit, inout, not, of, operations, out, return, set, then, types, while, use},
%     keywordstyle=\color{lred}\bfseries,
%     ndkeywords={},
%     ndkeywordstyle=\color{yellow}\bfseries,
%     identifierstyle=\color{black},
%     sensitive=false,
%     comment=[l]{//},
%     commentstyle=\color{lgreen}\ttfamily,
%     string=[l]{"},
%     stringstyle=\color{lblue}\ttfamily
%  }

%%%%%%%%%%%%%%%%%%%%%%%%%%%%%%%%%%%%%%%%%%%%%%%%%%%%%%%%%%%%%%%%%
%
% Project     : Bachelorarbeit
% Title       : Machbarkeitsanalyse für eine ressourcenorientierte Schnittstelle zur Verarbeitung grundlegender Probleme der Informatik
% File        : header.tex Rev. 01
% Date        : 01.03.2015
% Author      : Raffael Santschi
%
%%%%%%%%%%%%%%%%%%%%%%%%%%%%%%%%%%%%%%%%%%%%%%%%%%%%%%%%%%%%%%%%%

\usepackage{xparse}
\usepackage[nopostdot,nonumberlist,acronym]{glossaries}
\DeclareDocumentCommand{\newdualentry}{O{} O{} m m m m }{%
  \newglossaryentry{gls-#3}{name={#5},text={#5\glsadd{#3}},
    description={#6},#1
  }
  \newacronym[#2]{#3}{#4}{\gls{gls-#3}}
}
\makeglossaries
%\renewcommand*{\glstextformat}[1]{\textsf{#1}}
% Customize the format of the first use.  See the manual for details if
% you want to include more information here such as the definition.
\defglsdisplayfirst[\glsdefaulttype]{\textit{#1}}

\newdualentry{ipc}    % label
  {IPC}               % abbreviation
  {Inter Process Communication}  % long form
  {Möglichkeiten der Interaktion zwischen Betriebssystemprozessen. Beispielsweise mithilfe von Sockets, Pipes oder Message Queues.}
\newdualentry{pid}    % label
  {PID}               % abbreviation
  {Process ID}  % long form
  {Eindeutige Kennung eines Prozesses innerhalb des gleichen Linux Namespace.}
\newdualentry{vm}    % label
  {VM}
  {Virtuelle Maschine}
  {Virtualisiertes Betriebssystem, welches durch eine Abstraktionsschicht von physischer Hardware getrennt ist. Vgl.~\ref{sub:VMs}.}
\newdualentry{mmu}    % label
  {MMU}
  {Memory Management Unit}
  {Chip innerhalb eines Computers, der die Übersetzung virtueller \gls{ram} Adressen in physische Adressen innerhalb der Hardware vornimmt.}
\newdualentry{vmm}    % label
  {VMM}
  {Virtual Machine Monitor}
  {Abstraktionsebene zwischen Hardware und (mehreren) Gast-Betriebssystemen, die mit der Bereitstellung der virtuellen Umgebung betraut ist.}
\newdualentry{ebs}    % label
  {EBS}
  {Einschreibe- und Bewertungssystem}
  {Online System zur Verwaltung von Vorlesungen, Noten und Projekten des Studienganges Informatik an der ZHAW Zürich (ehem. HSZ-T).}
\newdualentry{saas}    % label
  {SaaS}
  {Software as a Service}
  {Service Model für Clouds, bei dem der Anwenderin vollständige Software-Lösungen zur Verfügung gestellt werden. Beispiele stellen hier Gmail für eMail oder Office360 für Textbearbeitung dar.}
\newdualentry{paas}    % label
  {PaaS}
  {Platform as a Service}
  {Service Model für Clouds, bei dem der Anwenderin Umfelder zum Ausführen ihrer Applikationen zur Verfügung gestellt werden. Beispiele stellen hier Heroku als Umfeld für Ruby on Rails Applikationen oder OpenShift für diverse weitere Sprachen und Frameworks dar.}
\newdualentry{iaas}    % label
  {IaaS}
  {Infrastructure as a Service}
  {Service Model für Clouds, bei dem der Anwenderin grundlegende Ressourcen wie \gls{cpu}, \gls{ram} und Speicherplatz zur Verfügung gestellt werden. Beispiele sind hier \gls{aws} oder \gls{azure}.}
\newdualentry{gcp}    % label
  {GCP}
  {Google Cloud Platform}
  {Cloud Computing Angebot der Google Inc..}
\newdualentry{aws}    % label
  {AWS}               % abbreviation
  {Amazon Web Services}
  {Cloud Computing Angebot der Amazon.com Inc..}
\newdualentry{ec2}    % label
  {EC2}               % abbreviation
  {Elastic Compute Cloud}
  {\gls{aws} Angebot für auf Rechenkapazität optimierte Virtuelle Maschinen.}
\newdualentry{nist}    % label
  {NIST}               % abbreviation
  {National Institute of Standards and Technology}
  {Amerikanische Bundesbehörde, die unter Anderem eine zentrale Anlaufstelle für Standards in der IT darstellt.}
\newdualentry{api}    % label
  {API}               % abbreviation
  {Application Programming Interface}
  {Nicht-graphische Schnittstelle zu einer Applikation, über die Zugriff auf die Funktionalitäten angeboten wird.}
\newdualentry{cpu}    % label
  {CPU}               % abbreviation
  {Central processing unit}  % long form
  {Der (Haupt-)Prozessor eines Computers ist für die effektive Berechnung von Programmen, beispielsweise den Implementationen von Algorithmen, zuständig.}
\newdualentry{gpu}    % label
  {GPU}               % abbreviation
  {Graphics processing unit}  % long form
  {Ein Grafikprozessor stellt eine spezielle Unterart an Prozessoren dar, die auf die Berechnung der Ausgabe am Bildschirm spezialisiert sind, dar. Da in der Entwicklung solcher Grafikprozessoren besonderer Wert auf hohe Parallelisierung von Berechnungen gelegt wurde, besitzen GPUs nach heutigem Stand bei solchen Berechnungen Vorteile gegenüber herkömmlichen \gls{cpu}s.}
\newdualentry{ram}    % label
  {RAM}               % abbreviation
  {Hauptspeicher}  % long form
  {Im random access memory werden (Zwischen-)Ergebnisse von Berechnungen abgelegt. Es handelt sich hierbei um einen flüchtigen Speicher, dessen Inhalt bei Verlust der Energieversorgung verloren geht.}
\newdualentry{cgroups}    % label
  {cgroups}               % abbreviation
  {control groups}  % long form
  {Mit Kernel Verion 2.6.24 24 (Januar 2008) \cite{kernel264} eingeführtes Feature, welches ein Ressourcenmanagement für Prozesse, beispielsweise die Beschränkung der Menge an \gls{ram} ermöglicht.}
\newdualentry{ssh}    % label
  {SSH}               % abbreviation
  {Secure Shell}  % long form
  {Netzwerkprotokoll auf dem Application layer zum Aufbau einer verschlüsselten Verbindung zwischen zwei Rechnern. Gemeinhein auch die Bezeichnung für das Programm OpenSSH, mit dem ein remote  Zugriff auf Rechner möglich ist.}
\newdualentry{http}    % label
  {HTTP}               % abbreviation
  {Hypertext Transfer Protocol}  % long form
  {Netzwerkprotokoll auf dem Application layer zur zustandslosen Übertragung von Daten. Weitverbreitet für den Zugriff auf Websites, kann es jedoch beispielsweise auch zum Angebot von \gls{rest}-Services genutzt werden. Mit der Erweiterung HTTPS ist HTTP über eine mit TLS gesicherte Verbindung über gemeint.}
\newdualentry{bitkom}    % label
  {BITKOM}               % abbreviation
  {Bundesverband Informationswirtschaft, Telekommunikation und neue Medien}
  {Brachen- und Lobbyverband der deutschen Informations und Tekekommunikationsbranche.}
\newdualentry{iana}    % label
  {IANA}               % abbreviation
  {Internet Assigned Numbers Authority}
  {Abteilung der Internet Corporation for Assigned Names and Numbers, die unter Anderem mit der Zuweisung reservierter Ports betraut ist.}
\newdualentry{url}    % label
  {URL}               % abbreviation
  {Uniform resource locator}
  {Identifikator einer spezifischen Ressource, die im Allgemeinen über das Netzwerk erreichbar ist. In der URL ist weiterhin ein Hinweis auf die Art der Kommunikation enthalten, so definiert die URL https://schrimpf.ch, dass ein Server mit dem Domain Namen "`schrimpf.ch"' über \gls{http} angesprochen werden soll.}
\newdualentry{rest}    % label
  {REST}               % abbreviation
  {Representational State Transfer}
  {Programmierparadigma für die Implementation von Webservices. Es bestehen verschiedene Vorgaben, wie Zustandslosigkeit zwischen Anfragen, Idempotenz oder auch Sicherheit einzelner Anfragen. Endpunkte innerhalb der Kommunikation wer als Ressourcen bezeichnet, sie können beispielsweise als \gls{url} über \gls{http} angesprochen werden.}
\newdualentry{ssd}    % label
  {SSD}               % abbreviation
  {Solid-state drive}
  {Persistenter Datenspeicher, der im Gegensatz zu klassichen Festplatten ohne mechanische Teile auskommt und wesentlich höhere Datentransferraten erreicht.}
\newdualentry{poc}    % label
  {PoC}               % abbreviation
  {Proof of Concept}
  {Test der grundsätzlichen Machbarkeit eines Konzeptes oder einer Anwendung.}
\newdualentry{gce}    % label
  {GCE}
  {Google Compute Engine}
  {\gls{iaas} Angebot von \gls{gcp}.}
\newdualentry{rtt}    % label
  {RTT}
  {Round Trip Time}
  {Zeit, die ein Datenpaket benötigt, um den Weg zu einem Ziel und zurück zum Ausgangspunkt zurückzulegen.}
\newdualentry{mips}    % label
  {MIPS}
  {million instructions per second}
  {Masseinheit für die Leistung von Prozessoren. Hierfür wird die Menge an Maschineninstruktionen pro Sekunde, die die \gls{cpu} verarbeiten kann, gemessen.}
\newdualentry{icmp}    % label
  {ICMP}
  {Internet Control Message Protocol}
  {Auf IP basierendes Protokoll zur Kommunikation von Netzwerkgeräten zum Austausch von Informationen und Fehlermeldungen.}
\newdualentry{flops}    % label
  {FLOPS}
  {floating point operations per second}
  {Masseinheit für die Leistung von Prozessoren. Hierbei wird die Anzahl an Gleieitkommazahl Operationen, die die \gls{cpu} verarbeiten kann, gemessen.}
\newdualentry{tcp}    % label
  {TCP}
  {Transmission Control Protocol}
  {Verbindungsorientiertes Transport layer Protokoll zur zuverlässigen Übertragung von Datenpaketen in einem Netzwerk.}
\newdualentry{ip}    % label
  {IP}
  {Internet Protocol}
  {Zustandsloses Network layer Protokoll zur Übertragung von Datenpaketen in einem Netzwerk.}
\newdualentry{json}    % label
  {JSON}               % abbreviation
  {JavaScript Object Notation}  % long form
  {Menschenlesbare Notation zum Austausch strukturierter Daten.}
\newdualentry{xml}    % label
  {XML}               % abbreviation
  {Extensible Markup Language}  % long form
  {Markup Sprache zum Austausch strukturierter Daten.}
\newdualentry{bash}    % label
  {Bash}               % abbreviation
  {Bourne-again shell}  % long form
  {Kommandozeilensprache zur Interaktion von Nutzer und unixoidem Betriebssystem. Durch die Verwendung in Scripts können mithilfe von Bash viele iterative Arbeitsabläufe automatisiert werden.}
%%%%%%%%%%%%%%%%%%%%%%%%%%%%%%%%%%%%%%%%%%%%%%%%%%%%%%%%%%%%%%%%%%%%%%%%%%%%%%
% Dependency Injection
% Python
% Kernel
\newglossaryentry{postgres}{%
  name={PostgreSQL},
  description={SQL basierendes objektrelationales Datenbanksystem.}
}
\newglossaryentry{yo}{%
  name={Yeoman},
  description={Aus dem Umfeld von Google stammendes Opensource-Generator-Framework für die Erstellung von Webapplikationen.}
}
\newglossaryentry{spring}{%
  name={Spring},
  description={Weitverbreitetes Java-Framework zur Erstellung von Enterprise-Applikationen.}
}
\newglossaryentry{angular}{%
  name={AngularJS},
  description={2009 von Google veröffentlichtes Opensource-Framework zur Erstellung von Websites mithilfe von intensivem JavaScript Einsatz.}
}
\newglossaryentry{jhipster}{%
  name={jHipster},
  description={Auf dem \gls{yo} Generator-Framework basierendes Opensource-Werkzeug zur Generation von Webapplikationen. Als Backend-Komponente wird hierbei \gls{java} mit dem \gls{spring}-Framework und als Frontend-Komponente \gls{angular} verwendet.}
}
\newglossaryentry{openstack}{%
  name={OpenStack},
  description={Opensource Projekt für den Betrieb einer Cloud-Infrastruktur.}
}
\newglossaryentry{java}{%
  name={Java},
  description={1995 entwickelte objektorientierte Programmiersprache. Java erfreut sich grosser Popularität und besitzt zahlreiche Frameworks zur Vereinfachung der Entwicklung.}
}
\newglossaryentry{jclouds}{%
  name={jclouds},
  description={Von der Apache Software Foundation unterstütztes Opensource-Framework zur Abstraktion der Interaktion von \gls{java}-Programmen und Cloud-Providern..}
}
\newglossaryentry{cli}{%
  name={cloud-init},
  description={Programm mit dem verschiedene grundlegende Einstellungen eines Linux-Systems, beispielsweise hostname oder locale, beim start einer \gls{vm} über eine Textdatei vorgenommen werden können.}
}
\newglossaryentry{slack}{%
  name={Slack},
  description={Chatprogramm zur Team-Kommunikation mit Clients für die gängigsten Betriebssysteme und umfassenden Webservice \gls{api}.}
}
\newglossaryentry{iperf}{%
  name={iperf},
  description={Tool zum Test von \gls{tcp} und UDP Verbindungen. Über einen konfigurierbaren Zeitraum werden beispielsweise jeweils die maximal mögliche Menge an Daten übertragen, um die Bandbreite der Verbindung zu testen.}
}
\newglossaryentry{sysbench}{%
  name={sysbench},
  description={Programm zur Messung verschiedener Systemeigenschaften, die für den Betrieb eines Datenbankservers relevant sind bzw. sein können. Durch sysbench können beispielsweise die Perfomance von \gls{cpu}, \gls{ram}, POSIX Threads oder der Festplatte gemessen werden.}
}
\newglossaryentry{ruby}{%
  name={Ruby},
  description={1995 erschienene höhere objektorientierte Programmiersprache mit dynamischer Typisierung.}
}
\newglossaryentry{switch}{%
  name={SWITCH},
  description={1987 gegründete Stiftung die als Technologiedienstleister der Schweizer Hochschulen die Bereitstellung von Dienstleistungen wie Cloud Speicher oder \glspl{vm}, Netzwerken und Ähnlichem übernimmt.}
}
\newglossaryentry{gentoo}{%
  name={Gentoo},
  description={Eine nach dem schnellsten Schwimmer unter den Pinguinen, dem Eselspinguin (englisch gentoo penguin), benannte Linux \gls{distro}, deren besonderes Ziel es ist, der Benutzerin möglichst viele Möglichkeiten zur Kontrolle über ihr System, beispielsweise bei der Installation von Software, zu gewähren.}
}
\newglossaryentry{deployment}{%
  name={Deployment},
  description={Verteilung einer Applikation und ggf. ihrer Abhängigkeiten in einem gegebenen Umfeld.}
}
\newglossaryentry{azure}{%
  name={Azure},
  description={Cloud Computing Angebot der Microsoft Corporation.}
}
\newglossaryentry{yml}{%
  name={YAML},
  description={Markup Language, mit der eine Daten Serialisierung in möglichst menschenlesbarer Form ohne den grossen Overhead vergleichbarer Sprachen wie XML vorgenommen werden soll. Der Name ist ein recursives Akronym für "`YAML Ain't Markup Language"' \cite{yml2} (früher "`Yet Another Markup Language"' \cite{yml1}).}
}
\newglossaryentry{git}{%
  name={git},
  description={Tool zur verteilten Versionsverwaltung. Ehemals für die Verwaltung des Linux Kernels entwickelt, besitzt es heute grosse Popularität in der gesammten Softwareentwicklung und ein ausgeprägtes Ökosystem.}
}
\newglossaryentry{rsa}{%
  name={RSA},
  description={Von Ronald L. Rivest, Adi Shamir und Leonard Adleman entwickeltes asymetrisches (public/private-key) Kryptosystem.}
}
\newglossaryentry{systemd}{%
  name={systemd},
  description={Tool, welches ehemals für die Steuerung des Linux Initialisierungsprozesses entwickelt wurde, jedoch im Laufe der Zeit immer weitere Funktionalitäten bot und somit Gegenstand einer intensiven Diskussion in der Linux Community wurde. Mit systemd ist es möglich Programme als sogenannte Services im Hintergrund auszuführen und zu überwachen.}
}
\newglossaryentry{distro}{%
  name={Distribution},
  plural={Distributionen},
  description={Im Kontext der Arbeit bezogen auf verschiedene Distributionen des Linux Betriebssystemes. Zusammenstellung von Softwarepaketen zu einem vollständigen Betriebssystem. Häufige Unterschiede sind der verwendete Package Manager, das Init system und ähnliche redundante Implementationen der gleichen grundlegenden Funktionalitäten.}
}

\begin{document}
%\bibliographystyle{plainnat}
%\bibliographystyle{alphadin}
\bibliographystyle{alphadin}


\title{Bachelorarbeit}
\author{Raffael Santschi}


%%%%%%%%%%%%%%%%%%%%%%%%%%%%%%%%%%%%%%%%%%%%%%%%%%%%%%%%%%%%%%%%%
%
% Project     : Bachelorarbeit
% Title       : Machbarkeitsanalyse für eine ressourcenorientierte Schnittstelle zur Verarbeitung grundlegender Probleme der Informatik
% File        : titlepage.tex Rev. 01
% Date        : 01.03.2015
% Author      : Raffael Santschi
%
%%%%%%%%%%%%%%%%%%%%%%%%%%%%%%%%%%%%%%%%%%%%%%%%%%%%%%%%%%%%%%%%%

\begin{titlepage}

% Logo
\ThisTileWallPaper{\paperwidth}{\paperheight}{images/logos/SoE.pdf} % {images/logos/*.pdf}
% Wählen Sie aus folenden pdf Files: ICP, IDP, IEFE, IMES, IMPE, IMS, INE, InES, InIT, KSR, SoE, ZAMP, ZAV, ZIL, ZPP, ZSN

\begin{minipage}[b]{0.117\textwidth}
\hskip 0.05cm
\end{minipage}
\begin{minipage}[b]{0.91\textwidth}
\begin{tiny}.\end{tiny}\vskip 2.8cm
	{\huge
	
	% Projekt Name
	\textbf{\underline{Bachelorarbeit Informatik}}\\
	
	% Projekt Titel
	
	\myTitle
	\vskip 0.5cm}
	
	\begin{minipage}[b]{0.27\textwidth}
	\hrule\vskip 0.5cm
		\textbf{Autor}\\
		\\
		\\
		\\
	\end{minipage}
	\begin{minipage}[b]{0.03\textwidth}
	\hskip 0.5cm
	\end{minipage}
	\begin{minipage}[b]{0.7\textwidth}
	\hrule\vskip 0.5cm
		Raffael Santschi\\
		santsraf@students.zhaw.ch\\
		Student im 8. Semester\\
		\\
	\end{minipage}
	
	\begin{minipage}[b]{0.27\textwidth}
	\hrule\vskip 0.5cm
		\textbf{Hauptbetreuung}\\
		\\
		\\
	\end{minipage}
	\begin{minipage}[b]{0.03\textwidth}
	\hskip 0.5cm
	\end{minipage}
	\begin{minipage}[b]{0.7\textwidth}
	\hrule\vskip 0.5cm
		Alain M. Lafon\\
		lafo@zhaw.ch\\
		\\
	\end{minipage}

	\begin{minipage}[b]{0.27\textwidth}
	\hrule\vskip 0.5cm
		\textbf{Experte}\\
		\\
		\\
	\end{minipage}
	\begin{minipage}[b]{0.03\textwidth}
	\hskip 0.5cm
	\end{minipage}
	\begin{minipage}[b]{0.7\textwidth}
	\hrule\vskip 0.5cm
		Silvan Spross\\
		silvan.spross@gmail.com\\
		\\
	\end{minipage}
	
	\begin{minipage}[b]{0.27\textwidth}
	\hrule\vskip 0.5cm
		\textbf{Abgabedatum}
	\end{minipage}
	\begin{minipage}[b]{0.03\textwidth}
	\hskip 0.5cm
	\end{minipage}
	\begin{minipage}[b]{0.7\textwidth}
	\hrule\vskip 0.5cm
		26.06.2015
	\end{minipage}

\end{minipage}
\vskip 0.5cm


\end{titlepage}

\setcounter{page}{1}
%\include{content/Kontakt}
%%%%%%%%%%%%%%%%%%%%%%%%%%%%%%%%%%%%%%%%%%%%%%%%%%%%%%%%%%%%%%%%%
%
% Project     : Bachelorarbeit
% Title       : Machbarkeitsanalyse für eine ressourcenorientierte Schnittstelle zur Verarbeitung grundlegender Probleme der Informatik
% File        : abstract.tex Rev. 01
% Date        : 01.03.2015
% Author      : Raffael Santschi
%
%%%%%%%%%%%%%%%%%%%%%%%%%%%%%%%%%%%%%%%%%%%%%%%%%%%%%%%%%%%%%%%%%

\thispagestyle{empty}


\newpage
\thispagestyle{empty}
\chapter*{Abstract}\label{abstract}
Ziel dieser Arbeit war es zu analysieren, ob eine generische Schnittstelle für grundlegende Probleme der Informatik möglich ist. Die Schnittstelle sollte keinerlei Kenntnisse 
der theoretischen Informatik oder der jeweiligen Probleme voraussetzen. Dazu wurden fünf Probleme mit hoher Laufzeitkomplexität aus 'Computers and Intractability: A Guide to the Theory of 
NP-Completeness' von Micheal Garey und David S. Johnson ausgewählt. Die Problemfelder wurden auf ihre Ein- und Ausgabeparameter analysiert und die dazugehörigen 
Algorithmen wurden betrachtet. Im späteren Verlauf der Arbeit wurde noch ein sechstes Problem dazugenommen, welches sehr ähnlich zu einem bereits ausgewählten Problem war. Dies 
um zu schauen, wie sich die beiden Probleme bei der Implementierung im Gegensatz zu den anderen Problemen verhalten.\\

Beim Erstellen des Konzept wurden die Gemeinsamkeiten des Berechnungsablaufs betrachtet. Um mehr Freiheiten zu haben und Benutzerfreundlichkeit zu garantieren, wurde die Nutzer- und 
Algorithmus-Welt voneinander entkoppelt. Dies bot die Möglichkeit jeweils eine andere Domänensprache zu verwenden. Zwischen den beiden Welten kamen pre- und post-Aktionen zum Einsatz, 
welche die Datenaufbereitung für den Algorithmus bzw. den Nutzer durchführten.\\

Vor der Umsetzung wurde eine Nutzwertanalyse zur Auswahl eines geeigneten Datenbanksystems durchgeführt. Nach der Vorselektierung standen die relationalen, objektorientierten und 
dokumentorientierten Datenbanken zur Auswahl. Beim Vergleich wies das dokumentorientierte Datenbanksystem einige Vorzüge auf, wie seine Flexibilität, welche eine schnelle und unkomplizierte 
Erweiterung bzw. Anpassung ermöglicht.\\

Als Prototyp wurde ein REST API implementiert, welches die Funktionalität bereitstellt, Berechnung von den sechs verschiedenen Problemen zu erstellen. Hinter einem dieser Probleme wurde ein 
Algorithmus angebunden, womit der ganze Prozess durchgespielt werden konnte. Bei den anderen Problemen wurde anhand der Recherche Ein- und Ausgabeschemata der Algorithmen definiert. 
Zusätzlich wurde zu jedem Problem ein Validator geschrieben, welcher überprüft, ob ein Resultat gültig ist oder nicht.\\

Das Konzept für die Schnittstelle konnte für alle sechs Probleme angewandt und der Ablauf generisch gehalten werden. Die Machbarkeitsstudie ist als erfolgreich zu betrachten. Das 
Auswahlverfahren der Probleme hat sich bewährt, die Probleme zeigten unterschiedliche Ausprägungen. Einige Probleme könnten mit dem gleichen generische Algorithmus gelöst werden. Zwei 
andere Probleme, welche beide aus dem gleichen Bereich stammen, benötigen sehr unterschiedliche Herangehensweisen. Die Schnittstelle bietet genug Flexibilität um ganz unterschiedliche 
Probleme zu behandeln und verschiedene Algorithmen anzusteuern.

%%%%%%%%%%%%%%%%%%%%%%%%%%%%%%%%%%%%%%%%%%%%%%%%%%%%%%%%%%%%%%%%%
%
% Project     : Bachelorarbeit
% Title       : Machbarkeitsanalyse für eine ressourcenorientierte Schnittstelle zur Verarbeitung grundlegender Probleme der Informatik
% File        : abstract.tex Rev. 01
% Date        : 01.03.2015
% Author      : Raffael Santschi
%
%%%%%%%%%%%%%%%%%%%%%%%%%%%%%%%%%%%%%%%%%%%%%%%%%%%%%%%%%%%%%%%%%

\thispagestyle{empty}


\newpage
\thispagestyle{empty}

\chapter*{Bestätigung der Selbstständigkeit} 
\markboth{\MakeUppercase{Bestätigung der Selbstständigkeit}}{}

Hiermit bestätigt die oder der Unterzeichnende, dass die Bachelorarbeit mit dem Thema "`\textbf{\myTitle}"' gemäss freigegebener Aufgabenstellung mit Freigabe vom \releaseDate ohne jede 
fremde Hilfe im Rahmen der gültigen Reglements selbstständig ausgeführt wurde.\\ [2cm]
..............................................\\ 
Raffael Santschi

\newpage{}
%\includepdf{images/Erklaerung_BA.pdf} % Entsprechendes auskommentieren
% \newpage

%Inhaltsverzeichnis
\tableofcontents
\newpage



%\textbf{}
%\setcounter{page}{1}
%\pagenumbering{arabic}

%\include{content/howtoLaTeX} % Für das Schlussdokument auskommentieren

%%%%%%%%%%%%%%%%%%%%%%%%%%%%%%%%%%%%%%%%%%%%%%%%%%%%%%%%%%%%%%%%%
%
% Project     : Bachelorarbeit
% Title       : Machbarkeitsanalyse für eine ressourcenorientierte Schnittstelle zur Verarbeitung grundlegender Probleme der Informatik
% File        : uebersicht.tex Rev. 01
% Date        : 01.03.2015
% Author      : Raffael Santschi
%
%%%%%%%%%%%%%%%%%%%%%%%%%%%%%%%%%%%%%%%%%%%%%%%%%%%%%%%%%%%%%%%%%

\chapter{Projektübersicht}\label{chap.projektuebersicht}
Die Übersicht dient dem Zweck einen generellen Überblick über das Dokument zu verschaffen. Sie beinhaltet die Ausganglage, das Ziel dieser Arbeit, die Aufgabenstellung, die erwarteten Resultate und die Nicht-Ziele. Zusätzlich wird der Aufbau dieses Dokumentes erklärt.

\section{Ausgangslage}\label{ausganglage}
Bei einigen Problemen der Informatik kann deterministisch die exakte Lösung nicht in Polynomialzeit berechnet werden. Um in sinnvoller Zeit eine brauchbare Lösung zu erhalten, müssen diese Probleme im Allgemeinen mit Hilfe von Approximierungsalgorithmen angegangen werden. Zu dieser Kategorie gehören zum Beispiel das Problem des Handlungsreisenden oder das Rucksack Problem. Praktische Anwendung finden solche Probleme beispielsweise in der Logistik, bei der Routenplanung und beim Verladen von Fracht.

Die bekannten Approximierungsalgorithmen haben verschiedene Ausprägungen und auch unterschiedliche Eingabeparameter. Es gibt keine Schnittstelle für die Benutzung von diesen Algorithmen, welche keine detaillierte Kenntnis der darunterliegenden Probleme und Algorithmen erfordert. Eine Schnittstelle mit dieser Eigenschaft kann die Handhabung solcher Probleme enorm erleichtern.

\section{Ziele der Arbeit}\label{ziele}
In der Arbeit soll ein Konzept für eine Schnittstelle zur Lösung verschiedener grundlegender Probleme der Informatik erarbeitet werden. Diese Schnittstelle soll basierend auf den Erkenntnissen einer Analyse über die Gemeinsamkeiten dieser Approximierungsalgorithmen aufgebaut werden. Dadurch soll es einem Benutzer ermöglicht werden seine jeweiligen Probleme, beispielsweise die effizienten Verpackung von Gegenständen, ohne ein Verständnis der darunterliegenden Probleme der Informatik anzugehen.

Bei der Erarbeitung der Schnittstelle stehen eine geeignete Persistenz-Lösung und sowie Datenstrukturen für Ein- und Ausgabe im Vordergrund.

\section{Aufgabenstellung}\label{aufgabenstellung}
Folgende Punkte werden in der Semesterarbeit behandelt:
\begin{enumerate}
\item Recherche von real auftretenden Problemen, welche ausschliesslich durch den Einsatz von Algorithmen mit hoher Laufzeitkomplexität gelöst werden können. Einarbeiten und Analyse in die ausgewählten Algorithmen.
\item Ist-Analyse der verwendeten Datenstrukturen für die Algorithmen.
\item Anforderungsanalyse einer Schnittstelle für die ausgewählten Algorithmen.
\item Erarbeiten eines Konzeptes für die Implementierung der Schnittstelle sowie einer zugehörigen Persistenz-Schicht.
\item Implementierung eines Prototypen für die Schnittstelle und der Persistenz-Schicht.
\item Automatisiertes Testen der Schnittstelle.
\end{enumerate}

\section{Erwartete Resultate}\label{erwartete_resultate}
Folgende Punkte werden als Resultate der Semesterarbeit erwartet:
\begin{enumerate}
\item Übersicht der Probleme mit den dazugehörigen Algorithmen und Beschreibung der Algorithmen mit ihren Kerneigenschaften.
      \begin{enumerate}
        \item Ausführungen zum Einfluss der Parameter der jeweiligen Probleme auf die Komplexität.
      \end{enumerate}
\item Übersicht über die verwendeten Datenstrukturen als Input / Output der Probleme.
\item Anforderungskatalog an die Schnittstelle.
\item Konzept einer generellen Schnittstelle zur Lösung der komplexen Probleme und Datendiagramm des Datenspeichers.
\item Prototypische Implementation der Schnittstelle und des Datenspeichers.
\item Automatische Tests mit dem dazugehörigen Testprotokoll.
\end{enumerate}

\section{Nicht-Ziele}\label{nicht_ziele}
Folgende Punkte wurden mit dem Auftraggeber als Nicht-Ziele definiert und sind somit nicht Teil dieses Projekts:
\begin{itemize}
\item Der Sicherheitsaspekt einer Schnittstelle wird in diesem Projekt nicht behandelt.
\item Es werden keine Algorithmen implementiert.
\end{itemize}
\todo{Erweitern Nicht-Ziele}

\section{Dokumentstruktur}\label{document_structure}
\todo{Überfliegen und anpassen}
Dieses Dokument spiegelt die geleistete Arbeit wieder und ist in einzelne Kapitel unterteilt.
\begin{itemize}
\item Projektplanung: Schritte für die Erstellung des Projektplanes und der Risikoanalyse
\item Einleitung: Beischreibung der wichtigsten verwendeten Begriffe und Theorien, welche für das Verstehen der Arbeit notwendig sind
\item Analyse und Auswahl der Probleme: Erläuterung der Problemauswahl und Beschreibung der einzelnen Probleme
\item Anforderungsdokument: System- und Kontextabgrenzung, \glossarmark{Stakeholder}, getroffene Annahmen und der Anforderungskatalog mit Use Cases und Anforderungen
\item Architektur: Übersicht über das ganze System, Nutzwertanalyse der verschiedenen Lösungsvarianten und die Architekturbeschreibung der Schnittstelle
\item Umsetzung: Beschreibung der Entwicklungsumgebung, Umsetzung von App und Backend
\item Tests: Erläuterung der Test-Methoden und das Abnahme Protokoll
\end{itemize}

Im Anhang sind das Glossar, in welchem die blau markierten Begriffe im Dokument erklärt werden, und alle Verzeichnisse zu finden. Falls es zu einem Begriff eine gängige Abkürzung gibt, wird diese beim ersten Auftauchen des Wortes in Klammern geschrieben und danach verwendet, im Glossar finden sich dann beide Einträge. Die grün markierten Begriffe sind in der \nameref{chap.einleitung} erklärt.
%%%%%%%%%%%%%%%%%%%%%%%%%%%%%%%%%%%%%%%%%%%%%%%%%%%%%%%%%%%%%%%%%
%
% Project     : Bachelorarbeit
% Title       : Machbarkeitsanalyse für eine ressourcenorientierte Schnittstelle zur Verarbeitung grundlegender Probleme der Informatik
% File        : projektplanung.tex Rev. 01
% Date        : 01.03.2015
% Author      : Raffael Santschi
%
%%%%%%%%%%%%%%%%%%%%%%%%%%%%%%%%%%%%%%%%%%%%%%%%%%%%%%%%%%%%%%%%%

\chapter{Projektplanung}\label{chap.projektplanung}
Dieses Kapitel handelt von der Projektplanung und den verschiedenen Arbeitspaketen für dieses Projekt.

\section{Meilensteine}\label{meilensteine}
Folgende Meilensteine wurden für dieses Projekt festgelegt:

\begin{table}[ht]
\centering
  \begin{tabular}{ l | r }
	\hline
	\rowcolor{gray}
	\textbf{Projektstart}			&	\textbf{23.01.2015}\\ \hline
	Anforderungsdokument fertig		&	28.02.2015	\\ \hline
	Wissen über Probleme aufgebaut		&	19.04.2015	\\ \hline
	Erster \gls{vertikaler_durchstich}		& 	26.04.2015	\\ \hline
	Architektur festgelegt			&	03.05.2015	\\ \hline
	Prototyp fertig				&	29.05.2015	\\ \hline
	Dokumentation fertig			&	12.06.2015	\\ \hline
	Dokumentation korrigiert			&	26.06.2015	\\ \hline
	Präsentation					&	08.07.2015 \\ \hline
  \end{tabular}
   \caption{Meilensteine}\label{table:milestones}
\end{table}

\section{Arbeitspakete}\label{arbeitspakete}
Das Projekt beinhaltet sieben Arbeitspakete:
\begin{itemize}
\item Planung
\item Analyse und Auswahl der Probleme
\item Requirement Engineering
\item Konzept
\item Umsetzung Prototyp
\item Testing
\item Dokumentation
\end{itemize}

\subsection{Planung}\label{planung}
In der Planungsphase wird geschaut, was in dem Projekt erreicht werden muss und wie diese Tätigkeiten auf die vorhandene Zeit aufgeteilt werden. Es wird auch das erste Mal mit dem 
\gls{stakeholder} geredet und erste Abmachungen getroffen.

\subsection{Analyse und Auswahl der Probleme}\label{analyse_auswahl_probleme}
Ein sehr wichtiges Paket ist die Analyse und die Auswahl der Probleme. Es ist wichtig, dass die Probleme möglichst vielfältig gewählt werden und sie gut analysiert werden, damit das Konzept 
mit den erhobenen Daten sauber geplant werden kann.

\subsection{Requirement Engineering}\label{rqe}
Bei der Erstellung eines neuen Systems ist es immer wichtig, dass die Grundanforderungen bekannt sind. Um die Anforderungen zu erfassen, wird der \gls{stakeholder} befragt, was seine Wünsche 
sind. Oft werden bei der Anforderungsanalyse einige Anforderungen nicht aufgelistet, sondern einfach vorausgesetzt, sogenannte \glspl{basisfaktor}. Diese Anforderungen müssen dann vom 
Entwickler erfasst werden. Der Anforderungskatalog wird nach der Vollendung nochmals mit dem \gls{stakeholder} in einem Review angeschaut (siehe dazu auch \cite{req_eng_book}).

\subsection{Konzept}\label{ref_backend}
Das Hauptziel dieser Arbeit ist ein Konzept, welches generisch ist und viele verschiedene Probleme mit wenig Aufwand verarbeiten kann. Die Architektur muss gut durchdacht sein und die 
Möglichkeiten müssen gegeneinander abgewägt werden. Schliesslich muss eine Konzept entstehen, welches in einem Prototypen umsetzbar ist.

\subsection{Umsetzung Prototyp}\label{eng_prototyp}
In diesem Arbeitspaket werden die Anforderungen mit dem festgelegten Konzept umgesetzt. Die Schnittstelle wird entworfen, die ersten Tests werden durchgeführt und Unstimmigkeiten in den 
Anforderungen werden mit dem \gls{stakeholder} geklärt.

\subsection{Testing}\label{testing}
Das Projekt benötigt automatische Tests, welche erstellt und überprüft werden müssen. Die Tests sollten einen Grossteil des Projekts abdecken und bei einer Anpassung oder Erweiterung des 
Codes Sicherheit bieten.

\subsection{Dokumentation}\label{dokumentation}
Die Dokumentation wird während des ganzen Projekts hindurch aktuell gehalten. Für das Erfassen dieses Dokuments wird \LaTeX\ und das \LaTeX-Template der ZHAW \cite{zhaw_latex_template} mit ein 
paar kleinen Anpassungen verwendet.

\section{Zeitplan}\label{zeitplan}
Der Zeitplan gibt eine grobe Übersicht, wann an dem Projekt gearbeitet werden kann und wann die verschiedenen Tätigkeiten fertig sein sollten. Die Angaben sind nur Richtwerte, da neben dem 
Projekt noch berufliche Verpflichtungen und andere Tätigkeiten Zeit benötigen.

\subsection{Geplante Abwesenheiten}
\begin{table}[ht]
\centering
  \begin{tabular}{ l | r }
	\hline
	\rowcolor{gray}
	\textbf{Abwesenheit}					&	\textbf{Start - Ende}	\\ \hline
	Ferien								&	07.02.2015 - 15.02.2015	\\ \hline
	Seminararbeiten						&	06.04.2015 - 19.04.2015	\\ \hline
	Modulprüfungen und Vorträge Seminararbeit		&	15.06.2015 - 28.06.2015	\\ \hline
  \end{tabular}
   \caption{Geplante Abwesenheiten}\label{table:holidays}
\end{table}

\newpage 

\subsection{Projektplan}\label{projektplan}
Der Zeitplan basierend auf den Arbeitspaketen und den geplanten Abwesenheiten sieht wie folgt aus:
\begin{figure}[h]
\centering
\includegraphics[scale=0.422]{images/project/projectplan.png}
\caption[Projektplan]{Projektplan \selfmade{}}
\label{fig:psp}
\end{figure}

\FloatBarrier

\subsection{Zeitschätzung auf Arbeitspaketebene}
\begin{table}[ht]
\centering
  \begin{tabular}{ l | c | c }
	\hline
	\rowcolor{gray}
	\textbf{Arbeitspaket}					&	\textbf{Schätzung (h)}	& \textbf{Tatsächlich (h)}	\\ \hline
	Requirement Engineering					&	20			& 30	\\ \hline
	Reale Optimierungsprobleme suchen			&	50			& 70	\\ \hline
	Wissensaufbau Algorithmen				&	70			& 50	\\ \hline
	Konzept Planung						&	60			& 70	\\ \hline
	Prototyp Entwicklung					&	50			& 80	\\ \hline
	Tests								&	10			& 20	\\ \hline
	Dokumentation						&	120			& 100	\\ \hline \hline
	Total								&	380			& 420	\\ \hline
  \end{tabular}
   \caption{Zeitschätzung auf Arbeitspaketebene}\label{table:time_estimation}
\end{table}

\FloatBarrier

\subsubsection{Erklärung der Abweichungen}
Das Requirement Engineering und Testen wurde unterschätzt, die Zeit floss jedoch vor allem in Detailarbeiten. Bei der Suche von realen Optimierungsprobleme wurde mehr Zeit beansprucht, als
ursprünglich geplant. Während der Suche konnte jedoch bereits Wissen über die Algorithmen aufgebaut werden, was dazu führte, dass in diesem Bereich weniger Zeit benötigt wurde. Als 
Evaluation des gewählten Konzept wurde ein \gls{vertikaler_durchstich} durchgeführt, was sich in der benötigten Zeit widerspiegelt. Die Entwicklung des Prototyps war aufwändiger 
als gedacht, was daran lag, dass sechs statt fünf Probleme umgesetzt wurden. Die Dokumentation benötigte nicht so viel Zeit, jedoch ist dieser Punkt auch etwas ungenau zu 
messen, da in den anderen Paketen bereits Dokumentation entsteht.

%%%%%%%%%%%%%%%%%%%%%%%%%%%%%%%%%%%%%%%%%%%%%%%%%%%%%%%%%%%%%%%%%
%
% Project     : Bachelorarbeit
% Title       : Machbarkeitsanalyse für eine ressourcenorientierte Schnittstelle zur Verarbeitung grundlegender Probleme der Informatik
% File        : einleitung.tex Rev. 01
% Date        : 01.03.2015
% Author      : Raffael Santschi
%
%%%%%%%%%%%%%%%%%%%%%%%%%%%%%%%%%%%%%%%%%%%%%%%%%%%%%%%%%%%%%%%%%

\chapter{Einleitung}\label{chap.einleitung}
Die Einleitung dient dazu, wichtige Informationen zum Verständnis der Arbeit zu erläutern. Es werden die Komplexitätsklassen der Theoretischen Informatik und die verschiedenen Algorithmentypen erklärt.

\section{Komplexitätsklassen der Theoretischen Informatik}\label{cat_theo_inf}
In der Theoretischen Informatik wird zwischen verschiedenen Komplexitätsklassen unterschieden. In diesem Kapitel werden nur auf die Komplexitätsklassen NP, P, NP-schwer und NP-vollständig eingegangen, weitere Klassen wie zum Beispiel RP (Random Polynomial) oder ZPP (Zero-Error, Probabilistic, Polynomial) werden nicht erläutert, da sie nicht zum Verständnis der Arbeit notwendig sind. Die nachfolgenden Erklärungen sind aus \cite{hopcroft2011einfuehrung} und \cite{slides_p_np} abgeleitet.

\begin{figure}[h]
\includegraphics[scale=0.7]{images/einleitung/p_np_np-complete_np-hard.png}
\caption[Übersicht der Komplexitätsklassen ($P!=NP$ und $P=NP$)]{Übersicht der Komplexitätsklassen ($P \neq NP$ und $P=NP$) (Grafik entnommen aus \cite{pic_p_np})}
\label{fig:complexity_overview}
\end{figure}

Die Abbildung \ref{fig:complexity_overview} zeigt die Aufteilung der verschiedenen Komplexitätsklassen für die beiden Fälle $P \neq NP$ und $P=NP$. Momentan wird davon ausgegangen, dass $P \neq NP$ ist und somit die linke Aufteilung stimmt. Keine der beiden Behauptungen konnte bis jetzt bewiesen werden. Der letzte Versuche $P \neq NP$ zu beweisen wurde von Vinay Deolalikar im Jahr 2010 unternommen \cite{p_neq_np_paper}. In diesem Versuch wurden jedoch zwei Fehler entdeckt (siehe \cite{p_neq_np_paper_blog}) und somit bleibt der Beweis weiter offen.

\subsection{NP}\label{np}
Falls die Korrektheit einer Lösung zu einem Problem in polynomialer Zeit überprüft werden kann, also ein Polynomialzeit-Verifizierer vorhanden ist, liegt das Problem in NP.

\subsection{P}\label{p_complet}
Probleme, welche in polynomialer Zeit lösbar sind, gehören zu der Klasse der P-vollständigen Probleme. In polynomialer Zeit lösbar heisst, dass die Laufzeitkomplexität in einem Polynom mit der Form $n^k$, wobei n die Eingabelänge und k eine Konstante ist, dargestellt werden kann.

\subsection{NP-schwer}\label{np_hard}
Probleme, welche nicht in polynomialer Zeit lösbar sind, das heisst eine Laufzeitkomplexität höher als polynomial haben, zum Beispiel $k^n$ (exponentiell) oder $n!$ (faktoriell), gehören zu den NP-schweren Problemen. Um zu beweisen, dass das Problem NP-schwer ist, wird versucht ein anderes bekanntes NP-schweres Problem auf dieses Problem zu reduzieren. Mit diesem Beweis wird gezeigt, dass das Problem mindestens so schwer wie das andere Problem ist.

\subsection{NP-vollständig}\label{np_complet}
Stephen Cook hat mit dem Beweis der NP-Vollständigkeit des SAT-Problems (siehe \cite{cook_complexity}), die Ausgangslage für den Beweise der NP-Vollständigkeit vieler weiteren Probleme bereit gestellt. Damit ein Problem als NP-vollständig gilt, müssen folgende zwei Punkte erfüllt sein:
\begin{itemize}
	\item Ein Polynomialzeit-Verifizierer für das Problem ist vorhanden.
	\item Ein anderes bekanntes NP-vollständiges Problem ist auf dieses Problem reduzierbar.
\end{itemize}

\section{Algorithmentypen}\label{algo_types}

\subsection{Backtracking Algorithmen}\label{backtracking_algos}
Beim Backtracking geht es darum, sich einer Lösung eines Problems schrittweise zu nähern. Bei jedem neuen Schritt wird geprüft, ob es noch eine gültige Lösung darstellt. Falls dies nicht der Fall ist, wird der letzte Schritt rückgängig gemacht und es wird ein anderer Weg eingeschlagen. In \cite{backtracking} wird dieses Verfahren an Hand des Damenproblems aufgezeigt.

\subsection{Greedy Algorithmen}\label{greedy_algos}
Greedy Algorithmen liefern oft eine schnelle Lösung, welche aber meist nicht optimal ist. Die Algorithmen entscheiden bei jedem Schritt, was die aktuell beste Möglichkeit ist. Da sie nicht alle Möglichkeiten betrachten, finden sie oft nur ein lokales Minimum bzw. Maximum. In \autoref{fig:greedy_algo} ist in grün der Weg des Greedy Algorithmus zu sehen. Der Algorithmus entscheidet sich für die 12 auf der zweiten Ebene, da diese in dem Moment als beste Lösung gilt. Mit dem violetten Pfad könnte jedoch einen viel höheren Wert erzielen werden.

\begin{figure}[h]
\centering 
\includegraphics[scale=1]{images/einleitung/greedy_algo.png}
\caption[Suchablauf eines Greedy Algorithmus]{Suchablauf eines Greedy Algorithmus (Eigene Darstellung, Daten entnommen aus: \cite{pic_greedy_algo})}
\label{fig:greedy_algo}
\end{figure}

\subsection{Evolutionäre Algorithmen}\label{ea_algos}
Evolutionäre Algorithmen nähern sich einer optimalen Lösung an. Sie basieren auf Kombinationen (Generationen) von Objekten und einer Fitnessfunktion zur Bewertung der einzelnen Generationen. Ein Ablauf eines Evolutionären Algorithmus sieht meist wie folgt aus:
\begin{enumerate}
	\item Initialisierung: Die erste Generation wird meist zufällig erzeugt
	\item Iteration durch folgende Schritte bis die Lösung den gewünschten Wert erreicht hat:
     	\begin{enumerate}
		\item Evaluation: Mit Hilfe der Fitnessfunktion wird die erstellte Generation bewertet
         		\item Selektion: Auswahl von Objekten (Individuen) für die Rekombination
         		\item Rekombination: Erstellen einer neuen Generation durch die Kombination der ausgewählten Individuen
         		\item Mutation: Veränderung der Eigenschaften (Gene) der Nachfahren
      	\end{enumerate}
\end{enumerate}
Die Mutation und Rekombination kann positive, negative oder neutrale Eigenschaften haben. Wie in der Natur überlebt der Stärkste und somit wird die Lösung immer besser.

%%%%%%%%%%%%%%%%%%%%%%%%%%%%%%%%%%%%%%%%%%%%%%%%%%%%%%%%%%%%%%%%%
%
% Project     : Bachelorarbeit
% Title       : Machbarkeitsanalyse für eine ressourcenorientierte Schnittstelle zur Verarbeitung grundlegender Probleme der Informatik
% File        : probleme.tex Rev. 01
% Date        : 01.03.2015
% Author      : Raffael Santschi
%
%%%%%%%%%%%%%%%%%%%%%%%%%%%%%%%%%%%%%%%%%%%%%%%%%%%%%%%%%%%%%%%%%

\chapter{Analyse und Auswahl der Probleme \resultAssignment{[R1]}}\label{chap.problemauswahl}
In diesem Kapitel werden die verschiedene Probleme, welche für die Erstellung der Schnittstelle betrachtet wurden, und das dazugehörige Auswahlverfahren erläutert.

\section{Auswahl der Probleme}\label{cat_theo_inf}
Da es in der Welt unzählige Probleme mit hoher Laufzeitkomplexität gibt, musste ein geeignetes Auswahlverfahren gefunden werden. Mit diesem Verfahren sollten möglichst unterschiedliche 
Typen dieser Probleme für die Schnittstelle evaluiert werden. Dafür wurde eine Kategorisierung der Probleme gesucht, auf welche sich gestützt werden konnte. Immer wieder wird sich in der 
Informatik auf das Buch 'Computers and Intractability: A Guide to the Theory of NP-Completeness' \cite{garey1979computers} von Micheal Garey und David S. Johnson bezogen. Laut einer 
Studie von CiteSeer war es im Jahr 2006 das meist zitierte Buch der Informatik \cite{citeseer_algo_buch}. In diesem Buch werden verschiedene NP-vollständige und NP-schwere 
Probleme vorgestellt und in Kategorien unterteilt. Diese Kategorien (siehe Auflistung unten) werden auch benutzt, um die zu analysierenden Probleme auszuwählen.

\begin{itemize}
	\item Graphentheorie
	\item Netzwerk Design
	\item Sets und Partitionen
	\item Speicherung und Wiederherstellung
	\item Sequenzierung und Planung
	\item Mathematisches Programmieren
	\item Algebra und Zahlentheorie
	\item Spiele und Puzzles
	\item Logik
	\item Automaten und Sprachtheorie
	\item Programm Optimierung
	\item Sonstiges
	\item Offene Probleme
\end{itemize}

Im Rahmen dieser Arbeit konnten nicht alle Probleme behandelt werden. Deshalb wurde eine Auswahl von fünf Problemen getroffen, welche in der Realität auftreten und nicht nur von rein 
wissenschaftlichem Interesse sind. Bei der Auswahl wurde darauf geachtet, dass die Probleme aus unterschiedlichen Kategorien kommen. Es wurden jedoch auch zwei aus der gleichen 
Kategorie ausgewählt, um zu analysieren wie sich diese im Gegensatz zu den anderen verhalten. 

Die ausgewählten Probleme werden im folgenden Abschnitt genauer erläutert und für die weiteren Schritte der Erstellung der Schnittstelle berücksichtigt. Durch dieses Vorgehen konnte eine 
hohe Diversität von Problemen sichergestellt werden, was bei der Erstellung der Anforderungen und des Konzepts hilfreich war.

\subsection{Hierarchie der Reduktion}\label{hierarchy_reduction}
Wie bereits in der Einleitung beschrieben, muss für den Beweis der NP-Vollständigkeit ein bekanntes NP-vollständiges Problem auf das Bestehende reduziert werden können. In 
\autoref{fig:hierarchy_reduction} ist die Hierarchie der Reduktion für die ausgewählten Probleme aufgezeigt.

\begin{figure}[h]
\centering 
\includegraphics[scale=0.75]{images/visio/problem_hierarchy.png}
\caption[Hierarchie der Reduktion zum Beweis der NP-Vollständigkeit]{Hierarchie der Reduktion zum Beweis der NP-Vollständigkeit \selfmade{, Daten aus \cite{garey1979computers}}}
\label{fig:hierarchy_reduction}
\end{figure}

\subsection{Graphentheorie}\label{graph_theory}

	\subsubsection{Färbung (Graphtheorie)}\label{colarability_graph_theory}
	Die Färbung in der Graphtheorie ist ein NP-vollständiges Problem. Dies wurde durch die Reduktion des 3-SAT Problemes bewiesen.

	\paragraph{Beschreibung}
	Englischer Name: Graph k-colorability

	Bei diesem Problem geht es darum die Knoten eines Graphens so zu färben, dass keine zwei benachbarten Knoten die gleiche Farbe tragen. Ein Graph heisst k-färbbar, wenn die 
Färbung mit k Farben korrekt durchgeführt werden kann. Dieses Problem ist für $k = 2$ in \glslink{polynomialzeit}{polynomieller} Zeit lösbar, für $k \ge 2$ jedoch nicht mehr. Es gibt Spezialfälle, 
bei welchen auch ein Problem mit $k \ge 2$ in \glslink{polynomialzeit}{polynomieller} Zeit lösbar sind (siehe \cite{garey1979computers}).

	\paragraph{Beispiel}
	Gegeben sei ein Graph mit 10 Knoten mit einer vorgegebenen Konfiguration (siehe \autoref{fig:graph_faerbung}).\\
	Gesucht ist die Färbung der Knoten, damit keine zwei benachbarten Knoten die gleiche Farbe haben. Weiter die minimale Anzahl Farben ($k$), welche verwendet werden müssen, damit 
	die Bedingung erfüllt ist. Der Graph \ref{fig:graph_faerbung_incorrect} zeigt eine ungültige Lösung mit zwei Farben. Durch einen Algorithmus kann ermittelt werden, dass dieser Graph 
	mindestens drei Fraben benötigt, um die Bedingung zu erfüllen (siehe Graph \ref{fig:graph_faerbung_correct}). Somit ist $k=3$.
\begin{figure}[ht]
\centering
\subfigure[Inkorrekte Färbung]{
  \includegraphics[scale=0.5]{images/visio/graph_faerbung_incorrect.png}
  \label{fig:graph_faerbung_incorrect}
}
\subfigure[Korrekte Färbung]{
  \includegraphics[scale=0.5]{images/visio/graph_faerbung_correct.png}
  \label{fig:graph_faerbung_correct}
}
\caption[Kanten Färbung eines Graphen mit 10 Knoten]{Kanten Färbung eines Graphen mit 10 Knoten \selfmade{}}
\label{fig:graph_faerbung}
\end{figure}

\FloatBarrier 
	\paragraph{Eingabe- und Ausgabedaten}\mbox{}\\
	Eingabedaten: Knoten mit ihren Verbindungen zu anderen Knoten\\
	Ausgabedaten: Knoten mit ihrer Färbung und $k$ (Anzahl benötigter Farben)

	\paragraph{Einfluss der Parameter auf die Komplexität}\mbox{}\\
	Bei der Knotenfärbung steigert sich die Komplexität mit Anzahl Kanten. Die Knoten sind nur insofern relevant, dass sie neue Kanten aufspannen. Dass die Komplexität nur von der Anzahl Kanten 
	abhängt, zeigt auch die Formel zur Berechnung des Maximums von $k$: $k \le \frac{1}{2} + \sqrt{2m + \frac{1}{4}}$, wobei m für die Anzahl Kanten steht 
	(siehe \cite{seminar_rainer_graph}). Je grösser $k$, desto häufiger treten Kollisionen auf und umso mehr Varianten müssen ausprobiert werden.

	\paragraph{Bekannte Algorithmen}
	(siehe \cite{seminar_algo_graph}, \cite{krumke2012graphentheoretische} und \cite{seminar_rob_graphen})
	\begin{itemize}
		\item Spalten-Generierungs-Ansatz
		\item Sequentielles Färben (Sequential Coloring)
		\item Backtracking Algorithmus
		\item Greedy Algorithmus
		\item Johnson-Algorithmus
	\end{itemize}	

	\paragraph{Bekannte reale Probleme}	
	Es gibt diverse reale Probleme, welche mit der Knotenfärbung gelöst werden können, hier sind nur einige davon aufgelistet:
	\begin{itemize}
		\item Stundenplan: Anhand der eingegebenen Daten wird ein Konfliktmatrix erstellt und diese dann in ein Färbungsproblem umgewandelt. Die Anzahl benötigten Farben sind dann 
			die Anzahl der verschiedenen Perioden, welche es benötigt, um eine Stundenplan ohne Konflikte zu erstellen. \cite{ieee_exam_table_graph_coloring} \cite{time_table_graph_coloring} \cite{timetabling_abdullah}
		\item Frequenzverteilung (Mobilfunk): Im Mobilfunk hat jede Antenne einen Frequenzbereich. Im Graph werden die Antennen miteinander verbunden, bei denen sich die 
			Reichweite überschneidet. Die Farben können durch Frequenzen ersetzt werden. Die Lösung stellt eine konfliktfreie Mobilnetzabdeckung dar. \cite{seminar_rob_graphen}
		\item Färben von Landkarten: Die Länder sind die Knoten, die Kanten die Verbindung zu den Nachbarländern und die errechnete Farbe entspricht der Einfärbung auf der Landkarte. 
			\cite{seminar_rob_graphen}
	\end{itemize}

\newpage
\subsection{Netzwerk Design}\label{network_design}

	\subsubsection{Problem des Handlungsreisenden}\label{tsp}
	Das Problem des Handlungsreisenden ist ein NP-vollständiges Problem. Dies wurde durch die Reduktion des Hamiltonkreisproblemes bewiesen.

	\paragraph{Beschreibung}
	Englischer Name: Traveling Salesman Problem\\
	Beim Problem des Handlungsreisenden geht es darum, mit einer optimalen Route von einem Ausgangspunkt verschiedene Wegpunkte abzufahren und wieder zum Ausgangspunkt zurück 
	zu kehren. Bei zehn Wegpunkten und ungleichen Hin- und Zurückwegen sind das über dreieinhalb Millionen Möglichkeiten. Bei der symmetrischen Variante sind die Verbindungen 
	zwischen zwei Punkten gleich lang, was die Komplexität halbiert. 

	\paragraph{Beispiel} Gegeben seien vier Wegpunkte (A, B, C, D), der Startpunkt sei A.\\
	Gesucht ist die optimale Route, welche alle Wegpunkte beinhaltet und wieder bei A endet.\\
	Die Abbildung \ref{fig:tsp_example} zeigt alle möglichen Lösungen mit ihren errechneten Werten. Die Route A-D-C-B-A ist mit einem Wert von 16 die beste Route. Weiter ist zu sehen, 
	dass schon bei 4 Wegpunkten 6 Möglichkeiten vorhanden sind.
\begin{figure}[h]
\centering
\includegraphics[scale=0.55]{images/visio/tsp.png}
\caption[Problem des Handlungsreisenden mit 4 Wegpunkten]{Problem des Handlungsreisenden mit 4 Wegpunkten \selfmade{}}
\label{fig:tsp_example}
\end{figure}

	\paragraph{Eingabe- und Ausgabedaten}\mbox{}\\
	Eingabedaten: Startpunkt und Wegpunkte, welche passiert werden müssen\\
	Ausgabedaten: Reihenfolge der Wegpunkte für eine optimale Route

	\paragraph{Einfluss der Parameter auf die Komplexität}\mbox{}\\
	Die Komplexität beim Problem des Handlungsreisenden wird durch die Anzahl Wegpunkte bestimmt. Für ein asymmetrisches Problem ist die Formel zur Berechnung der Möglichkeit $(n-1)!$ 
	für ein symmetrisches Problem hingegen $\frac{(n-1)!}{2}$.
	
	 \newpage
	\paragraph{Bekannte Algorithmen}\cite{tsp_algorithmen} \cite{tsp_semesterarbeit}
	\begin{itemize}
		\item Nearest-Neighbor
		\item 2Opt
		\item Christofides
	\end{itemize}

	\subsubsection{Briefträgerproblem}\label{chinese_postman}
	Das Briefträgerproblem ist ein NP-vollständiges Problem. Dies wurde durch die Reduktion des 3-SAT Problemes bewiesen.

	\paragraph{Beschreibung}
	Englischer Name: Chinese postman problem\\
	Das Briefträgerproblem ist vergleichbar mit dem Problem des Handlungsreisenden, jedoch geht es darum jede Kante mindestens ein Mal abzufahren. Die Knoten stellen Kreuzungen dar, die 
	Kanten entsprechen den Strassen. Die minimale Länge kann mit Hilfe des \glslink{eulerkreis}{Eulerkreises} relativ einfach berechnet werden. Falls der Graph die Kriterien des 
	\glslink{eulerkreis}{Eulerkreises} nicht erfüllt, werden alle Knoten mit ungerader Anzahl Kanten mit einem anderen solchen Knoten über die kürzeste Route verbunden. 
	Die minimal Läng ist die Summe aller Strecken und Hilfsstrecken.
	\cite{pearson2004decision}

	\paragraph{Beispiel} Gegeben sei der Graph mit den Punkten A bis H und den Verbindungen in blau.\\
Gesucht ist eine Route mit dem kürzesten Weg, welche jede Kante mindestens ein Mal abfährt. \cite{pearson2004decision}\\
Die grün eingezeichneten Verbindungen sind Hilfslinien, um den Graphen in einen \gls{eulerkreis} zu verwandeln. Eine mögliche Lösung mit der minimal Länge von 1000 ist die 
Route A-D-C-G-H-C-A-B-D-F-B-E-F-H-F-B-A.
\begin{figure}[h]
\centering
\includegraphics[scale=0.8]{images/visio/chinese_postman.png}
\caption[Beispiel für ein Briefträgerproblem]{Beispiel für ein Briefträgerproblem \selfmade{, Daten entnommen aus: \cite{pearson2004decision}}}
\label{fig:chinese_postman_example}
\end{figure}

	\paragraph{Eingabe- und Ausgabedaten}\mbox{}\\
	Eingabedaten: Startpunkt und Knoten mit ihren Verbindungen mit einer Gewichtung\\
	Ausgabedaten: Reihenfolge der Knoten für eine minimale Strecke

	\paragraph{Einfluss der Parameter auf die Komplexität}\mbox{}\\
	Wie bei der Knotenfärbung spielt die Anzahl Kanten die Hauptrolle bei der Komplexität. Wie viele Knoten ein Graph hat und ob dieser bereits \glslink{eulerkreis}{eulersch} ist, hat einen 
	geringeren Einfluss.

	\paragraph{Bekannte Algorithmen}\mbox{}\\
	Briefträgeralgorithmus (Chinese postman algorithm)

\subsection{Sequenzierung und Planung}\label{sequencing_scheduling}

	\subsubsection{Stundenplan-Erstellung}\label{tsp}
	Das Erstellen eines Stundenplans ist ein NP-vollständiges Problem. Dies wurde durch die Reduktion des 3-SAT-Problems bewiesen.

	\paragraph{Beschreibung}
	Englischer Name: Timetable design\\
	Die Erstellung von Stundenplänen ist ein sehr komplexes Problem. Basierend auf Fächer, Lehrer, Klassen und Klassenzimmern wird versucht, eine optimale Verteilung der Stunden zu 
	erreichen. Die wichtigste Bedingung ist, dass eine Ressource, sei das ein Lehrer, eine Klasse oder ein Schulzimmer, zu jedem Zeitpunkt höchstens ein Mal verplant ist. 
	Zusätzlich kann es beliebige weitere Kriterien geben, beispielsweise dass eine Klasse nie mehr als neun Lektionen pro Tag haben soll oder ein Lehrer nie mehr als 
	fünf Lektionen nacheinander unterrichten sollte. Bei der Erstellung von Stundenplänen wird oft von Hard Constraints und Soft Constraints gesprochen. Ein Hard Constraint ist im Gegensatz zu 
	einem Soft Constraint unabdingbar. Es ist nicht möglich, dass ein Lehrer zwei verschiedene Fächer gleichzeitig unterrichtet, im Notfall könnte er aber mehr als fünf Lektionen hintereinander 
	unterrichten. \cite{Abramson92aparallel} \cite{Abramson91constructingschool} \cite{framework_timetabling} \cite{time_table_constraint_opti_ea}

	\paragraph{Beispiel} Gegeben seien die Fächer, Lehrer, Klassen und Klassenzimmer aus den Tabellen \ref{table:eg_subject}, \ref{table:eg_teacher}, \ref{table:eg_schoolclasses} und \ref{table:eg_schoolroom}.\\
Gesucht ist ein Stundenplan, bei welchem es keine Kollisionen für Lehrer, Klassen und Klassenzimmer gibt.

\begin{table}[ht]
\centering
  \begin{tabular}{ l | l }
	\hline
	\rowcolor{gray}
	\textbf{Fachname}	& \textbf{Kürzel}\\ \hline
	Mathematik		& M\\ \hline
	Deutsch		& D\\ \hline
	Englisch		& E\\ \hline
	Französisch		& F\\ \hline
	Sport			& Sp
  \end{tabular}
   \caption{Schulfächer}\label{table:eg_subject}
\end{table}

\begin{table}[ht]
\centering
  \begin{tabular}{ l | l }
	\hline
	\rowcolor{gray}
	\textbf{Lehrername} 	& \textbf{Ausbildung}\\ \hline
	Angst				& Deutsch, Mathematik, Sport\\ \hline
	Arm				& Sport\\ \hline
	Bernasconi			& Deutsch, Mathematik, Französisch\\ \hline
	Müller				& Deutsch, Mathematik, Englisch, Französisch\\ \hline
	Pfister				& Englisch, Französisch
  \end{tabular}
   \caption{Lehrer}\label{table:eg_teacher}
\end{table}

\begin{table}[ht]
\centering
  \begin{tabular}{ l | l }
	\hline
	\rowcolor{gray}
	\textbf{Klassenname} 	& \textbf{Benötigte Fächer}\\ \hline
	Tja13				& Deutsch, Mathematik, Sport\\ \hline
	Tja12				& Deutsch, Mathematik, Sport, Französisch\\ \hline
	Tja11				& Deutsch, Mathematik, Sport, Französisch, Englisch\\ \hline
	Tja10				& Deutsch, Mathematik, Sport, Französisch, Englisch
  \end{tabular}
   \caption{Klassen}\label{table:eg_schoolclasses}
\end{table}

\begin{table}[ht]
\centering
  \begin{tabular}{ l }
	\hline
	\rowcolor{gray}
	\textbf{Zimmername}\\ \hline
	Zimmer 101\\ \hline
	Zimmer 103\\ \hline
	Zimmer 201\\ \hline
	Turnhalle
  \end{tabular}
   \caption{Klassenzimmer}\label{table:eg_schoolroom}
\end{table}

\FloatBarrier
\autoref{table:timetable_1} zeigt eine mögliche Lösung. Es wurde beachtet, dass die Lehrer möglichst gleich viele Stunden unterrichten und die Schüler keine doppelten Freistunden haben. Es 
fällt auf, dass das Zimmer 201 nur selten besetzt ist. In der zweiten Variante (siehe \autoref{table:timetable_2}), wurden die Stunden der Klasse Tja11 so verschoben, dass 
das Zimmer 201 gar nicht mehr benötigt wird. Die Klasse Tja11 hat nun aber eine doppelte Freistunde. Die Eingabedaten sind im Beispiel noch sehr überschaubar, 
trotzdem gibt es bereits enorm viele verschiedenen Möglichkeiten und Ausprägungen.

\begin{table}[ht]
\centering
  \begin{tabular}{ l | l | l | l | l }
	\hline
	\rowcolor{gray}
	\textbf{Uhrzeit} 	& \textbf{Turnhalle}	& \textbf{Zimmer 101} 	& \textbf{Zimmer 103}	&  \textbf{Zimmer 201}\\ \hline
	0800-0900		& Sp / Tja13 / Arm		& D / Tja11 / Müller		& 				& \\ \hline
	0900-1000		& Sp / Tja13 / Arm		& M / Tja11 / Müller		& F / Tja12 / Pfister		& \\ \hline
	1000-1100		& Sp / Tja12 / Arm		& D / Tja10 / Bernasconi	& M / Tja13 / Angst		& F / Tja11 / Müller\\ \hline
	1100-1200		& Sp / Tja12 / Arm		& M / Tja10 / Angst		& D / Tja13 / Bernasconi	& E / Tja11 / Pfister\\ \hline \hline
	1300-1400		& Sp / Tja11 / Angst	& F / Tja10 / Bernasconi	& D / Tja12 / Müller		& \\ \hline
	1400-1500		& Sp / Tja11 / Angst	& E / Tja10 / Pfister		& M / Tja12 / Bernasconi	& \\ \hline
	1500-1600		& Sp / Tja10 / Angst	& 				& 				& \\ \hline
	1600-1700		& Sp / Tja10 / Angst	& 				& 				& \\ \hline
  \end{tabular}
   \caption{Möglicher Stundenplan - Variante 1}\label{table:timetable_1}
\end{table}

\begin{table}[ht]
\centering
  \begin{tabular}{ l | l | l | l }
	\hline
	\rowcolor{gray}
	\textbf{Uhrzeit} 	& \textbf{Turnhalle}	& \textbf{Zimmer 101} 	& \textbf{Zimmer 103}	\\ \hline
	0800-0900		& Sp / Tja13 / Arm		& D / Tja11 / Müller		& 				\\ \hline
	0900-1000		& Sp / Tja13 / Arm		& M / Tja11 / Müller		& F / Tja12 / Pfister		\\ \hline
	1000-1100		& Sp / Tja12 / Arm		& D / Tja10 / Bernasconi	& M / Tja13 / Angst		\\ \hline
	1100-1200		& Sp / Tja12 / Arm		& M / Tja10 / Angst		& D / Tja13 / Bernasconi	\\ \hline \hline
	1300-1400		& Sp / Tja11 / Angst	& F / Tja10 / Bernasconi	& D / Tja12 / Müller		\\ \hline
	1400-1500		& Sp / Tja11 / Angst	& E / Tja10 / Pfister		& M / Tja12 / Bernasconi	\\ \hline
	1500-1600		& Sp / Tja10 / Angst	& F / Tja11 / Müller		& 				\\ \hline
	1600-1700		& Sp / Tja10 / Angst	& E / Tja11 / Pfister		& 				\\ \hline
  \end{tabular}
   \caption{Möglicher Stundenplan - Variante 2}\label{table:timetable_2}
\end{table}

\FloatBarrier
	Eine sehr gut ausgearbeitete Software ist 'Units' \cite{unit_express}, sie liefert umfangreiche Funktionen zur Erstellung von Stundenplänen.

	\paragraph{Eingabe- und Ausgabedaten}\mbox{}\\
	Eingabedaten: Fächer, Lehrer, Klassen und Klassenzimmer mit verschiedenen Einschränkungen und Zusatzinformationen\\
	Ausgabedaten: Einteilung der Fächer mit den dazugehörigen Klassen, Lehrern und Zimmer auf verschiedene Tage und Uhrzeiten, welche keine Konflikte enthält

	\paragraph{Einfluss der Parameter auf die Komplexität}\mbox{}\\
	Beim Stundenplanproblem sind es nicht alle vier Listen, welche die Komplexität des Problems ausmachen. Den grössten Einfluss haben die Anzahl Stunden, welche verplant werden 
	müssen. Die Anzahl Möglichkeiten berechnen sich wie folgt (vergleiche \cite{scheduling_komplex}): 
	\[(Anzahl Räume * Anzahl Zeitfenster * Anzahl Lehrer)^{Anzahl Veranstaltungen}\]

	\paragraph{Bekannte Algorithmen}
	(siehe \cite{framework_timetabling})
	\begin{itemize}
		\item Backtracking Algorithmus
		\item Evolutionäre Algorithmen
		\item Constraint Logic Programming
	\end{itemize}	

	\paragraph{Bekannte ähnliche Probleme}	
	Es gibt diverse andere Planungsprobleme, welche auf dem gleichen Konzept basieren, jedoch unterschiedliche Eingabedaten und Beschränkungen haben:
	\begin{itemize}
		\item Spielplan
		\item Prüfungsplan
		\item Schichtenplanung
	\end{itemize}

\subsection{Mathematisches Programmieren}\label{mathematical_programming}

	\subsubsection{Rucksack-Problem}\label{knapsack}
	Das Rucksack-Problem ist ein NP-vollständiges Problem. Dies wurde durch die Reduktion des Problems der exakten Überdeckung bewiesen.

	\paragraph{Beschreibung}
	Englischer Name: Knapsack\\
	Beim Rucksack-Problem geht es um einen Container, symbolisch der Rucksack, mit einer Gewichtsschranke. Zudem gibt es Objekte, welche in den Container gepackt werden 
	können. Diese Objekte besitzen ein Gewicht und einen Profit. Das Ziel ist es, den grösstmöglichen Profit zu erlangen, ohne die Gewichtsschranke zu überschreiten. 
	\cite{knapsack_desc_web}

	\paragraph{Beispiel} Gegeben sei ein Rakete mit der Gewichtsschranke 645 kg und die Objekte in Tabelle \ref{table:knapsack_objects}.\\
Gesucht ist die Objektauswahl mit dem grössten Profit, welche die Gewichtsschranke nicht überschreitet.\\
In diesem Beispiel wäre die optimale Lösung 1, 2, 3 und 5. \cite{knapsack_desc_web}

\begin{table}[ht]
\centering
  \begin{tabular}{ c | r | r }
	\hline
	\rowcolor{gray}
	\textbf{Objekt-Nr.}	&	\textbf{Gewicht in kg}	&	\textbf{Profit}\\ \hline
	1			&	153			&	232\\ \hline
	2			&	54			&	73\\ \hline
	3			&	191			&	201\\ \hline
	4			&	66			&	50\\ \hline
	5			&	239			&	141\\ \hline
	6			&	137			&	79\\ \hline
	7			&	148			&	48\\ \hline
	8			&	249			&	38
  \end{tabular}
   \caption[Knapsack Objekte mit Gewicht und Profit]{Knapsack Objekte mit Gewicht und Profit (Daten aus \cite{knapsack_desc_web})}\label{table:knapsack_objects}
\end{table}

	\paragraph{Eingabe- und Ausgabedaten}\mbox{}\\
	Eingabedaten: Gewichtsschranke und Elemente mit Gewicht und Nutzwert\\
	Ausgabedaten: Zusammenstellung der Elemente mit dem höchsten Nutzwert

	\paragraph{Einfluss der Parameter auf die Komplexität}\mbox{}\\
	Die Komplexität des Rucksack-Problems hängt von der Anzahl der Objekte ab, die Gewichtsschranke trägt nicht zur Komplexität bei.

	\paragraph{Bekannte Algorithmen}
	\begin{itemize}
		\item Backtracking Algorithmus
		\item Greedy Algorithmus
		\item Algorithmus von Nemhauser und Ullmann \cite{knapsack_desc_web}
	\end{itemize}	

	\paragraph{Bekannte reale Probleme}	
	Es gibt diverse reale Probleme, welche mit dem Rucksack-Problem gelöst werden können \cite{kellerer2004knapsack}, hier sind nur einige davon aufgelistet:
	\begin{itemize}
		\item Ausschneiden von verschiedenen Stücken aus einer Metall- oder Holzplatte: Optimale Nutzung der Fläche(n), damit am Schluss genug Einzelstücke für das Herstellen 
			des Endproduktes vorhanden sind.
		\item Kreditvergabe: Optimale Ausnutzung des Kreditbudgets mit der Vergabe von Krediten an Kunden, welche einen Kredit über eine gewisse Höhe haben möchten und eine 
			bestimmte Risikoeinstufung aufweisen.
	\end{itemize}


\newpage
\subsection{Logik}\label{logic}
Die in diesem Abschnitt aufgeführten Probleme sind der Vollständigkeit halber aufgelistet und beschrieben. Sie werden nicht für die weiteren Schritte der Schnittstelle verwendet. Sie sind die 
Grundpfeiler der NP-Vollständigkeit, auf welche zahlreiche Beweise der NP-Vollständigkeit von Probleme basieren.

	\subsubsection{Erfüllbarkeitsproblems der Aussagenlogik (SAT)}\label{sat}
	Das Erfüllbarkeitsproblem der \gls{aussagenlogik} ist ein NP-vollständiges Problem. Dies wurde durch Stephen A. Cook in den 1970er Jahren bewiesen \cite{cook_complexity}.

	\paragraph{Beschreibung}
	Englischer Name: Satisfiability (SAT)\\
	Beim Erfüllbarkeitsproblem der \gls{aussagenlogik} ist zu überprüfen, ob eine beliebige \glslink{aussagenlogik}{aussagenlogische} Formel erfüllbar ist.	

	\paragraph{Beispiel} Gegeben sei eine \glslink{aussagenlogik}{aussagenlogische} Formel \ref{eq:aussagenlogik}.\\
	Gesucht ist die Antwort auf die Erfüllbarkeit dieser Formel.\\
	Die \autoref{table:sat_results} zeigt, dass die Formel für die Kombinationen 3, 4 und 8 erfüllbar ist.

	\begin{equation}
   		(A \vee B) \wedge C
  		 \label{eq:aussagenlogik}
	\end{equation}

\begin{table}[ht]
\centering
  \begin{tabular}{ c | c | c | c | c | c}
	\hline
	\rowcolor{gray}
	\textbf{Kombination-Nr.}	&	\textbf{A}	&	\textbf{B} 	& 	\textbf{C} 	&	$A \vee B$	&	\textbf{Resultat}\\ \hline
	1			&	wahr	& 	falsch	& 	falsch	&	wahr		&	falsch\\ \hline
	2			&	wahr	& 	wahr	& 	falsch	&	wahr		&	falsch\\ \hline
	\textbf{3}		&	wahr	& 	falsch	& 	wahr	&	wahr		&	\textbf{wahr}\\ \hline
	\textbf{4}		&	wahr	& 	wahr	& 	wahr	&	wahr		&	\textbf{wahr}\\ \hline
	5			&	falsch	& 	falsch	& 	falsch	&	falsch		&	falsch\\ \hline
	6			&	falsch	& 	wahr	& 	falsch	&	wahr		&	falsch\\ \hline
	7			&	falsch	& 	falsch	& 	wahr	&	falsch		&	falsch\\ \hline
	\textbf{8}		&	falsch	& 	wahr	& 	wahr	&	wahr		&	\textbf{wahr}\\ \hline
  \end{tabular}
   \caption{Wahrheitstabelle zur \glslink{aussagenlogik}{aussagenlogischen} Formel}\label{table:sat_results}
\end{table}

\newpage
	\subsubsection{3-SAT}\label{3sat}
	Das 3-SAT Problem ist ein NP-vollständiges Problem. Dies wurde durch die Reduktion des SAT Problems bewiesen.

	\paragraph{Beschreibung}
	Englischer Name: 3-Satisfiability (3-SAT)\\
	Das 3-SAT Problem ist eine Spezialfall des SAT Problems, es dürfen maximal 3 Literale in einer Klausel enthalten sein.

	\paragraph{Beispiel} Gegeben sei eine 3-SAT Formel \ref{eq:aussagenlogik_3sat}.\\
	Gesucht ist die Antwort auf die Erfüllbarkeit dieser Formel.\\
	Die \autoref{table:3sat_results} beweist, dass die Formel erfüllbar ist, lediglich für die Kombinationen 5, 11, 13 und 15 ist sie nicht korrekt.

	\begin{equation}
   		(A \wedge B \wedge C) \vee (B \wedge \neg C \wedge D)
  		 \label{eq:aussagenlogik_3sat}
	\end{equation}

\begin{table}[ht]
\centering
  \begin{tabular}{ c | c | c | c | c | c | c | c}
	\hline
	\rowcolor{gray}
	\textbf{Kombination-Nr.}	& \textbf{A}	& \textbf{B} 	& \textbf{C} & \textbf{D}  & $A \wedge B \wedge C$ & $B \wedge \neg C \wedge D$	& \textbf{Resultat}\\ \hline
	\textbf{1}			& wahr	& falsch	& falsch	& wahr	& wahr			 & wahr				& \textbf{wahr}\\ \hline
	\textbf{2}			& wahr	& wahr	& falsch	& wahr	& wahr			 & wahr 				& \textbf{wahr}\\ \hline
	\textbf{3}			& wahr	& falsch	& wahr	& wahr	& wahr			 & wahr				& \textbf{wahr}\\ \hline
	\textbf{4}			& wahr	& wahr	& wahr	& wahr	& wahr			 & wahr				& \textbf{wahr}\\ \hline
	5				& falsch	& falsch	& falsch	& wahr	& falsch			 & wahr				& falsch\\ \hline
	\textbf{6}			& falsch	& wahr	& falsch	& wahr	& wahr			 & wahr				& \textbf{wahr}\\ \hline
	\textbf{7}			& falsch	& falsch	& wahr	& wahr	& wahr			 & wahr 				& \textbf{wahr}\\ \hline
	\textbf{8}			& falsch	& wahr	& wahr	& wahr	& wahr			 & wahr 				& \textbf{wahr}\\ \hline
	\textbf{9}			& wahr	& falsch	& falsch	& falsch	& wahr			 & wahr				& \textbf{wahr}\\ \hline
	\textbf{10}			& wahr	& wahr	& falsch	& falsch	& wahr			 & wahr 				& \textbf{wahr}\\ \hline
	11				& wahr	& falsch	& wahr	& falsch	& wahr			 & falsch				& falsch\\ \hline
	\textbf{12}			& wahr	& wahr	& wahr	& falsch	& wahr			 & wahr				& \textbf{wahr}\\ \hline
	13				& falsch	& falsch	& falsch	& falsch	& falsch			 & wahr				& falsch\\ \hline
	\textbf{14}			& falsch	& wahr	& falsch	& falsch	& wahr			 & wahr				& \textbf{wahr}\\ \hline
	15				& falsch	& falsch	& wahr	& falsch	& wahr			 & falsch 				& falsch\\ \hline
	\textbf{16}			& falsch	& wahr	& wahr	& falsch	& wahr			 & wahr 				& \textbf{wahr}\\ \hline
  \end{tabular}
   \caption{Wahrheitstabelle zur 3-SAT Formel}\label{table:3sat_results}
\end{table}

\newpage
\subsection{Übersicht Eingabe- und Ausgabedaten \resultAssignment{[R2]}}\label{overview_input_output}
Die Eingabe- und Ausgabedaten der Probleme sind sehr unterschiedlich und weisen unterschiedliche Komplexität auf. Um einen besseren Überblick zu erhalten, wurden sie in der 
\autoref{table:input_output} zusammengefasst.
\begin{table}[ht]
\centering
  \begin{tabular}{ p{3cm} | p{5.4cm} | p{5.4cm} }
	\hline
	\rowcolor{gray}
	\textbf{Problem}				& \textbf{Eingabedaten}								& \textbf{Ausgabedaten}\\ \hline
	Färbung (Graphtheorie)			& \begin{enumerate}
								\item Knoten mit ihren Verbindungen zu anderen Knoten
							   \end{enumerate}				
							&  \begin{enumerate}
								\item Knoten mit ihrer Färbung
								\item $k$ (Anzahl benötigter Farben)
							   \end{enumerate}	\\ \hline
	Problem des Handlungsreisenden		& \begin{enumerate}
								\item Startpunkt
								\item Wegpunkte, welche passiert werden müssen
							   \end{enumerate}				
							&  \begin{enumerate}
								\item Wegpunkte in der Reihenfolge der optimalen Route
								\item Länge der Strecke
							   \end{enumerate}	\\ \hline
	Briefträgerproblem	 			& \begin{enumerate}
								\item Startpunkt
								\item Knoten mit ihren Verbindungen mit Gewichtung
							   \end{enumerate}				
							&  \begin{enumerate}
								\item Knoten in der Reihenfolge der minimalen Strecke
								\item Länge der Strecke
							   \end{enumerate}	\\ \hline
	Stundenplan-Erstellung			& \begin{enumerate}
								\item Fächer
								\item Lehrer
								\item Klassen
								\item Klassenzimmer
								\item Stundenplan Rahmenbedingungen
							   \end{enumerate}				
							&  \begin{enumerate}
								\item  Einteilung der Fächer mit den dazugehörigen Klassen, Lehrern und Zimmer auf verschiedene Tage und Uhrzeiten, welche keine 
									Konflikte enthält
							   \end{enumerate}	\\ \hline
	Rucksack-Problem				& \begin{enumerate}
								\item Gewichtsschranke
								\item Elemente mit Gewicht und Nutzwert
							   \end{enumerate}				
							&  \begin{enumerate}
								\item Zusammenstellung von den Elementen mit höchstem Nutzwert
								\item Nutzwert
							   \end{enumerate}	\\ \hline
  \end{tabular}
   \caption{Eingabe- und Ausgabedaten der ausgewählten Probleme}\label{table:input_output}
\end{table}

\newpage
\subsection{Beispiel für den Einfluss der Parameter auf die Komplexität \resultAssignment{[R1a]}}\label{example_complexity_knapsack}
Bei den vorgestellten Problemen wurde der Einfluss der Parameter auf die Komplexität der Probleme erwähnt. Um dies zu verdeutlichen wurde ein praktisches Beispiel anhand des 
Rucksack-Problems erstellt.

\subsubsection{Ausgangslage}
Das Rucksack-Problem hat zwei Eingabeparameter, zum einen die Gewichtsschranke und zum anderen die Elemente. Im Beispiel wurden verschiedene Anzahl Elemente verwendet und dies mit 
unterschiedlichen Gewichtsschranken kombiniert. Zum Vergleich wurde die Zeit gestoppt, welche der Brute Force-Algorithmus benötigt, um die optimale Lösung herauszufinden. Bei dieser 
Methode wird das Ergebnis zwar durch andere Prozesse auf dem Testsystem verfälscht, reicht jedoch für eine Veranschaulichung der Einflüsse.

\subsubsection{Resultat}
Die Berechnungszeiten waren ungenau und unterschiedlich. Das Resultat bestätigt jedoch bei jedem Versuch die Theorie, dass die Komplexität nur durch die Anzahl an Elementen 
beeinflusst wird. Die verschiedenen Berechnungszeiten wurden in der \autoref{table:example_complexity_knapsack} festgehalten. Mit einer höheren Gewichtsschranke bleibt die 
Berechnungszeit in etwa gleich, bei einer Erhöhung der Anzahl Elemente nimmt die Komplexität jedoch sehr schnell zu.

\begin{table}[ht]
\centering
  \begin{tabular}{ l | r | r | r | r | r }
	\hline
	\rowcolor{gray}
	\backslashbox{Gewichtsschranke:}{Anzahl an Elemente:}	& \textbf{10}	& \textbf{15} 	& \textbf{18}	& \textbf{20}	& \textbf{22}\\ \hline
	\textbf{10}							& 4ms			& 23ms		& 129ms		& 1079ms		& 7267ms 	\\ \hline
	\textbf{100}							& 1ms			& 16ms		& 149ms		& 1761ms		& 6054ms	\\ \hline
	\textbf{1000}						& 1ms			& 7ms			& 595ms		& 1038ms		& 6331ms	\\ \hline
	\textbf{10000}						& 1ms			& 7ms			& 48ms		& 583ms		& 7174ms	\\ \hline
  \end{tabular}
   \caption[Berechnungszeiten bei verschiedenen Eingabeparametern für das Rucksack-Problem]{Berechnungszeiten bei verschiedenen Eingabeparametern für das Rucksack-Problem}
   \label{table:example_complexity_knapsack}
\end{table}

In \autoref{fig:example_complexity_knapsack} wird die exponentielle Steigerung der Berechnungsdauer mit zunehmender Anzahl an Elementen sehr schön verdeutlicht.

\begin{figure}[h]
\centering
\includegraphics[scale=0.45]{images/excel/knapsack_complexity_example.png}
\caption[Berechnungszeiten für das Rucksack-Problem mit verschiedenen Eingabeparametern im Vergleich]{Berechnungszeiten für das Rucksack-Problem mit verschiedenen Eingabeparametern 
im Vergleich \selfmade{}}
\label{fig:example_complexity_knapsack}
\end{figure}

%%%%%%%%%%%%%%%%%%%%%%%%%%%%%%%%%%%%%%%%%%%%%%%%%%%%%%%%%%%%%%%%%
%
% Project     : Bachelorarbeit
% Title       : Machbarkeitsanalyse für eine ressourcenorientierte Schnittstelle zur Verarbeitung grundlegender Probleme der Informatik
% File        : anforderungsdokument.tex Rev. 01
% Date        : 01.03.2015
% Author      : Raffael Santschi
%
%%%%%%%%%%%%%%%%%%%%%%%%%%%%%%%%%%%%%%%%%%%%%%%%%%%%%%%%%%%%%%%%%

\chapter{Anforderungsdokument \resultAssignment{[R3]}}\label{chap.anforderungsdokument}

Das Anforderungsdokument legt die Basis für die Implementation. Es ist für den Verlauf des Projekts wichtig, dass zu Beginn die Anforderungen aufgestellt werden. Bei der 
Erstellung des Anforderungsdokuments werden verschiedene Betrachtungsweisen aufgezeigt und die Anforderungen an das System in verschiedenen Detailstufen angeschaut.


\section{Übersicht}\label{anf_uebersicht}

In diesem Abschnitt  wird die System- und Kontextabgrenzung dargelegt, die Systemumgebung beschrieben, die getroffenen Annahmen festgehalten und die verschiedenen \gls{stakeholder} mit 
ihren Erwartungen aufgelistet.

\subsection{System- und Kontextabgrenzung}\label{systemabgrenzung}
Der Systemkontext umfasst alle Aspekte, die für die Anforderungen des geplanten Systems relevant sind und nicht im Rahmen der Entwicklung dieses System gestaltet werden können.
\cite{req_eng_book} 

\begin{figure}[h]
\centering
\includegraphics[scale=0.8]{images/visio/systemkontext.png}
\caption[Systemkontext]{Systemkontext \selfmade{}}
\label{fig:systemkontext}
\end{figure}

Der Systemkontext (siehe \autoref{fig:systemkontext}) zeigt, dass das System relevante Schnittstellen zu den Nutzern und zum Verarbeitungssystem hat. Das System muss Daten 
für das Verarbeitungssystem zur Verfügung stellen und auch solche annehmen. Zudem wird das System von den Nutzern, Optimierungsproblemen und den dazugehörigen Algorithmen 
beeinflusst.

\FloatBarrier
\subsection{Systemumgebung}\label{systemumgebung}
\autoref{fig:systemumgebung} zeigt die Systemumgebung, welche die Ausgangslage für das Projekt definiert. Am Anfang des Projekts war bekannt, dass Nutzer und ein Verarbeitungssystem 
Dienste des zu erstellenden Systems beziehen werden. Die genaue Ausprägung dieser Dienste werden in diesem Kapitel behandelt.

\begin{figure}[h]
\centering
\includegraphics[scale=0.8]{images/visio/systemumgebung.png}
\caption[Systemumgebung]{Systemumgebung \selfmade{}}
\label{fig:systemumgebung}
\end{figure}

\FloatBarrier

\subsection{Stakeholder}\label{stakeholder}
Zum Erfassen aller Nutzergruppen, die Einfluss auf das Projekt haben können, dient die \gls{stakeholder}-Analyse. Zudem ermöglicht sie die Erfassung aller Gruppen, die potenziell 
Anforderungen an das Projekt stellen. In \autoref{table:stakeholder} wurden die \gls{stakeholder} dieses Projekts zusammengetragen und ihre Erwartungen, ihre 
Einstellung und ihr Einfluss gegenüber dem Projekt festgehalten. Da es sich um eine Machbarkeitsanalyse handelt, befinden sich nur der Auftraggeber, ein einzelner potenzieller Kunde und der 
Entwickler in der Auflistung.

\begin{table}[ht]
\centering
  \begin{tabular}{ p{5cm} | p{5cm} | p{1.5cm} | p{1.5cm} }
	\hline
	\rowcolor{darkgray}
	\textbf{Name}					&	\textbf{Erwartung}	&	\textbf{Einstellung} 	&	\textbf{Einfluss}	\\ \hline
	\rowcolor{gray}
								&				&	-Positiv \mbox{-Neutral} \mbox{-Negativ} 	&	-Hoch \mbox{-Mittel} \mbox{-Niedrig} \\ \hline
	\textbf{Phil Hofmann} (Vorsteher der Geschäftsführung der 200ok GmbH)						
								&	Der Auftraggeber in diesem Projekt erwartet von der Machbarkeitsanalyse Informationen für ein mögliches Projekt zur 
									Umsetzung der Gesamtidee.
												& 	Positiv		&	Hoch		\\ \hline
	\textbf{Potenzieller Kunde}
								&	Er wünscht sich eine einfache Abwicklung für seine Probleme, er möchte sich nicht mit Algorithmen und 
									theoretischer Informatik beschäftigen.
												& 	Positiv		&	Mittel		\\ \hline
	\textbf{Entwickler}
								&	Er hofft, dass sich die Schnittstelle wie gewünscht umsetzen lässt und aus der Machbarkeitsanalyse ein weiteres Projekt entsteht.
												& 	Positiv		&	Mittel		\\ \hline												
  \end{tabular}
   \caption{Liste der \gls{stakeholder}}\label{table:stakeholder}
\end{table}

\newpage
\section{Anforderungen}\label{sec.anfoderungen}
Zum Erfassen der Anforderungen an das System wurden zuerst verschiedene Use Cases definiert, mit deren Hilfe anschliessend der Anforderungskatalog erstellt werden konnte.

\subsection{Use Cases}\label{use_cases}
Das Use Case Diagramm (siehe Abbildung \ref{fig:use_case}) zeigt einen Akteur, ein System und sechs Use Cases, welche für diese Arbeit relevant sind.
\begin{figure}[h]
\includegraphics{images/anforderungen/use_cases.png}
\caption[Use-Case Diagramm]{Use-Case Diagramm \selfmade{}}
\label{fig:use_case}
\end{figure}

Alle Use Cases wurden anhand der Vorlage in \autoref{table:use_case_template} spezifiziert. Diese Vorlage basiert auf Angaben von \cite{req_eng_book}.

\begin{table}[ht]
\centering
  \begin{tabular}{ l | p{10cm} }
	\hline
	\rowcolor{darkgray}
	\textbf{Attribute}&	\textbf{Beschreibung}\\ \hline
	\rowcolor{gray}
	\textbf{Bezeichner}&	\textbf{Eindeutiger Bezeichner}\\ \hline
	\textbf{Name}		&	Eindeutiger Name des Use Case\\ \hline
	\textbf{Beschreibung}	&	Komprimierte Beschreibung\\ \hline
	\textbf{Auslösendes Ereignis} &	Ereignis, das den Use Case auslöst.\\ \hline
	\textbf{Akteure}		&	Auflistung der Akteure, die mit dem Use Case in Beziehung stehen.\\ \hline
	\textbf{Vorbedingung}	&	Liste notwendiger Voraussetzungen, die erfüllt sein müssen, bevor die Ausführung des Use Case beginnen kann.\\ \hline
	\textbf{Nachbedingung}	&	Liste von Zuständen, in denen sich das System unmittelbar nach der Ausführung des Hauptszenarios befindet.\\ \hline
	\textbf{Ergebnis}		&	Beschreibung der Ausgaben, die während der Ausführung des Use Case erzeugt werden.\\ \hline
	\textbf{Hauptszenario}	&	Beschreibung des Hauptszenarios des Use Case\\ \hline
	\textbf{Alternativszenarien}	&	Beschreibung von Alternativszenarien des Use Case oder Angabe der auslösenden Ereignisse. 
					Hier gelten oftmals andere Nachbedingungen.\\ \hline
  \end{tabular}
   \caption{Vorlage für Use Case Spezifikation}\label{table:use_case_template}
\end{table}

\begin{table}[ht]
\centering
  \begin{tabular}{ l | p{10cm} }
	\hline
	\rowcolor{gray}
	\textbf{Bezeichner}	&	\textbf{UC-1}\\ \hline
	\textbf{Name}		&	Lösung beauftragen\\ \hline
	\textbf{Beschreibung}	&	Ein Nutzer möchte eine Lösung eines Problem mit spezifischen Parametern beauftragen.\\ \hline
	\textbf{Auslösendes Ereignis} &	Nutzer möchte ein Problem lösen.\\ \hline
	\textbf{Akteure}		&	Nutzer\\ \hline
	\textbf{Vorbedingung}	&	Das System bietet zur Lösung dieses Problems eine Schnittstelle.\\ \hline
	\textbf{Nachbedingung}	&	Das System hat die nötigen Informationen für die Lösung des Problems und der Nutzer erhält eine ID, mit welcher er den Status bzw. das Resultat 
						abfragen kann.\\ \hline
	\textbf{Ergebnis}		&	Erfassung der Informationen für die Lösung des Problems\\ \hline
	\textbf{Hauptszenario}	&	\begin{enumerate}
					\item Der Nutzer ruft die Funktion für das zu lösende Problem mit den Parametern auf.
					\item Das System speichert die Parameter für die weitere Verarbeitung.
					\item Der Nutzer erhält eine ID für das Abrufen des Status bzw. des Resultats.
					\end{enumerate}
					\\ \hline
	\textbf{Alternativszenarien}	&	\begin{enumerate}
					\item[3a] Der Nutzer erhält eine Fehlermeldung, wenn das Problem nicht korrekt erfasst werden konnte.
					\end{enumerate}
					\\ \hline
  \end{tabular}
   \caption{Use Case UC-1: Lösung beauftragen}\label{table:use_case_1}
\end{table}

\begin{table}[ht]
\centering
  \begin{tabular}{ l | p{10cm} }
	\hline
	\rowcolor{gray}
	\textbf{Bezeichner}	&	\textbf{UC-2}\\ \hline
	\textbf{Name}			&	Berechnung starten\\ \hline
	\textbf{Beschreibung}	&	Das System startet die Berechnung beim Verarbeitungssystem.\\ \hline
	\textbf{Auslösendes Ereignis} &	Nutzer möchte ein Problem lösen.\\ \hline
	\textbf{Akteure}		&	Verarbeitungssystem\\ \hline
	\textbf{Vorbedingung}	&	Die Informationen für die Lösung des Problems sind erfasst.\\ \hline
	\textbf{Nachbedingung}	&	Das Verarbeitungssystem beginnt mit der Berechnung.\\ \hline
	\textbf{Ergebnis}		&	Starten der Berechnung\\ \hline
	\textbf{Hauptszenario}	&	\begin{enumerate}
					\item Das System startet die Berechnung. 
					\item Das System übergibt eine ID für das Abrufen der abgelegten Daten.
					\end{enumerate}
					\\ \hline
	\textbf{Alternativszenarien}	&	\begin{enumerate}
					\item[2a] Das System speichert die Fehlermeldung, falls das Starten der Berechnung fehlschlägt.
					\end{enumerate}
					\\ \hline
  \end{tabular}
   \caption{Use Case UC-2: Berechnung starten}\label{table:use_case_2}
\end{table}

\begin{table}[ht]
\centering
  \begin{tabular}{ l | p{10cm} }
	\hline
	\rowcolor{gray}
	\textbf{Bezeichner}	&	\textbf{UC-3}\\ \hline
	\textbf{Name}			&	Eingabe Parameter abholen\\ \hline
	\textbf{Beschreibung}	&	Das Verarbeitungssystem benötigt für die Lösung des Problems die Eingabeparameter.\\ \hline
	\textbf{Auslösendes Ereignis}&	Das Verarbeitungssystem startet eine neue Berechnung.\\ \hline
	\textbf{Akteure}		&	Verarbeitungssystem\\ \hline
	\textbf{Vorbedingung}	&	Das Verarbeitungssystem wurde angestossen, das Problem zu lösen.\\ \hline
	\textbf{Nachbedingung}	&	Das Verarbeitungssystem hat die Eingabeparameter erhalten.\\ \hline
	\textbf{Ergebnis}		&	Erhalt von Eingabeparametern\\ \hline
	\textbf{Hauptszenario}	&	\begin{enumerate}
					\item Das Verarbeitungssystem fordert die Eingabeparameter für das zu lösende Problem an.
					\item Das System leitet die Eingabeparameter weiter.
					\item Das Verarbeitungssystem erhält die Eingabeparameter.
					\end{enumerate}
					\\ \hline
	\textbf{Alternativszenarien}	&	\begin{enumerate}
					\item[2a] Das System liefert eine Fehlermeldung zurück, falls keine Eingabeparameter vorhanden sind.
					\end{enumerate}
					\\ \hline
  \end{tabular}
   \caption{Use Case UC-3: Eingabe Parameter abholen}\label{table:use_case_3}
\end{table}


\begin{table}[ht]
\centering
  \begin{tabular}{ l | p{10cm} }
	\hline
	\rowcolor{gray}
	\textbf{Bezeichner}	&	\textbf{UC-4}\\ \hline
	\textbf{Name}			&	Status abfragen\\ \hline
	\textbf{Beschreibung}	&	Der Nutzer kann den Status einer Berechnung abfragen, da die Verarbeitung einige Zeit benötigen könnte.\\ \hline
	\textbf{Auslösendes Ereignis}&	Der Nutzer möchte den Status der Berechnung wissen.\\ \hline
	\textbf{Akteure}		&	Nutzer\\ \hline
	\textbf{Vorbedingung}	&	Der Nutzer hat bereits eine Berechnung beauftragt und kennt die ID.\\ \hline
	\textbf{Nachbedingung}	&	Der Nutzer kennt den Status der Berechnung.\\ \hline
	\textbf{Ergebnis}		&	Kenntnis des Status\\ \hline
	\textbf{Hauptszenario}	&	\begin{enumerate}
					\item Der Nutzer fragt den Status einer Berechnung ab.
					\item Das System fragt den Status in der Datenbank ab.
					\item Das System sendet den Status zurück.
					\item Der Nutzer erhält den Status.
					\end{enumerate}
					\\ \hline
	\textbf{Alternativszenarien}	&	\begin{enumerate}
					\item[3a] Das System sendet eine Fehlermeldung zurück, falls der Status nicht ermittelt werden kann.
					\end{enumerate}
					\\ \hline
  \end{tabular}
   \caption{Use Case UC-4: Status abfragen}\label{table:use_case_4}
\end{table}

\begin{table}[ht]
\centering
  \begin{tabular}{ l | p{10cm} }
	\hline
	\rowcolor{gray}
	\textbf{Bezeichner}	&	\textbf{UC-5}\\ \hline
	\textbf{Name}			&	Resultat speichern\\ \hline
	\textbf{Beschreibung}	&	Um das Resultat nicht zu verlieren, muss das Ergebnis nach der Berechnung gespeichert werden.\\ \hline
	\textbf{Auslösendes Ereignis}&	Das Verarbeitungssystem möchte das Resultat speichern.\\ \hline
	\textbf{Akteure}		&	Verarbeitungssystem\\ \hline
	\textbf{Vorbedingung}	&	Das Verarbeitungssystem hat ein Resultat berechnet.\\ \hline
	\textbf{Nachbedingung}	&	Das Resultat ist gespeichert.\\ \hline
	\textbf{Ergebnis}		&	Speicherung des Resultats\\ \hline
	\textbf{Hauptszenario}	&	\begin{enumerate}
					\item Das Verarbeitungssystem schickt das Resultat der Berechnung an das System.
					\item Das System erhält das Resultat.
					\item Das System speichert das Resultat.
					\item Das System bestätigt das Erhalten des Resultats.
					\item Das Verarbeitungssystem erhält die Bestätigung.
					\end{enumerate}
					\\ \hline
	\textbf{Alternativszenarien}	&	\begin{enumerate}
					\item[4a] Das System liefert eine Fehlermeldung zurück, wenn das Resultat nicht korrekt gespeichert werden konnte.
					\end{enumerate}
					\\ \hline
  \end{tabular}
   \caption{Use Case UC-5: Resultat speichern}\label{table:use_case_5}
\end{table}

\begin{table}[ht]
\centering
  \begin{tabular}{ l | p{10cm} }
	\hline
	\rowcolor{gray}
	\textbf{Bezeichner}	&	\textbf{UC-6}\\ \hline
	\textbf{Name}			&	Resultat abholen\\ \hline
	\textbf{Beschreibung}	&	Der Nutzer holt das Resultat zu einem bestimmten Zeitpunkt ab.\\ \hline
	\textbf{Auslösendes Ereignis}&	Der Nutzer möchte das Resultat der Berechnung abholen.\\ \hline
	\textbf{Akteure}		&	Nutzer\\ \hline
	\textbf{Vorbedingung}	&	Der Nutzer hat bereits eine Berechnung beauftragt und kennt die ID.\\ \hline
	\textbf{Nachbedingung}	&	Der Nutzer hat das Resultat erhalten.\\ \hline
	\textbf{Ergebnis}		&	Erhalt des Resultats\\ \hline
	\textbf{Hauptszenario}	&	\begin{enumerate}
					\item Der Nutzer fragt das Resultat der Berechnung ab.
					\item Das System sucht das Resultat der Berechnung.
					\item Das System sendet das Resultat.
					\item Der Nutzer erhält das Resultat.
					\end{enumerate}
					\\ \hline
	\textbf{Alternativszenarien}	&	\begin{enumerate}
					\item[3a] Das System sendet eine Fehlermeldung, wenn kein Resultat vorhanden ist.
					\end{enumerate}
					\\ \hline
  \end{tabular}
   \caption{Use Case UC-6: Resultat abholen}\label{table:use_case_6}
\end{table}

\newpage
\FloatBarrier
\subsection{Anforderungen}\label{anforderungen}
Alle Anforderungen wurden anhand der folgenden Vorlage (siehe Tabelle \ref{table:req_template}) erfasst. Diese Vorlage basiert auf Angaben von \cite{req_eng_book} und wurde um 
eigene Attribute erweitert.

\begin{table}[ht]
\centering
  \begin{tabular}{ l | p{8cm} }
	\hline
	\rowcolor{gray}
	\textbf{Bezeichner}&	\textbf{Eindeutiger Identifikator}\\ \hline
	\textbf{Priorität} 		&	Must, Should, Nice to have\\ \hline
	\textbf{Anforderungstyp}	&	Funktionale Anforderung, Qualitätsanforderung, Randbedingung\\ \hline
	\textbf{Name} 			&	Eindeutiger, charakterisierender Name\\ \hline
	\textbf{Use Case} 		&	Referenz zum zugehörigen Use Case\\ \hline
	\textbf{Beschreibung} 	&	Beschreibung der Anforderung\\ \hline
	\textbf{Begründung} 		&	Bedeutung der Anforderung für das geplante System\\ \hline
	\textbf{Akzeptanz Kriterium}	&	Messbare Abnahmekriterien\\ \hline
	\textbf{Abhängigkeiten} 	&	Referenz zu anderen Anforderungen\\ \hline
  \end{tabular}
   \caption{Vorlage für Anforderungen}\label{table:req_template}
\end{table}


Die Beschreibung der Anforderungen wurden zusätzlich mit der Satzschablone in \autoref{fig:satzschablone}) aus \cite{req_eng_book} erstellt. Dies hat den Vorteil, dass die 
Anforderungen normiert und exakt sind. Die Abstufungen "`muss"' und "`sollte"' werden verwendet, um die Wichtigkeit der Anforderungen auszudrücken.
\begin{figure}[h]
\includegraphics[scale=0.95]{images/anforderungen/satzschablone.png}
\caption[Satzschablone]{Satzschablone (Grafik entnommen aus \cite{req_eng_book})}
\label{fig:satzschablone}
\end{figure}


\newpage
\FloatBarrier
\subsubsection{Funktionale Anforderungen}\label{func_anforderungen}
Mit den funktionalen Anforderungen wird festgelegt, was die Schnittstelle tun sollte.

\begin{table}[ht]
\centering
  \begin{tabular}{ l | p{8cm} }
	\hline
	\rowcolor{gray}
	\textbf{Bezeichner}&	\textbf{RE-F1}\\ \hline
	\textbf{Priorität} 		&	Must\\ \hline
	\textbf{Anforderungstyp}	&	Funktionale Anforderung\\ \hline
	\textbf{Name} 			&	Bereitstellung der Schnittstellen für Probleme mit hoher Laufzeitkomplexität\\ \hline
	\textbf{Use Case} 		&	\nameref{table:use_case_1}\\ \hline
	\textbf{Beschreibung} 	&	Das System muss dem Nutzer die Möglichkeit bieten, die Lösung verschiedener Probleme zu beauftragen.\\ \hline
	\textbf{Begründung} 		&	Die Schnittstellen ist die Anlaufstelle des Nutzers. Er beauftragt das System, eine Berechnung zu starten.\\ \hline
	\textbf{Akzeptanz Kriterium}	&	\begin{enumerate}
					\item Der Nutzer kann eine Schnittstelle für die bereitgestellten Berechnungsfunktionen ansprechen.
					\end{enumerate}
					\\ \hline
	\textbf{Abhängigkeiten} 	&	-\\ \hline
  \end{tabular}
   \caption{Anforderung RF-F1}\label{table:req_1}
\end{table}

\begin{table}[ht]
\centering
  \begin{tabular}{ l | p{8cm} }
	\hline
	\rowcolor{gray}
	\textbf{Bezeichner}&	\textbf{RE-F2}\\ \hline
	\textbf{Priorität} 		&	Must\\ \hline
	\textbf{Anforderungstyp}	&	Funktionale Anforderung\\ \hline
	\textbf{Name} 			&	Speicherung der Eingabeparameter\\ \hline
	\textbf{Use Case} 		&	\nameref{table:use_case_1}\\ \hline
	\textbf{Beschreibung} 	&	Falls eine Berechnung in Auftrag gegeben wurde, muss das System fähig sein, die Eingabeparameter abzuspeichern.\\ \hline
	\textbf{Begründung} 		&	Die Berechnung wird von einem anderen System ausgeführt. Damit diese auf die Parameter zugreifen können, müssen die Daten persistiert werden.\\ \hline
	\textbf{Akzeptanz Kriterium}	&	\begin{enumerate}
					\item Die Parameter sind persistent.
					\item Es gibt eine Fehlermeldung, falls bei der Speicherung etwas fehlschlägt oder die Eingabeparameter nicht gültig sind.
					\end{enumerate}
					\\ \hline
	\textbf{Abhängigkeiten} 	&	\\ \hline
  \end{tabular}
   \caption{Anforderung RF-F2}\label{table:req_3}
\end{table}

\begin{table}[ht]
\centering
  \begin{tabular}{ l | p{8cm} }
	\hline
	\rowcolor{gray}
	\textbf{Bezeichner}&	\textbf{RE-F3}\\ \hline
	\textbf{Priorität} 		&	Must\\ \hline
	\textbf{Anforderungstyp}	&	Funktionale Anforderung\\ \hline
	\textbf{Name} 			&	Rückgabe einer ID bei Beauftragung\\ \hline
	\textbf{Use Case} 		&	\nameref{table:use_case_1}\\ \hline
	\textbf{Beschreibung} 	&	Falls eine Berechnung in Auftrag gegeben wurde, muss das System dem Ersteller eine ID zurückliefern.\\ \hline
	\textbf{Begründung} 		&	Die ID hilft dem Nutzer den Status der Berechnung abzuholen und wird am Schluss für das Resultat benötigt.\\ \hline
	\textbf{Akzeptanz Kriterium}	&	\begin{enumerate}
					\item Der Nutzer erhält nach dem Start einer Berechnung eine ID.
					\end{enumerate}
					\\ \hline
	\textbf{Abhängigkeiten} 	&	\nameref{table:req_3}\\ \hline
  \end{tabular}
   \caption{Anforderung RF-F3}\label{table:req_2}
\end{table}

\begin{table}[ht]
\centering
  \begin{tabular}{ l | p{8cm} }
	\hline
	\rowcolor{gray}
	\textbf{Bezeichner}&	\textbf{RE-F4}\\ \hline
	\textbf{Priorität} 		&	Must\\ \hline
	\textbf{Anforderungstyp}	&	Funktionale Anforderung\\ \hline
	\textbf{Name} 			&	Start der Berechnung\\ \hline
	\textbf{Use Case} 		&	\nameref{table:use_case_2}\\ \hline
	\textbf{Beschreibung} 	&	Falls eine Berechnung in Auftrag gegeben wurde, muss das System fähig sein, die Berechnung beim Verarbeitungssystem zu starten.\\ \hline
	\textbf{Begründung} 		&	Der Nutzer kennt das Verarbeitungssystem nicht, das System muss dem Verarbeitungssystem den Start-Befehl geben.\\ \hline
	\textbf{Akzeptanz Kriterium}	&	\begin{enumerate}
					\item Der Befehl für den Start wird versendet und die ID dabei übergeben.
					\item Die Fehlermeldung bei einem Fehlversuch wird gespeichert.
					\end{enumerate}
					\\ \hline
	\textbf{Abhängigkeiten} 	&	\nameref{table:req_3}\\ \hline
  \end{tabular}
   \caption{Anforderung RF-F4}\label{table:req_4}
\end{table}

\begin{table}[ht]
\centering
  \begin{tabular}{ l | p{8cm} }
	\hline
	\rowcolor{gray}
	\textbf{Bezeichner}&	\textbf{RE-F5}\\ \hline
	\textbf{Priorität} 		&	Must\\ \hline
	\textbf{Anforderungstyp}	&	Funktionale Anforderung\\ \hline
	\textbf{Name} 			&	Abfrage der Eingabeparameter\\ \hline
	\textbf{Use Case} 		&	\nameref{table:use_case_3}\\ \hline
	\textbf{Beschreibung} 	&	Falls eine Berechnung in Auftrag gegeben wurde, muss das System dem Verarbeitungssystem die Möglichkeit bieten, die Eingabeparameter abzufragen.\\ \hline
	\textbf{Begründung}		&	Damit das Verarbeitungssystem die Berechnung durchführen kann, braucht es die Eingabeparameter.\\ \hline
	\textbf{Akzeptanz Kriterium}	&	\begin{enumerate}
					\item Das Verarbeitungssystem erhält die Eingabeparameter.
					\item Das Verarbeitungssystem erhält eine Fehlermeldung, falls keine Eingabeparameter vorhanden sind.
					\end{enumerate}
					\\ \hline
	\textbf{Abhängigkeiten} 	&	\nameref{table:req_3}\\ \hline
  \end{tabular}
   \caption{Anforderung RF-F5}\label{table:req_5}
\end{table}

\begin{table}[ht]
\centering
  \begin{tabular}{ l | p{8cm} }
	\hline
	\rowcolor{gray}
	\textbf{Bezeichner}&	\textbf{RE-F6}\\ \hline
	\textbf{Priorität} 		&	Should\\ \hline
	\textbf{Anforderungstyp}	&	Funktionale Anforderung\\ \hline
	\textbf{Name} 			&	Abfrage des Status\\ \hline
	\textbf{Use Case} 		&	\nameref{table:use_case_4}\\ \hline
	\textbf{Beschreibung} 	&	Falls eine Berechnung in Auftrag gegeben wurde, sollte das System dem Nutzer die Möglichkeit bieten, den Status der Berechnung abzufragen.\\ \hline
	\textbf{Begründung} 		&	Da die Verarbeitung asynchron läuft, weiss der Benutzer nicht, wann seine Berechnung fertig ist.\\ \hline
	\textbf{Akzeptanz Kriterium}	&	\begin{enumerate}
					\item Der Nutzer erhält einen Status seiner Berechnung.
					\end{enumerate}
					\\ \hline
	\textbf{Abhängigkeiten} 	&	\nameref{table:req_2}\\ \hline
  \end{tabular}
   \caption{Anforderung RF-F6}\label{table:req_6}
\end{table}

\begin{table}[ht]
\centering
  \begin{tabular}{ l | p{8cm} }
	\hline
	\rowcolor{gray}
	\textbf{Bezeichner}&	\textbf{RE-F7}\\ \hline
	\textbf{Priorität} 		&	Nice to have\\ \hline
	\textbf{Anforderungstyp}	&	Funktionale Anforderung\\ \hline
	\textbf{Name} 			&	Registrierung eines Dienstes zur Benachrichtigung für Statusänderungen\\ \hline
	\textbf{Use Case} 		&	\nameref{table:use_case_4}\\ \hline
	\textbf{Beschreibung} 	&	Falls eine Berechnung in Auftrag gegeben und dazu ein Dienst zur Benachrichtigung eingetragen wurde, sollte das System den Nutzer über eine 
						Änderung des Status mittels dieses Dienstes informieren.\\ \hline
	\textbf{Begründung} 		&	Da die Verarbeitung lange dauern könnte, weiss der Benutzer nicht, wann seine Berechnung fertig ist. Um ein ständiges Pollen zu vermeiden, 
							können Dienst zur Benachrichtigung (zum Beispiel WebHooks) verwenden werden.\\ \hline
	\textbf{Akzeptanz Kriterium}	&	\begin{enumerate}
					\item Der Nutzer wird über die Änderung des Status auf dem eingetragen Dienst informiert.
					\end{enumerate}
					\\ \hline
	\textbf{Abhängigkeiten} 	&	\nameref{table:req_2}\\ \hline
  \end{tabular}
   \caption{Anforderung RF-F7}\label{table:req_7}
\end{table}

\begin{table}[ht]
\centering
  \begin{tabular}{ l | p{8cm} }
	\hline
	\rowcolor{gray}
	\textbf{Bezeichner}&	\textbf{RE-F8}\\ \hline
	\textbf{Priorität} 		&	Must\\ \hline
	\textbf{Anforderungstyp}	&	Funktionale Anforderung\\ \hline
	\textbf{Name} 			&	Speicherung des Resultats\\ \hline
	\textbf{Use Case} 		&	\nameref{table:use_case_5}\\ \hline
	\textbf{Beschreibung} 	&	Nach der Berechnung muss das System dem Verarbeitungssystem die Möglichkeit bieten, das Resultat abspeichern zu können.\\ \hline
	\textbf{Begründung} 		&	Das Resultat muss, bis der Nutzer es abholt, zwischengespeichert werden.\\ \hline
	\textbf{Akzeptanz Kriterium}	&	\begin{enumerate}
					\item Das Verarbeitungssystem kann das Resultat abspeichern.
					\item Das Verarbeitungssystem erhält eine Fehlermeldung, falls das Speichern fehlgeschlagen ist.
					\end{enumerate}
					\\ \hline
	\textbf{Abhängigkeiten} 	&	\nameref{table:req_4}\\ \hline
  \end{tabular}
   \caption{Anforderung RF-F8}\label{table:req_8}
\end{table}

\begin{table}[ht]
\centering
  \begin{tabular}{ l | p{8cm} }
	\hline
	\rowcolor{gray}
	\textbf{Bezeichner}&	\textbf{RE-F9}\\ \hline
	\textbf{Priorität} 		&	Must\\ \hline
	\textbf{Anforderungstyp}	&	Funktionale Anforderung\\ \hline
	\textbf{Name} 			&	Abfrage des Resultats\\ \hline
	\textbf{Use Case} 		&	\nameref{table:use_case_6}\\ \hline
	\textbf{Beschreibung} 	&	Das System muss dem Nutzer die Möglichkeit bieten, das Resultat der Berechnung abzufragen.\\ \hline
	\textbf{Begründung} 		&	Der Nutzer möchte das Resultat der Berechnung wissen.\\ \hline
	\textbf{Akzeptanz Kriterium}	&	\begin{enumerate}
					\item Der Nutzer erhält das Resultat der Berechnung.
					\item Der Nutzer erhält eine entsprechende Fehlermeldung, wenn beim Bereitstellen des Resultats ein Fehler aufgetreten ist.
					\end{enumerate}
					\\ \hline
	\textbf{Abhängigkeiten} 	&	\nameref{table:req_2}\\ \hline
  \end{tabular}
   \caption{Anforderung RF-F9}\label{table:req_9}
\end{table}

\newpage
\FloatBarrier
\subsubsection{Qualitätsanforderung}\label{non_func_anforderungen}
Die erfassten Qualitätsanforderungen definieren, was für Eigenschaften die Schnittstelle haben sollte.

\begin{table}[ht]
\centering
  \begin{tabular}{ l | p{8cm} }
	\hline
	\rowcolor{gray}
	\textbf{Bezeichner}&	\textbf{RE-NF1}\\ \hline
	\textbf{Priorität} 		&	Should\\ \hline
	\textbf{Anforderungstyp}	&	Qualitätsanforderung\\ \hline
	\textbf{Name} 			&	Prozess-agnostische Schnittstelle\\ \hline
	\textbf{Use Case} 		&	\nameref{table:use_case_1}\\ \hline
	\textbf{Beschreibung} 	&	Das System sollte fähig sein, die Lösung eines Problems so bereitzustellen, dass kein Wissen über den Verarbeitungsprozess erforderlich ist.\\ \hline
	\textbf{Begründung} 		&	Der Verarbeitungsprozess kann spezifisch und von Problem zu Problem unterschiedlich sein, der Nutzer sollte eine möglichst einfache 
							Schnittstelle dafür haben.\\ \hline
	\textbf{Akzeptanz Kriterium}	&	\begin{enumerate}
					\item Das Interface kann verwendet werden, ohne dass das Verarbeitungssystem bekannt ist.
					\end{enumerate}
					\\ \hline
	\textbf{Abhängigkeiten} 	&	-\\ \hline
  \end{tabular}
   \caption{Qualitätsanforderung RF-NF1}\label{table:req_nf_1}
\end{table}

\begin{table}[ht]
\centering
  \begin{tabular}{ l | p{8cm} }
	\hline
	\rowcolor{gray}
	\textbf{Bezeichner}&	\textbf{RE-NF2}\\ \hline
	\textbf{Priorität} 		&	Should\\ \hline
	\textbf{Anforderungstyp}	&	Qualitätsanforderung\\ \hline
	\textbf{Name} 			&	Entgegennahme generischer Eingabeparameter\\ \hline
	\textbf{Use Case} 		&	\nameref{table:use_case_1}\\ \hline
	\textbf{Beschreibung} 	&	Das System sollte in der Lage sein, unterschiedliche Ausprägungen von Eingabeparametern entgegenzunehmen.\\ \hline
	\textbf{Begründung} 		&	Da es bei den Problemen unterschiedliche Ausprägungen gibt, ist auf eine generische Deserialisierung der Eingabeparameter hinzuarbeiten.\\ \hline
	\textbf{Akzeptanz Kriterium}	&	\begin{enumerate}
					\item Unterschiedliche Ausprägungen eines Problems benutzen das gleiche API.
					\end{enumerate}
					\\ \hline
	\textbf{Abhängigkeiten} 	&	-\\ \hline
  \end{tabular}
   \caption{Qualitätsanforderung RF-NF2}\label{table:req_nf_2}
\end{table}

\begin{table}[ht]
\centering
  \begin{tabular}{ l | p{8cm} }
	\hline
	\rowcolor{gray}
	\textbf{Bezeichner}&	\textbf{RE-NF3}\\ \hline
	\textbf{Priorität} 		&	Should\\ \hline
	\textbf{Anforderungstyp}	&	Qualitätsanforderung\\ \hline
	\textbf{Name} 			&	Speicherung generischer Eingabeparameter\\ \hline
	\textbf{Use Case} 		&	\nameref{table:use_case_1}\\ \hline
	\textbf{Beschreibung} 	&	Das System sollte Eingabeparameter einheitlich abspeichern.\\ \hline
	\textbf{Begründung} 		&	Da es viele unterschiedliche Probleme gibt, ist eine generische Persistierung anzustreben.\\ \hline
	\textbf{Akzeptanz Kriterium}	&	\begin{enumerate}
					\item Unterschiedliche Probleme haben kein abweichendes Persistierungsschema.
					\end{enumerate}
					\\ \hline
	\textbf{Abhängigkeiten} 	&	-\\ \hline
  \end{tabular}
   \caption{Qualitätsanforderung RF-NF3}\label{table:req_nf_3}
\end{table}

\begin{table}[ht]
\centering
  \begin{tabular}{ l | p{8cm} }
	\hline
	\rowcolor{gray}
	\textbf{Bezeichner}&	\textbf{RE-NF4}\\ \hline
	\textbf{Priorität} 		&	Should\\ \hline
	\textbf{Anforderungstyp}	&	Qualitätsanforderung\\ \hline
	\textbf{Name} 			&	Erweiterbare Schnittstelle\\ \hline
	\textbf{Use Case} 		&	-\\ \hline
	\textbf{Beschreibung} 	&	Das System sollte einfach erweiterbar sein.\\ \hline
	\textbf{Begründung} 		&	Falls ein Nachfrage für ein anderes Problem besteht, wäre es gut, wenn die Schnittstelle mit wenig Aufwand erweitert werden könnte.\\ \hline
	\textbf{Akzeptanz Kriterium}	&	\begin{enumerate}
					\item Die Schnittstelle kann mit wenig Aufwand erweitert werden.
					\end{enumerate}
					\\ \hline
	\textbf{Abhängigkeiten} 	&	-\\ \hline
  \end{tabular}
   \caption{Qualitätsanforderung RF-NF4}\label{table:req_nf_4}
\end{table}

\FloatBarrier
\clearpage
\newpage

\subsection{Zusammenfassung der Anforderungen}\label{toc_anfoderungen}
Die Priorität der einzelnen Anforderungen ist wichtig, falls nicht alle Anforderungen umgesetzt werden können. Die Priorität wurde zusammen mit den \glslink{stakeholder}{Stakeholdern} 
festgelegt und in \autoref{table:req_priorities} zur besseren Übersicht zusammengetragen. Falls es nicht möglich gewesen wäre, alle Anforderungen in der Zeit umzusetzen, wären sie anhand 
dieser Listen zurückgestellt worden.

\begin{table}[ht]
\centering
  \begin{tabular}{ l | p{9cm}| l }
	\hline
	\rowcolor{gray}
	\textbf{Bezeichner}	& \textbf{Name}	&	\textbf{Priorität}\\ \hline
	RE-F1 			&  Bereitstellung der Schnittstellen für Probleme mit hoher Laufzeitkomplexität	& Must\\ \hline
	RE-F2 			&  Speicherung der Eingabeparameter	& Must\\ \hline
	RE-F3 			&  Rückgabe einer ID bei Beauftragung	& Must\\ \hline
	RE-F4 			&  Start der Berechnung	& Must\\ \hline
	RE-F5 			&  Abfrage der Eingabeparameter	& Must\\ \hline
	RE-F6 			&  Abfrage des Status	& Should\\ \hline
	RE-F7 			&  Registrierung eines Dienstes zur Benachrichtigung für Statusänderungen	& Nice to have\\ \hline
	RE-F8 			&  Speicherung des Resultats	& Must\\ \hline
	RE-F9 			&  Abfrage des Resultats	& Must\\ \hline
	RE-NF1 			&  Prozess-agnostische Schnittstelle & Should\\ \hline
	RE-NF2 			&  Entgegennahme generischer Eingabeparameter & Should\\ \hline
	RE-NF3 			&  Speicherung generischer Eingabeparameter & Should\\ \hline
	RE-NF4 			&  Erweiterbare Schnittstelle & Should\\ \hline	
  \end{tabular}
   \caption{Priorität der Anforderungen}\label{table:req_priorities}
\end{table}

\subsection{Annahmen}\label{annahmen}
Zwischen dem Endnutzer und dem zu erstellenden System existiert noch eine Sicherheitsschicht, welcher nicht Teil dieser Arbeit ist. Später wird dieses Projekt allenfalls um diese Sicherheitsschicht 
erweitert oder eine übergeordnete Schnittstelle erstellt, welche diese anspricht. 

Je nach Implementierung der Algorithmen ist das Ansteuern und die Aufbereitung der Daten unterschiedlich. Eine Referenzimplementierung zu jedem Problem zu finden, stellte sich als schwierig 
heraus. Deshalb wurden die Ein- und Ausgabe-Schemata der Algorithmen aus der Literaturrecherche abgeleitet.
%%%%%%%%%%%%%%%%%%%%%%%%%%%%%%%%%%%%%%%%%%%%%%%%%%%%%%%%%%%%%%%%%
%
% Project     : Bachelorarbeit
% Title       : Machbarkeitsanalyse für eine ressourcenorientierte Schnittstelle zur Verarbeitung grundlegender Probleme der Informatik
% File        : architektur.tex Rev. 01
% Date        : 01.03.2015
% Author      : Raffael Santschi
%
%%%%%%%%%%%%%%%%%%%%%%%%%%%%%%%%%%%%%%%%%%%%%%%%%%%%%%%%%%%%%%%%%

\chapter{Konzept der Schnittstelle \resultAssignment{[R4]}}\label{chap.architektur}
In diesem Kapitel wird auf die Strukur und das Konzept der Schnittstelle eingegangen. Dieses Konzept legt die Grundlage für die Umsetzung und entscheidet somit auch, ob die Aufgabe gelöst werden kann.

\section{Übersicht}\label{architektur_uebersicht}
Die Systemumgebung (siehe Abbildung \ref{fig:system_scope}) hat zwei Berührungpunkte zur Aussenwelt. Der eine ist zum Nutzer hin, der andere zu einem Verarbeitungssystem. Um die Daten zu speichern, wird eine Datenbank benötigt. Das ermöglicht einen asynchronen Ablauf und ein mehrfaches Abfragen der Daten.
\begin{figure}[h]
\centering
\includegraphics[scale=0.8]{images/visio/SystemScope.png}
\caption{System Übersicht (Eigene Darastellung)}
\label{fig:system_scope}
\end{figure}

\section{Konzept}\label{arch_backend}
Das Konzept sollte simpel und flexibel sein, damit das System möglichst schnell erweitert werden kann. Beim Betrachten der Probleme und der Abläufe wurde bemerkt, dass die Vorgänge Ähnlichkeiten aufweisen. Die Daten werden angenommen und für einen bestimmten Algorithmus umgewandelt. Das Verarbeitungssystem schickt das Resultat zurück und dieses wird dann wieder umgewandelt, so dass es für den Nutzer brauchbar ist. Es finden also zwei Umwandlungen staht. Die Umwandlungen an sich sind von Problem zu Problem verschieden, jedoch gibt es auch dort gewiese Ähnlichkeiten. In \autoref{fig:architektur} wird der Aufbau des System gezeigt.

\begin{figure}[h]
\centering
\includegraphics[scale=0.8]{images/visio/architektur_db.png}
\caption{Architekturaufbau des Systems (Eigene Darastellung)}
\label{fig:architektur}
\end{figure}
 
\FloatBarrier
\subsection{REST API}
Für die Schnittstelle wird ein REST API erstellt, welches die nötigen Funktionen bietet. Das API funktioniert nach dem De-facto-Standard (siehe \cite{wiki_restful}). Falls Fehler auftretten werden die HTTP-Statuscode verwendet, um diese an den Aufrufer weiterzugeben.

\subsection{Business Logik}
Die Business Logik besteht aus Translator, Service, Entity und Validator von den jeweiligen Problemen. Es gibt jeweils ein ProblemTranslator für die Umwandlung hin zum Algorithmus und einen SolutionTranslator für die Umwandlung vom Algorithmus zurück in das System. Der Service kann sehr generisch gehalten werden und benötigt nichts problemspezifisches. Die Entitäten sind von Problem zu Problem unterschiedlich, es muss analysiert werden, wie die Parameter am besten eingegeben werden und wie diese dann vom Algorithmus gebraucht werden. Der Validator entscheidet, ob eine Lösung gültig ist oder nicht. Wie bereits im \autoref{cat_theo_inf} erklärt, kann jedes NP-vollständige Problem kann in polynomialer Zeit validiert werden.

\subsection{Persistance API}
Die Abstraktion der Datenbank wird mittels eines Persistance APIs, welches mit der Datenbank interagiert, realisiert.

\subsection{Datenbank}
Jedes Problem hat seine eigene Ausprägung von Problem und Solution, welche abgespeichert werden müssen. Die Datenbank sollte wenn möglich eine ähnliche Flexibilität wie das Programm selber aufweisen. Die Vorgänge benötigen keine Transaktionen und sind zum grossen Teil nur Einfüge-Operationen, nur selten wird ein Einträg geändert.

\begin{landscape}
\subsection{Ablauf}
\thispagestyle{empty}
In \autoref{fig:workflow} wird der Ablauf des ganzen Vorganges und die Interaktion mit dem Nutzer und dem Verarbeitungssystem nochmals verdeutlicht.

\begin{figure}[h]
\centering
\includegraphics[scale=0.8]{images/visio/workflow.png}
\caption{Flussdiagramm des Arbeitsablaufs (Eigene Darastellung)}
\label{fig:workflow}
\end{figure}

\end{landscape}


Die \autoref{fig:sequenz_diagramm_start} zeigt das Sequenzdiagramm für den Start eines beliebigen Problems, alle Komponenten mit '{Problem}' sind spezifische Problem-Komponenten, die anderen sind generisch. Bei Speichern des Problems, wird das Problem mit dem Status 'CREATED' gespeichert. Nach dem Speichern wird über eine Solver-Komponente das Verarbeitungssystem asynchron gestartet. Das Verarbeitungssystem holt sich nun die benötigten Informationen. Der Controller lädt das Problem vom Service, dieser wiederum lädt es vom Repository und wandelt es für den Algorithmus um.

\begin{figure}[h]
\centering
\includegraphics[scale=0.8]{images/visio/sequenz_diagramm_start.png}
\caption{Start eines belieben Problemes (Eigene Darastellung)}
\label{fig:sequenz_diagramm_start}
\end{figure}

Die \autoref{fig:sequenz_diagramm_result} zeigt das Sequenzdiagramm für das Abspeichern des Resultates eines beliebigen Problems. Bei einem Speicheraufruf wird zuerst das Problem geladen, danach wird die Lösung vom Algorithmus mit Hilfe der Eingabeparameter transferiert und vom Validator validiert. Das Resultat wird dann mit dem Ergebnis der Validierung in die Datenbank gespeichert. Wenn der Nutzer das Resultat abholt, sucht der Controller über den Service ein Endresultat für die Berechnung und liefert dieses dann dem Nutzer.

\begin{figure}[h]
\centering
\includegraphics[scale=0.8]{images/visio/sequenz_diagramm_result.png}
\caption{Abspeichern des Resultates eines belieben Problemes (Eigene Darastellung)}
\label{fig:sequenz_diagramm_result}
\end{figure}

\section{Datenbank Varianten}\label{db_varianten}
Es gibt verschiedene Datenbanktypen und jede hat seine Vor- und Nachteile. In diesem Abschnitt werden drei verschiedenen Typen miteinander verglichen und die beste für diesen Anwendungszweck ausgewählt.

\subsection{Relationales Datenbanksystem}\label{rdbms}


\subsection{Objektorientiertes Datenbanksystem}\label{object_db}

\subsection{NoSQL Datenbanksystem}\label{no_sql_db}


\newpage
\subsection{Nutzwertanalyse}\label{architektur_nutzwertanalyse}

\subsubsection{Bewertungskriterien}\label{architektur_bewertungspunkte}

In der Nutzwertanalyse werden folgende Punkte betrachtet und nach dem angegebenen Schema bewertet und dann gewichtet. Die Kriterien sind grösstenteils aus den Softwarequalitäsmerkmalen nach \cite{iso_9126} abgeleitet.

\paragraph{Aufwand}
\begin{itemize}
	\item \textbf{Beschreibung}: Wie gross ist der geschätzte Aufwand mit dieser Methode?
	\item \textbf{Bewertung}: 1: sehr hoch, 10: sehr niedrig
	\item \textbf{Gewichtung}: 5 (Die Zeit für dieses Projekt ist beschränkt und die Entscheidung könnte zu einem Risiko werden)
\end{itemize}

\paragraph{Benutzbarkeit}
\begin{itemize}
	\item \textbf{Beschreibung}: Wie lange dauert die Einarbeitungsphase? 
	\item \textbf{Bewertung}: 1: sehr langsam, 10: sehr schnell
	\item \textbf{Gewichtung}: 4 (Umso mehr Zeit für das Erlernen investiert wird, umso weniger Zeit ist für die Implementierung vorhanden)
\end{itemize}

\paragraph{Änderbarkeit}
\begin{itemize}
	\item \textbf{Beschreibung}: Wie flexibel ist diese Variante in Bezug auf Erweiterungen/Vererbung?
	\item \textbf{Bewertung}: 1: sehr spezifisch, nicht portabel, 10: sehr flexibel
	\item \textbf{Gewichtung}: 5 (Die Anforderungen geben vor, dass das System einfach zu erweitern sein sollte)
\end{itemize}

\paragraph{Funktionalität}
\begin{itemize}
	\item \textbf{Beschreibung}: Wie gross ist der Funktionalitätsumfang?
	\item \textbf{Bewertung}: 1: sehr eingeschränkt, 10: sehr weit reichend
	\item \textbf{Gewichtung}: 1 (In der ersten Version der App wird noch nicht viel Funktionalität gebraucht)
\end{itemize}

\newpage
\subsubsection{Bewertung}\label{architektur_bewertung}


\FloatBarrier
\subsubsection{Fazit}\label{architektur_fazit}
Das Resultat der Nutzweranalyse ist eindeutig, die Entwicklung mittels Phonegap ist für dieses Projekt mit Abstand die beste Lösung. Die Methode hat noch weitere Vorteile, durch dass das der Code nur in eine App verpackt wird und er nicht in eine andere Sprache übersetzt wird, gibt es viel weniger Overhead. Zudem ist es möglich den Source auch in der App ohne Problem anzupassen, dass heisst, falls der Convert nicht mehr funktioniert, weil er zum Beispiel nicht mehr unterhalten wird, kann man torztdem noch weiterentwickeln und muss nicht zu erst ein Reverse Engineering machen.

\newpage
\section{Fazit}\label{architektur_fazit}

Obwohl die Probleme auf den ersten Blick sehr unterschiedlich sind, bleiben die groben Abläufe gleich. Somit kann einiges generalisiert werden. Ein neues Problem heisst schlussendlich neue Entities, zwei neue Translator, ein neuer Validator und ein neuer Einstiegspunkt für den Nutzer.  Controller, Services und Repositories werden generisch gehalten.



 
%%%%%%%%%%%%%%%%%%%%%%%%%%%%%%%%%%%%%%%%%%%%%%%%%%%%%%%%%%%%%%%%%
%
% Project     : Bachelorarbeit
% Title       : Machbarkeitsanalyse für eine ressourcenorientierte Schnittstelle zur Verarbeitung grundlegender Probleme der Informatik
% File        : umsetzung.tex Rev. 01
% Date        : 01.03.2015
% Author      : Raffael Santschi
%
%%%%%%%%%%%%%%%%%%%%%%%%%%%%%%%%%%%%%%%%%%%%%%%%%%%%%%%%%%%%%%%%%

\chapter{Umsetzung des Prototyps \resultAssignment{[R5]}}\label{chap.umsetzung}
In diesem Kapitel wird kurz auf die Erkenntnisse aus dem ersten Durchstich eingegangen. Danach wird erklärt, wie die Umsetzung für die ausgewählten Problemtypen schlussendlich 
durchgeführt wurde. Zu guter Letzt wird noch die Entwicklungsumgebung für dieses Projekt aufgezeigt.

\section{Erster Durchstich}\label{entwicklungsumgebung}
Zum Start der Umsetzung wurde ein erster \glossarmark{vertikaler Durchstich} anhand des Rucksack Problems gemacht, um zu sehen, ob sich das Konzept bewährt. Während der 
Implementation wurden bereits überlegt, wie die Logik möglichst generisch gehalten werden kann.

\begin{figure}[h]
\centering
\includegraphics[scale=0.74]{images/visio/prototype_knapsack.png}
\caption[Vertikaler Durchstich mit dem Rucksack Problem]{Vertikaler Durchstich mit dem Rucksack Problem \selfmade{}}
\label{fig:prototyp_knapsack}
\end{figure}

\subsection{Erkenntnisse nach dem ersten Durchstich}\label{learning_prototyp}
Die Kontaktschnittstelle nach Aussen generisch zu halten, macht aus Gründen der Benutzerfreundlichkeit keinen Sinn. Für den Kunden ist es transparenter, wenn er eine Schnittstelle aufruft, 
welche zum Beispiel 'timetabelScheduleComputations' heisst, als eine generische Schnittstelle, welche 'computations' heisst. Zudem ist nicht möglich, nur aus den Daten herauszufinden, was für 
ein Problem gelöst werden möchte. Dazu würde es einen zusätzlichen Parameter benötigen. Weiter stellt es sich in Java mit Spring als schwierig heraus, eine Endpunkt bereitzustellen, welcher 
verschiedene Ressourcentypen akzeptiert. Die Validierung der Eingabeparameter würde dadurch zusätzlich erschwert werden.

Das Konzept mit den beiden Translators bietet sehr viel Möglichkeiten und Flexibilität. Es entkoppelt die Nutzer-Schnittstelle komplett von der Schnittstelle für die Algorithmen. Um diese 
Flexibilität ausnutzen zu können, müssen jedoch oft verschiedene Entities für die Nutzer- und Algorithmus-Sicht erstellt werden.

In der Business Logik kann viel generisch gehalten werden. Die Repositories, Services und die Interfaces können generisch programmiert werden und bleiben für alle Probleme gleich. Beim 
Controller kann die Logik in einer abstrakten Klasse definiert werden und somit müssen nur noch die Namen der Schnittstellen in der spezifischen Implementation definiert werden. Es zeigte sich, 
dass sich das Konzept bewährt und die weiteren Probleme dementsprechend gelöst werden können.

\subsection{Anpassung des Konzepts nach dem ersten Durchstich}\label{doings_prototyp}
Das Konzept wurde nach dem ersten Durchstich dahingegen abgeändert, dass die Schnittstellen auf der Nutzer-Seite umbennant wurden und die Algorithmus-Seite einen anderen Namespace 
erhielt. Weiter verschwand die seperate Schnittstelle für das Abfrage des Resultats durch die Kombination mit der Status-Abfrage.

\section{Implementierung der Schnittstelle}\label{impl_interface}
In diesem Abschnitt wird über die Implementation der Schnittstelle im Allgemeinen geschrieben. Wie bereits erwähnt ist der Ablauf bei jedem Problem gleich, dementsprechend verhält sich die 
Schnittstelle im Allgemeinen gleich.

\subsection{Statusabfrage}
Damit der Nutzer möglichst wenig Schnittstellen ansprechen muss, erhält er bei einer Statusabfrage zugleich die vorhandenen Resultate. Der Nutzer erhält nur Resultate mit dem Status 'FINAL'. 
Resultate mit dem Status 'PARTIAL' werden zwar beim Status und der Zeit des zu letzt empfangenen Resultats berücksichtigt, aber nicht herausgegeben. Der Status besitzt neben den 
Resultaten, die ID und den Namen, weiter werden Start- und Endzeit angeben und zusätzlich noch die Zeit des zu letzt empfangenen Resultats. Falls eine Berechnung beendet ist, ist eine 
Begründung der Beendigung zu finden. Die Begründung kann zum Beispiel ein aufgetrettener Fehler während des Starts oder Berechnung sein.

\begin{lstlisting}[language=JSON, caption=Aufbau einer Antwort auf eine Statusabfrage, label=lst:status_response]  
{
  "id: "<Berechnung ID>",
  "name": "<Berechnungsname>",
  "computationStatus": "<Status der Berechnung>",
  "createDate": <Erstelldatum>,
  "finishDate": <Enddatum>,
  "finishedReason": "<Begruendung der Beendigung>",
  "lastResultReceived": <Datum des zu letzt empfangenen Resultats>,
  "solutions": <Resultate>
}
\end{lstlisting}

\subsection{WebHook Möglichkeit}
Die Berechnung können je nach Komplexität sehr lange dauern. Der Nutzer müsste immer wieder den Status abfragen, um zu sehen, ob ein Resultat vorhanden ist. \glossarmark{WebHooks} 
bieten Abhilfe für dieses Problem. Das Verfahren ist nicht standardisiert, ist aber sehr simpel und hilfreich. Beim Start einer Berechnung kann eine URL mitgegeben werden, zu dieser 
wird bei jeder Statusänderung ein POST-Request abgeschickt. Um diese Schnittstelle möglichst generisch zu halten, wurde eine Möglichkeit geschaffen, eine beliebige Payload für den 
POST-Request anzugeben. Die Nachricht des Statusänderung kann nach belieben mit dem Platzhalter '\_\_MESSAGE\_\_' irgendwo in die Payload eingebunden werden. Das Attribut 'message' 
wird in der Nachricht gesetzt, wenn der Platzhalter nirgends verwendet wird. Eine Konfiguraiton für die Chat Applikation 'Slack' würde wie in \autoref{lst:webhook_configuration} 
aussehen. Die Nachrichten bei den Statusänderungen sähen dann wie in \autoref{fig:slack_chat} aus.

\begin{lstlisting}[language=JSON, caption=Beispiel einer WebHook Konfiguration für Slack, label=lst:webhook_configuration]
  ...  
  "webHookConfiguration": {
    "url": "https://hooks.slack.com/services/T0000/B0000/XXX",
    "payload": {
        "text": "__MESSAGE__",
        "channel": "#simplatyser",
        "username": "simplatyser",
        "icon_emoji": ":squirrel:"
    }
  }
 ...
\end{lstlisting}

\begin{figure}[h]
\centering
\includegraphics[scale=0.8]{images/slack_chat.png}
\caption[Nachrichten der Statusänderungen von Berechnungen]{Nachrichten der Statusänderungen von Berechnungen \selfmade{}}
\label{fig:slack_chat}
\end{figure}

\subsection{Technische Umsetzung und Probleme}
Das Fundament der Software war mit Spring-Boot \cite{spring_boot} sehr schnell gebaut, die Anbindung an die MongoDB wurde mit SpringData \cite{spring_data} realisiert. Leider waren 
oftmals die Dokumentationen nicht ausreichend oder nur für die XML-Konfiguration von Spring. Weiter gab es fehlende Teile in SpringData, welche herausgefunden werden mussten. 
So gibt es zum Beispiel zu diesem Zeitpunkt standardmässig keine Möglichkeit ein 'LocalTime'-Objekt oder ein 'LocalDateTime'-Objekt von Java 8 zu persistieren. Dies führte zu einem 
Stackoverflow anstatt einem Fehler, was wiederum die Suche nach dem Fehler erheblich erschwerte. Um diese Probleme zu beheben, musste ein eigener Converter geschrieben und dieser 
dann bei der MongoDB Konfiguration angegeben werden.

Bei der Business Logik wurde geschaut, dass viel mit \glossarmark{Generics} gelöst werden konnte, vorallem bei den Services konnte viel Code für alle Probleme verwendet werden. Auch 
bei den beiden Scheduling Problemen, konnte viel Code-Duplizierung umgangen werden. Zusätzlich wurden häufig verwendete Funktionen, wie das Abbilden auf die Input Objekte, in Helfer 
Klassen ausgelagert, damit sie nur ein Mal implementiert werden mussten.

\section{Implementierung der Probleme}\label{impl_problems}
In diesem Unterkapitel wird für jedes Problem kurz erklärt, was beim Prototyp implementiert wurde, um die Möglichkeiten des Konzepts aufzuzeigen. Eine ausführliche Schnittstellen 
Dokumentation ist in \autoref{api_doc} zu finden. Zusätzlich wurde noch eine elektronische Dokumentation der Schnittstelle mit Hilfe von Swagger UI erstellt, diese stellt die angebotenen 
Schnittstellen, die Eingabeparemter und Resultat sehr übersichtlich dar.

\begin{figure}[h]
\centering
\includegraphics[scale=0.5]{images/swagger_api.png}
\caption[Nutzer Schnittstellenbeschreibung von Swagger]{Nutzer Schnittstellenbeschreibung von Swagger \selfmade{}}
\label{fig:swagger}
\end{figure}

\FloatBarrier

Beim Evaluieren des Stundenplan Problems wurde bemerkt, dass es sehr viele verschiedene Planungsprobleme gibt und deshalb wurde ein zusätzlich Planungsproblem gewählt, 
um zu schauen, wie sich die Schnittstelle bei sehr ähnlichen Problemen verhält.

%%%%%%%%%%%%%%%%%%%%%%%%%%%%%%%
%
%
%		Rucksack
%
%
%%%%%%%%%%%%%%%%%%%%%%%%%%%%%%%

\subsection{Rucksack}
Das Rucksack Problem ist aus Sicht der Schnittstelle ein relativ einfaches Problem. Aus Benutzerfreundlichkeit kann der Nutzer bei den Elementen eine Anzahl definieren und muss sie nicht 
doppelt angeben. Der Algorithmus hingegen bekommt eine Liste mit allen Elementen, es gibt nur noch einzelne Elemente und der Name wird nicht weiter gegeben, da er vom Algorithmus nicht 
benötigt wird. Der Algorithmus liefert als Resultat eine Liste von Boolean-Werten zurück, diese müssen zuerst wieder auf die eigentlichen Elemente abgebildet werden. Bei der Validierung wird 
überprüft, ob die Gewichtsschranke nicht überschritten wurde und ob ein Element nicht zu oft verwendet wurde. Letzeres ist zwar bei dem Test-Algorithmus nicht möglich, könnte jedoch bei 
einer anderen Implementation eventuell der Fall sein. Der Benutzer erhält als Resultat eine Liste mit allen Objekten, welche verwendet wurden. Wenn das Objekt mehrmals verwendet wurde, 
wird die entsprechende Anzahl angegeben.

\begin{figure}[h]
\centering
\includegraphics[scale=0.5]{images/probleme/knapsack.png}
\caption[Eingabeparameter für eine Rucksack Berechnung]{Eingabeparameter für eine Rucksack Berechnung \selfmade{}}
\label{fig:knapsack_input}
\end{figure}

\FloatBarrier

%%%%%%%%%%%%%%%%%%%%%%%%%%%%%%%
%
%
%		Knotenfärbung
%
%
%%%%%%%%%%%%%%%%%%%%%%%%%%%%%%%

\subsection{Knotenfärbung}
Beim Knotenfärbung geht es darum, Kollisionen zu vermeiden. Dies wurde für den Nutzer möglichst transparent umgesetzt. Der Nutzer kann mögliche Werte (z.B. Farben oder 
Frequenzen) angeben, welche die Element zugeteilt bekommen sollen. Der Algorithmus erhält eine Liste mit allen Elementen und ihren Nachbaren. Der Name der Elemente und die möglichen 
Werte werden nicht weiter gegeben. Es wird davon ausgegangen, dass der Algorithmus eine Liste von Elementen mit den zugewiesenen Werten zurückgibt. Beim Übersetzten werden, falls 
vorhanden, die Werte vom Algorithmus mit den möglichen Werten aus der Benutzereingabe ausgetauscht. Bei der Validierung wird überprüft, ob kein Element den gleichen Wert, wie einer 
seiner Nachbaren hat. Als Resultat erhält der Benutzer eine Liste mit allen Elementen und ihren zugewiesenen Werten.

\begin{figure}[h]
\centering
\includegraphics[scale=0.5]{images/probleme/graphcoloring.png}
\caption[Eingabeparameter für eine Knotenfärbung Berechnung]{Eingabeparameter für eine Knotenfärbung Berechnung \selfmade{}}
\label{fig:graphcoloring_input}
\end{figure}

%%%%%%%%%%%%%%%%%%%%%%%%%%%%%%%
%
%
%		Problem des Handlungsreisenden
%
%
%%%%%%%%%%%%%%%%%%%%%%%%%%%%%%%

\subsection{Problem des Handlungsreisenden}
Der Service bietet nicht nur die kürzeste Route für eine Liste von Wegpunkten an, es ist auch möglich für jeden Wegpunkt eine gewünschte Ankunftszeit und Aufenthaltszeit anzugeben. Der 
Nutzer kann eine Liste von Wegpunkten mit gewünschter Ankunftszeit und Aufenthaltszeit angeben. Der Algorithmus erhält vom System eine Liste mit allen Wegpunkten, bei diesem Schritt wird 
nichts übersetzt. Es wäre möglich, dass die Schnittstelle bereits die Distanzen zwischen den Wegpunkten berechnet oder die Daten sonst irgendwie für den Algorithmus aufbereitet. Es 
wird davon ausgegangen, dass der Algorithmus eine Liste von den Wegpunkten in der berechneten Reihenfolge zurückgibt. Beim Übersetzten werden die geplanten Ankunftszeiten berechnet, 
welche dann während der Validierung mittels der maximal angegebenen Abweichung überprüft werden. Der Benutzer erhält als Resultat eine Liste mit der berechneten Reihenfolge der 
Wegpunkte und jeder Wegpunkt besitzt die gewünschte und geplante Ankunftszeit.

\begin{figure}[h]
\centering
\includegraphics[scale=0.5]{images/probleme/tsp.png}
\caption[Eingabeparameter für eine Routen Berechnung]{Eingabeparameter für eine Routen Berechnung \selfmade{}}
\label{fig:tsp_input}
\end{figure}

%%%%%%%%%%%%%%%%%%%%%%%%%%%%%%%
%
%
%		Briefträgerproblem
%
%
%%%%%%%%%%%%%%%%%%%%%%%%%%%%%%%

\subsection{Briefträgerproblem}
Das Briefträgerproblem berechnet eine Route, welche alle bekannten Wege einer Strecke abfährt. Der Nutzer kann eine Liste von Wegpunkten mit den bekannten Verknüpfungen zu anderen 
Wegpunkten und ihrer Dinstanz angeben. Der Algorithmus erhält vom System eine Liste mit allen Wegpunkten, die Namen der Wegpunkte werden nicht weitergegeben. Es wird davon ausgegangen, 
dass der Algorithmus eine Liste von den Wegpunkten in der berechneten Reihenfolge zurückgibt. Beim Übersetzten werden die Wegpunkte wieder auf die Eingabewerte abgebildet und die komplette 
Länge der Route berechnet. Die Validierung überprüft, ob der Weg möglich ist und ob jede Verbindung ein Mal benutzt wurde. Der Benutzer erhält als Resultat eine Liste mit der berechneten 
Reihenfolge der Wegpunkte und die totale Länge der Strecke.

\begin{figure}[h]
\centering
\includegraphics[scale=0.5]{images/probleme/postman.png}
\caption[Eingabeparameter für eine Briefträger Routen Berechnung]{Eingabeparameter für eine Briefträger Routen Berechnung \selfmade{}}
\label{fig:postman_input}
\end{figure}

%%%%%%%%%%%%%%%%%%%%%%%%%%%%%%%
%
%
%		Stundenplan Erstellung
%
%
%%%%%%%%%%%%%%%%%%%%%%%%%%%%%%%

\subsection{Stundenplan Erstellung}
Das Erstellen eines Stundenplans ist ein Planungsproblem, welches sehr vielen Möglichkeiten und Einschränkungen besitzt und somit sehr komplex ist. In dieser Implementation sind bei weiten 
nicht alle Spezialitäten abgedeckt. Es wurde geschaut, dass bereits einige Einschränkungen, wie zum Beispiel freie Tage von Lehrern und blockierte Schulzimmer, miteinbezogen werden. 
Der Nutzer gibt eine Liste von Klassen, Lehrern, Schulzimmern und Schulfächern an. Die Klassen haben eine Grösse und definieren auch, welche Fächer sie besuchen müssen. Die Lehrer haben 
eine Liste mit Schulfächern, welche sie unterrichten können, eine List mit zugehörigen Klassen und eine Definition ihrer freien Tage. Die Klassenzimmer besitzen eine Liste mit möglichen 
Schulfächern und eine Sperrlist, in welcher definiert ist, wann der Raum nicht verfügbar ist. Die Schulfächer können definieren, ob sie eine Raum brauchen, welcher explizit dafür 
bestimmt ist, zum Beispiel Sport in der Turnhalle. Neben den Elementen, welche verplant werden, gibt es Randbedingungen wie die Pausenzeiten, die Lektionsdauer und die Definition, 
wann unterrichtet werden soll. Der Algorithmus erhält eine Liste mit allen Klassen, Lehrern, Schulzimmern und Schulfächern. Die Rahmenbedingungen werden in Zeitschlitze umgewandelt, 
welche vom Algorithmus verplant werden können. Zusätzlich werden die Elemente generischer benannt, damit der Algorithmus nicht verschiedene Konfigurationen für verschiedene 
Planungsprobleme besitzen muss. In \autoref{lst:cat_input_timetableScheduling} ist ein Beispiel für eine Eingabe der Rahmenbedingungen gezeigt, welche durch den Translator in die 36 Zahl 
umgewandelt werden würde. Die Zahl wird aus der Lektionsdauer, den Pausen und den einzelnen Zeitfenstern pro Tag berechnet, die \autoref{table:timeslice_calc} stellt dies visuell dar.

\begin{lstlisting}[language=JSON, caption=Ausschnitt einer Eingabe für das Stundenplanproblem für die Rahmenbedingungen, label=lst:cat_input_timetableScheduling]  
{
  ...
  "configuration": {
    "breakTimeSliceSize": [5, 20, 5, 90, 5, 20, 5],
    "dayTimeSlots": [
      {
        "monday": {"defaultTimes": true},
        "tuesday": {"defaultTimes": true},
        "wednesday": {"from": [8, 20, 0], "to": [12,0,0]},
        "thursday": {"defaultTimes": true},
        "friday": {"defaultTimes": true}
      }
    ],
    "lessonDuration": 45
  }
}
\end{lstlisting}

\begin{table}[ht]
\centering
  \begin{tabular}{ l | c | c | c | c | c }
	\hline
	\rowcolor{gray}
	\textbf{Uhrzeit} 	& \textbf{Mo}	& \textbf{Di} 	& \textbf{Mi}	&  \textbf{Do}	&  \textbf{Fr}\\ \hline
	0820-0905		& 1			& 9			& 17			& 21			& 29		\\ \hline
	0910-0955		& 2			& 10			& 18			& 22			& 30		\\ \hline
	1015-1100		& 3			& 11			& 19			& 23			& 31		\\ \hline
	1105-1150		& 4			& 12			& 20			& 24			& 32		\\ \hline \hline
	1320-1405		& 5			& 13			& -			& 25			& 33		\\ \hline
	1410-1455		& 6			& 14			& -			& 26			& 34		\\ \hline
	1515-1600		& 7			& 15			& -			& 27			& 35		\\ \hline
	1605-1650		& 8			& 16			& -			& 28			& 36		\\ \hline
  \end{tabular}
   \caption{Visuelle Darstellung der Zeitfenster Berechnung}\label{table:timeslice_calc}
\end{table}

\begin{figure}[h]
\centering
\includegraphics[scale=0.5]{images/probleme/timetableSchedule.png}
\caption[Eingabeparameter für eine Stundenplan Berechnung]{Eingabeparameter für eine Stundenplan Berechnung \selfmade{}}
\label{fig:timetableSchedule_input}
\end{figure}

\FloatBarrier

Es wird davon ausgegangen, dass der Algorithmus eine Liste von Planungskombinationen zurück gibt. Eine Kombination besteht aus einer Zeitfensternummer, einem Lehrer, einer Klasse, 
einem Schulfach und einem Klassenzimmer. Beim Übersetzten werden die IDs auf die ursprünglichen Elemente abgebildet und die Zeitfensternummern werden in Uhrzeiten umgewandelt. 
Zusätzlich wird eine Statistik für die Lehrer geführt. Bei der Validierung wird geschaut, ob ein Element zu einer Zeit zwei Mal verplant ist, ob der Lehrer die nötigen Fähigkeiten hat und ob ein 
Klassenzimmer für das Fach ausgelegt ist. Das Resultat für den Nutzer enthält eine Liste mit den berechneten Kombination, sortiert nach Wochentag und Uhrzeit. Die Lehrerstatistik zeigt, wie 
oft ein Lehrer ein bestimmtes Fach unterrichtet und wie viel Stunden er insgesamt unterrichtet.

%%%%%%%%%%%%%%%%%%%%%%%%%%%%%%%
%
%
%		Spielplan Erstellung
%
%
%%%%%%%%%%%%%%%%%%%%%%%%%%%%%%%

\subsection{Spielplan Erstellung}
Eine weitere Ausprägung des Planungsproblem ist das Erstellen eines Spielplans. Der Nutzer gibt eine Liste von Teams, Schiedsrichter, Spielfelder und Kategorien an. Die Teams haben eine 
bestimmte Kategorie. Die Schiedsrichter haben eine Liste mit Kategorien, welche sie leiten können und eine Liste von Teams, zu welchen sie gehören. Die Spielfelder besitzen eine Liste mit 
möglichen Kategorien. Neben den Elementen, welche verplant werden, gibt es Randbedingungen wie die Spieldauer, der Starzeit und die Pausenzeiten. Der Algorithmus erhält eine 
Liste mit allen Teams, Schiedsrichter, Spielfelder und Kategorien. Zusätzlich werden die Elemente generischer benannt, damit der Algorithmus nicht verschiedene Konfigurationen für 
verschiedene Planungsprobleme besitzen muss.

\begin{figure}[h]
\centering
\includegraphics[scale=0.5]{images/probleme/matchSchedule.png}
\caption[Eingabeparameter für eine Spielplan Berechnung]{Eingabeparameter für eine Spielplan Berechnung \selfmade{}}
\label{fig:matchschedule_input}
\end{figure}

Es wird davon ausgegangen, dass der Algorithmus eine Liste von Planungskombinationen zurück gibt. Eine Kombination besteht aus einer Zeitfensternummer, einem Schiedsrichter, 
einer Kombination von zwei Teams und einem Spielfeld. Beim Übersetzten werden die IDs auf die ursprünglichen Elemente abgebildet und die Zeitfensternummern werden in Uhrzeiten 
umgewandelt. Zusätzlich wird eine Statistik für die Schiedsrichter und die Teams geführt, wie viele Einsätze sie jeweils haben. Bei der Validierung wird geschaut, ob ein Element zu einer Zeit zwei 
Mal verplant ist, ob die Schiedsrichter die nötigen Fähigkeiten haben und ob ein Spielfeld für die Kategorie ausgelegt ist. Der Benutzer erhält als Resultat eine Liste mit den berechneten 
Kombination, sortiert nach Uhrzeit. Die Schiedsrichterstatistik zeigt, wie oft ein Schiedsrichter eine bestimmte Kategorie pfeifft und für wie viel Spiele er insgesamt verantwortlich ist. Die 
Teamstatistik zeigt, wie viele Spiele eine Kategorie insgesamt hat und wie viele Spiele ein Team bestreitet. In \autoref{lst:cat_result_matchScheduling} wird eine Ausschnitt der 
Schiedsrichterstatistik dargestellt.

\begin{lstlisting}[language=JSON, caption=Ausschnitt eines Resultats einer Spielplan Erstellung, label=lst:cat_result_matchScheduling]  
{
  ...
  name: "Sabine Pfister"
  statisticMap: {
    Knaben: 1
    Damen Kat A: 3
    Damen Kat B: 2
    _TOTAL: 6
  }
  ...
}
\end{lstlisting}

\subsection{Übersicht der Schnittstellen}
Für den Nutzer wurde ein Problem agnostischer Namen für die Schnittstelle gewählt. Die Namen unterscheiden sich somit zum Teil zwischen Nutzer- und Algorithmus-Sicht voneinander. Damit die 
Verknüpfung der beiden Schnittstellen nicht verloren geht und eine Übersicht über alle angebotenen Schnittstellen zu haben, wurden sie in der \autoref{table:overview_api_interfaces} 
zusammengetragen. Die Schnittstellen für die Algorithmen sind unter einem anderen Namespace '/algorithm', damit diese sauber voneinander getrennt sind.

\begin{table}[ht]
\centering
  \begin{tabular}{ l | l }
	\hline
	\rowcolor{gray}
	\textbf{Nutzer}							& \textbf{Algorithmus}					\\ \hline
	/									& /algorithm/							\\ \hline
	\multicolumn{2}{|c|}{\textbf{Allgemein}}\\ \hline
	 Berechnung starten				& Berechnungsinformationen abholen			\\ \hline
	Status bzw. Lösung	abholen & Lösung bekannt geben	\\ \hline
	\multicolumn{2}{|c|}{\textbf{Rucksack}}\\ \hline
	 POST /optimalPackComputations					& GET /knapsackComputations/\{ID\}			\\ \hline
	GET /optimalPackComputations/\{ID\}	& POST /knapsackComputations/\{ID\}/solutions	\\ \hline
	\multicolumn{2}{|c|}{\textbf{Knotenfärbung}}\\ \hline
	POST /avoidCollisionComputations				& GET /graphColoringComputations/\{ID\}		\\ \hline
	GET /avoidCollisionComputations/\{ID\}		& POST /graphColoringComputations/\{ID\}/solutions	\\ \hline
	\multicolumn{2}{|c|}{\textbf{Problem des Handlungsreisenden}}\\ \hline
	POST /shortestRouteComputations				& GET /tspComputations/\{ID\}				\\ \hline
	GET /shortestRouteComputations/\{ID\}		& POST /tspComputations/\{ID\}/solutions		\\ \hline
	\multicolumn{2}{|c|}{\textbf{Briefträgerproblem}}\\ \hline
	POST /coverAllConnectionComputations				& GET /postmanComputations/\{ID\}			\\ \hline
	GET /coverAllConnectionComputations/\{ID\} 	& POST /postmanComputations/\{ID\}/solutions	\\ \hline
	\multicolumn{2}{|c|}{\textbf{Stundenplan Erstellung}}\\ \hline
	POST /timetableComputations				& GET /timetableComputations/\{ID\}			\\ \hline
	GET /timetableComputations/\{ID\}	& POST /timetableComputations/\{ID\}/solutions	\\ \hline
	\multicolumn{2}{|c|}{\textbf{Spielplan Erstellung}}\\ \hline
	POST /matchesScheduleComputations				& GET /matchesScheduleComputations/\{ID\}			\\ \hline
	GET /matchesScheduleComputations/\{ID\}	& POST/matchesScheduleComputations/\{ID\}/solutions	\\ \hline
  \end{tabular}
   \caption{Übersicht der angebotenen Schnittstellen}\label{table:overview_api_interfaces}
\end{table}

\FloatBarrier

\subsection{Erstellung eines neuen Problems}
Das Ziel dieser Arbeit war die Erstellung einer Schnittstelle, welche einfach zu erweitern ist. Dieser Abschnitt soll zeigen, wie dies erreicht wurde und wie sie erweitert werden kann. Das 
Software Projekt ist in Packages gegliedert. Auf der ersten Ebenen befinden sich die generischen Klassen und die spezifischen Implementierung sind unter dem Package 'problem'  angesiedelt. 
Die Gliederung ist so gewählt, damit ein Problem nur an einem Ort im Projekt verhängt ist und nicht an vielen verschiedenen Stellen gesucht werden muss. Es wäre möglich, 
die verschiedenen Problem-Implementationen in ein anderes Projekt auszulagern.

Um ein neues Problem in den Katalog aufzunehmen, 
müsste ein neues Package mit dem Problemnamen erstellt werden. Darin sind weitere Packages zur besseren Übersicht definiert. Als erstes müsste das Problem analysiert werden und 
dementsprechend die Entities für die Eingabe und das Resultat bereit gestellen werden.

Sind die Entities definiert, muss ein Controller für den Algorithmus und einer für den Nutzer erstellt werden. Die Controller leiten von einer abstrakten Klasse ab und dienen lediglich zur Definition
der neuen Endpunkte. Die Dokumentation der Controller wird auch im den jeweiligen Klassen geschrieben, dies erweitert automatisch die elektronische Dokumentation von Swagger UI. 
Nun fehlen noch die beiden Translator für das Umwandeln der Nutzer-Sicht in die Algorithmus-Sicht, der Validator für das Problem und der Solver, welcher den Algorithmus 
startet. Alles andere ist generischer Code, welcher nur noch mit den jeweiligen Entity-Typen spezifiziert werden muss.

Zu guter Letzt wird noch ein Beispiel für eine Eingabe aus Nutzer-Sicht und ein Resultat aus Algorithmus-Sicht in JSON unter den Ressourcen abgelegt. Diese Beispiele können für Tests 
verwendet werden.

\newpage

\section{Entwicklungsumgebung}\label{entwicklungsumgebung}
Ein Softwareprojekt benötigt immer eine gewisse Entwicklungsumgebung. Bei der Entwicklung mit Spring-Boot sind die Anfoderungen minimal, da Spring-Boot bereits einen eigenen Webserver 
mitbringt.

\subsection{IDE - Integrated Development Environment}
Als \glossarmark{IDE} wurde IntelliJ verwendet, IntelliJ bietet gute Refactoring-Methoden an und ist eine grosse Unterstützung beim Programmieren von Java-Code. Die \glossarmark{IDE} 
bietet auch die Möglichkeit Klassen-Diagramme zu erstellt und hat ein \glossarmark{VIM}-Plugin. Mit diesem Plugin können \glossarmark{VIM}-Befehle benutzt werden, was die 
Geschwindigkeit beim Programmieren enorm erhöht.

\todo{Ausschnitt aus Intellij, welcher gleich die Struktur des Projekts zeigt}

\subsection{Versionierung}
Für die Versionierung der Software wurde git \cite{git} verwendet. Das Remote Repository wurde auf Github \cite{github_simplatyzer} erstellt. Es wurde darauf geachtet, dass 
der Code oft ins Repository geladen wurde, damit ein Backup existiert. Die Dokumentation wurde in Dropbox gespeichert, damit auf verschiedenen Computern darauf zu gegriffen werden konnte 
und immer ein Backup vorhanden war. Zu Korrekturzwecken wurde die Arbeit ebenfalls auf github hochgeladen und die Änderungsvorschläge wurden mit Latexdiff verglichen.

\begin{figure}[h]
\centering
\includegraphics[scale=0.5]{images/github.png}
\caption[Github Repository des Simplatyser Projekts]{Github Repository des Simplatyser Projekts \selfmade{}}
\label{fig:github_repo}
\end{figure}

\subsection{Testen - Analysieren}
 Über den Chrome App 'Advanced REST client' (siehe Abbildung \ref{fig:advanced_rest_client})  wurde die Schnittstelle manuell getestet. Für die Regression-Tests wurde Junit und Mockito 
verwendet. Die statische Code Analyse wurde mit \glossarmark{Sonar} \cite{sonar} durchgeführt.

\begin{figure}[h]
\centering
\includegraphics[scale=0.7]{images/advanced_rest_client.png}
\caption[Advanced Rest client]{Advanced Rest client \selfmade{}}
\label{fig:advanced_rest_client}
\end{figure}
%%%%%%%%%%%%%%%%%%%%%%%%%%%%%%%%%%%%%%%%%%%%%%%%%%%%%%%%%%%%%%%%%
%
% Project     : Bachelorarbeit
% Title       : Machbarkeitsanalyse für eine ressourcenorientierte Schnittstelle zur Verarbeitung grundlegender Probleme der Informatik
% File        : tests.tex Rev. 01
% Date        : 01.03.2015
% Author      : Raffael Santschi
%
%%%%%%%%%%%%%%%%%%%%%%%%%%%%%%%%%%%%%%%%%%%%%%%%%%%%%%%%%%%%%%%%%

\chapter{Tests}\label{chap.tests} 
In diesem Kapitel wird auf die verschiedenen Tests und Varianten, welche in diesem Projekt verwendet wurden, eingegangen.

\section{Einführung}

\section{Testing}

\subsection{Testprotokoll}

%%%%%%%%%%%%%%%%%%%%%%%%%%%%%%%%%%%%%%%%%%%%%%%%%%%%%%%%%%%%%%%%%
%
% Project     : Bachelorarbeit
% Title       : Machbarkeitsanalyse für eine ressourcenorientierte Schnittstelle zur Verarbeitung grundlegender Probleme der Informatik
% File        : fazit.tex Rev. 01
% Date        : 01.03.2015
% Author      : Raffael Santschi
%
%%%%%%%%%%%%%%%%%%%%%%%%%%%%%%%%%%%%%%%%%%%%%%%%%%%%%%%%%%%%%%%%%


\chapter{Schlussfolgerungen}\label{chap.Schlussfolgerungen}
In der Schlussfolgerung wird kurz auf das Fazit des Verfassers und den Ausblick eingegangen.

\section{Fazit}\label{fazit}

\section{Ausblick}\label{fazit_ausblick}

\pagenumbering{Roman}

\appendix
%%%%%%%%%%%%%%%%%%%%%%%%%%%%%%%%%%%%%%%%%%%%%%%%%%%%%%%%%%%%%%%%%
%
% Project     : Bachelorarbeit
% Title       : Machbarkeitsanalyse für eine ressourcenorientierte Schnittstelle zur Verarbeitung grundlegender Probleme der Informatik
% File        : abstract.tex Rev. 01
% Date        : 01.03.2015
% Author      : Raffael Santschi
%
%%%%%%%%%%%%%%%%%%%%%%%%%%%%%%%%%%%%%%%%%%%%%%%%%%%%%%%%%%%%%%%%%

\chapter{Anhang}\label{chap.anhang}

\section{Risikoanalyse des Projekts}\label{risikoanalyse}
Jedes Projekt birgt Risiken. Werden diese nicht am Anfang analysiert und über die gesamte Projektlaufzeit überwacht, kann es zu grossen Schwierigkeiten kommen. Die angewandten Methoden sind aus \cite{proj_mgmt_book} und aus dem dem Management Fach Projekt und Prozessmanagement bekannt.

\subsection{Risikoermittlung}\label{risikoermittlung}
Die Risikoermittlung wird zur Bestimmung und Folgenabschätzung möglicher Risiken angewendet.

\begin{table}[ht]
\centering
  \begin{tabular}{  p{5cm} | p{9cm} }
	\hline
	\rowcolor{gray}
	\textbf{Risiko}					&	\textbf{Mögliche Folgen}	\\ \hline
	Implementationsschwierigkeiten			&	\begin{itemize}
										\item Zeitplan nicht einhaltbar
										\item Einige Anforderungen müssen zurückgestellt werden
										\item Projektarbeit kann nicht durchgeführt werden
									\end{itemize}	\\ \hline
	Zeitengpässe
								&	\begin{itemize}
										\item Zeitplan nicht einhaltbar
										\item Einige Anforderungen müssen zurückgestellt werden
										\item Projektarbeit kann nicht durchgeführt werden
									\end{itemize}	\\ \hline
	Mangelhaftes Endprodukt		
								&	\begin{itemize}
										\item Produkt muss überarbeitet werden, da keine Abnahme durch den Stakeholder erfolgt
									\end{itemize}	\\ \hline	
	Anforderungen nicht vollständig	
								&	\begin{itemize}
										\item Wichtige Funktionen stehen den Benutzern nicht zur Verfügung
									\end{itemize}	\\ \hline			
  \end{tabular}
   \caption{Risikoermittlung}
\end{table}

\subsection{Risikobewertung}
Das Schadensausmass und die Eintrittswahrscheinlichkeit der Risiken sind nach folgendem Schema bewertet worden:

\begin{table}[ht]
\centering
  \begin{tabular}{ l | p{5cm} | p{5cm} }
	\hline
	\rowcolor{gray}
	\textbf{Wert}					&	\textbf{Eintrittswahrscheindlichkeit} &	\textbf{Schadensausmass}	\\ \hline			
	1							&	sehr unwahrscheinlich		&	vernachlässigbar	\\ \hline
	2							&	unwahrscheinlich			&	spürbar		\\ \hline
	3							&	wenig wahrscheinlich		&	verkraftbar		\\ \hline
	4							&	ziemlich wahrscheinlich		&	gefährlich		\\ \hline
	5							&	sehr wahrscheinlich			&	katastrophal		\\ \hline
  \end{tabular}
   \caption{Risikobewertungsschema}
\end{table}

\FloatBarrier
Die Risiken aus der Risikobewertung (siehe \ref{risikoermittlung}) wurden anhand dieses Schemas bewertet.

\begin{equation*}
Risikofaktor = Eintrittswahrscheindlichkeit * Schadensausmass
\end{equation*}

\begin{table}[ht]
\centering
  \begin{tabular}{ l | p{4cm} | p{3cm} | c }
	\hline
	\rowcolor{gray}
	\textbf{Risiko}					&	\textbf{Eintrittswahrscheindlichkeit} &	\textbf{Schadensausmass} 	&	\textbf{Risikofaktor}\\ \hline			
	Implementationsschwierigkeiten			&	3					&	4			&	\textbf{12}	\\ \hline
	Zeitengpässe					&	2					&	5			&	\textbf{10}	\\ \hline
	Mangelhaftes Endprodukt				&	2					&	4			&	\textbf{8}	\\ \hline
	Anforderungen nicht vollständig			&	2					&	2			&	4		\\ \hline
  \end{tabular}
   \caption{Risikobewertung}
\end{table}

\subsection{Risikomatrix}
Anhand der Risikobeurteilung konnten die Risiken in eine Risikomatrix eingesetzt werden. Diese Matrix bietet einen guten Überblick über die Risiken und zeigt schnell, welche Risiken beachtet werden müssen.
\begin{figure}[h]
\centering
\includegraphics[scale=0.5]{images/excel/risikomatrix.png}
\caption{Risikomatrix}
\label{fig:risikomatrix}
\end{figure}

\subsection{Massnahmen}
Es wurden Massnahmen für die gefundenen Risiken gesucht und festgehalten. Die Massnahmen sind wiederum in vorbeugende Massnahmen und Eventualmassnahmen unterteilt.

\begin{table}[ht]
\centering
  \begin{tabular}{  l | p{4.5cm} | p{4.5cm} }
	\hline
	\rowcolor{darkgray}
	\textbf{Risiko}					&	\multicolumn{2}{|c|}{\textbf{Massnahmen}} \\ \hline
	\rowcolor{gray}
								&	Vorbeugende Massnahmen & Eventualmassnahmen	\\ \hline
	Implementationsschwierigkeiten
								&	\begin{itemize}
										\item Im Zeitplan genügen Reserver einrechnen
										\item Kontaktpersonen zum Thema suchen
										\item Zeitplan einhalten, pünktlich mit den Arbeiten beginnen
									\end{itemize}
								&	\begin{itemize}
										\item Betreuer / Schulleitung informieren
										\item Verschiebungsgesuch stellen
									\end{itemize}						\\ \hline
	Zeitengpässe
								&	\begin{itemize}
										\item Fixe Zeiten einplanen
										\item Tätigkeiten priorisieren
									\end{itemize}
								&	\begin{itemize}
										\item Vorgezogene Präsentation verschieben
										\item Verschiebungsgesuch stellen
									\end{itemize}	\\ \hline
	Mangelhaftes Endprodukt		
								&	\begin{itemize}
										\item Anforderungskatalog sauber erstellen
										\item Kunden laufend über den Stand des Produktes informieren
									\end{itemize}
								&	\begin{itemize}
										\item Lösung mit dem Kunden suchen
										\item Projekt verlängern
									\end{itemize}	\\ \hline	
	Anforderungen nicht vollständig	
								&	\begin{itemize}
										\item Review des Anforderungskataloges
										\item Kunden laufend über den Stand des Produktes informieren
									\end{itemize}
								&	\begin{itemize}
										\item Anforderungen werden aufgenommen und in einen nächsten Release geplant							
									\end{itemize}	\\ \hline			
  \end{tabular}
   \caption{Risikoanalyse - Massnahmen}
\end{table}

\clearpage
\newpage

\section{Schnittstellen Dokumentation}\label{api_doc}

Die Beispiele sind im \glossarmark{JSON} Format, es ist jedoch auch sehr einfach möglich die Schnittstelle für \glossarmark{XML}, \glossarmark{YAML} oder andere Formate zu erweitern

%%%%%%%%%%%%%%%%%%%%%%%%%%%%%%%
%
%
%		Rucksack
%
%
%%%%%%%%%%%%%%%%%%%%%%%%%%%%%%%

\subsection{Rucksack}

\paragraph{POST /optimalPackComputations} - Erstellung einer neuen Rucksack Berechnung\mbox{}\\
Erstellt eine neue Berechnung des Rucksack Problems und speichert sie in die Datenbank.\\
\textbf{Status Codes:}\\
200 - OK: Berechnung wurde erstellt, zusätzlich erhält der Nutzer noch die ID der Berechnung.\\
400 - Bad Request: Die Validierung der Eingabedaten ist fehlgeschlagen.\\

\begin{lstlisting}[language=JSON, caption=Beispiel einer Eingabe für das Rucksack Problem, label=lst:input_knapsack]  
{
  "name": "<Berechnungsname>",
  "maxWeight": <Gewichtsschranke>,
  "items": [
    {
      "id": "<Objekt ID (optional)>",
      "name": "<Objekt Name>",
      "weight": <Gewicht>,
      "profit": <Profit>,
      "count": <Anzahl (optional)>
    }
  ]
}
\end{lstlisting}

\paragraph{GET /algorithm/knapsackComputations/\{ID\}} - Abrufen der Eingabedaten einer Rucksack Berechnung\mbox{}\\
Ruft die Eingabedaten einer bestimmten Berechnung des Rucksack Problems ab. Die Daten werden in der Schnittstelle zuerst für den Algorithmus umgewandelt.\\
\textbf{Status Codes:}\\
200 - OK: Die Eingabedaten wurden gefunden und zurückgeben.\\
404 - Not Found: Die Eingabedaten wurden nicht gefunden.\\

\begin{lstlisting}[language=JSON, caption=Beispiel für Eingabedaten des Rucksack Problems für den Algorithmus, label=lst:input_knapsack_algo]  
{
  "id": "<Berechnung ID>",
  "maxWeight": <Gewichtsschranke>,
  "items": [
    {
      "id": "<Objekt ID>",
      "weight": <Gewicht>,
      "profit": <Profit>
    }
  ]
}
\end{lstlisting}

\paragraph{POST /algorithm/knapsackComputations/\{ID\}/solutions} - Speichern einer Lösung für eine Rucksack Berechnung\mbox{}\\
Speichert eine mögliche Lösung für eine Berechnung des Rucksack Problems. Die Lösung wird zuerst für den Nutzer umgewandelt und danach noch validiert.\\
\textbf{Status Codes:}\\
200 - OK: Lösung wurde gespeichert.\\
400 - Bad Request: Die Validierung der Lösung zeigt Fehler auf, zusätzlich werden die Validierungsfehler mitgegeben.\\

\begin{lstlisting}[language=JSON, caption=Beispiel eines Resultates für das Rucksack Problem aus Algorithmus-Sicht, label=lst:solution_knapsack_algo]  
{
  "solutionType": "<Resultat Typ (PARTIAL | FINAL)>",
  "fitness": <Profit des Resultates>,
  "items": [<Liste der Boolean-Werte>]
}
\end{lstlisting}

\paragraph{GET /optimalPackComputations/\{ID\}} - Abrufen des Status einer Rucksack Berechnung\mbox{}\\
Ruft den Status und die Endresultate einer bestimmten Berechnung des Rucksack Problems ab, die Daten werden direkt aus der Datenbank genommen, da sie dort in Nutzer-Sicht abgespeichert sind.\\
\textbf{Status Codes:}\\
200 - OK: Das Endresultat wurden gefunden und zurückgeben.\\
404 - Not Found: Keine Berechnung gefunden.\\

\begin{lstlisting}[language=JSON, caption=Beispiel eines Endresultates für das Rucksack Problem, label=lst:solution_knapsack]  
{
  "computationId": "<Berechnung ID>",
  "profit": <Profit des Resultates>,
  "items": [
    {
      "id": "<Objekt ID>",
      "name": "<Objekt Name>",
      "weight": <Gewicht>,
      "profit": <Profit>,
      "count": <Anzahl>
    }
  ]
}
\end{lstlisting}

%%%%%%%%%%%%%%%%%%%%%%%%%%%%%%%
%
%
%		Knotenfärbung
%
%
%%%%%%%%%%%%%%%%%%%%%%%%%%%%%%%

\subsection{Knotenfärbung}

\paragraph{POST /avoidCollisionComputations} - Erstellung einer neue Knotenfärbungsberechnung\mbox{}\\
Erstellt eine neue Berechnung des Knotenfärbungsproblems und speichert sie in die Datenbank.\\
\textbf{Status Codes:}\\
200 - OK: Berechnung wurde erstellt, zusätzlich erhält der Nutzer noch die ID der Berechnung.\\
400 - Bad Request: Die Validierung der Eingabedaten ist fehlgeschlagen.\\

\begin{lstlisting}[language=JSON, caption=Beispiel einer Eingabe für das Knotenfärbungsproblem, label=lst:input_graphcoloring]  
{
  "name": "<Berechnungsname>",
  "items": [
    {
      "id": "<Element ID (optional)>",
      "name": "<Element Name>",
      "neighbors": [<Benachbarte Elemente>]
    }
  ],
  "possibleValues": [<Moegliche Werte>]
}
\end{lstlisting}

\paragraph{GET /algorithm/graphColoringComputations/\{ID\}} - Abrufen der Eingabedaten einer Knotenfärbungsberechnung\mbox{}\\
Ruft die Eingabedaten einer bestimmten Berechnung des Knotenfärbungsproblems ab. Die Daten werden in der Schnittstelle zuerst für den Algorithmus umgewandelt.\\
\textbf{Status Codes:}\\
200 - OK: Die Eingabedaten wurden gefunden und zurückgeben.\\
404 - Not Found: Die Eingabedaten wurden nicht gefunden.\\

\begin{lstlisting}[language=JSON, caption=Beispiel für Eingabedaten des Knotenfärbungsproblems für den Algorithmus, label=lst:input_graphcoloring_algo]  
{
  "id": "<Berechnung ID>",
  "items": [
    {
      "id": "<Element ID>",
      "neighbors": [<Benachbarte Elemente>]
    }
  ]
}
\end{lstlisting}

\paragraph{POST /algorithm/graphColoringComputations/\{ID\}/solutions} - Speichern einer Lösung für eine Knotenfärbungsberechnung\mbox{}\\
Speichert eine mögliche Lösung für eine Berechnung des Knotenfärbungsproblems. Die Lösung wird zuerst für den Nutzer umgewandelt und danach noch validiert.\\
\textbf{Status Codes:}\\
200 - OK: Lösung wurde gespeichert.\\
400 - Bad Request: Die Validierung der Lösung zeigt Fehler auf, zusätzlich werden die Validierungsfehler mitgegeben.\\

\begin{lstlisting}[language=JSON, caption=Beispiel eines Resultates für das Knotenfärbungsproblems aus Algorithmus-Sicht, label=lst:solution_graphcoloring_algo]  
{
  "solutionType": "<Resultat Typ (PARTIAL | FINAL)>",
  "differentValues": <Anzahl verschiedene Werte>,
  "items": [
    {
      "id": "<Element name>",
      "value": "<Zugewiesener Wert>"
    }
  ]
}
\end{lstlisting}

\paragraph{GET /optimalPackComputations/\{ID\}} - Abrufen des Status einer Rucksack Berechnung\mbox{}\\
Ruft den Status und die Endresultate einer bestimmten Berechnung des Knotenfärbungsproblems ab, die Daten werden direkt aus der Datenbank genommen, da sie dort in Nutzer-Sicht abgespeichert sind.\\
\textbf{Status Codes:}\\
200 - OK: Das Endresultat wurden gefunden und zurückgeben.\\
404 - Not Found: Keine Berechnung gefunden.\\

\begin{lstlisting}[language=JSON, caption=Beispiel eines Endresultates für das Knotenfärbungsproblems, label=lst:solution_graphcoloring]  
{
  "computationId": "<Berechnung ID>",
  "differentValues": <Anzahl verschiedene Werte>,
  "items": [
    {
      "id": "<Element name>",
      "value": "<Zugewiesener Wert (ersetzt durch angegebene Werte)>"
    }
  ]
}
\end{lstlisting}

%%%%%%%%%%%%%%%%%%%%%%%%%%%%%%%
%
%
%		Problem des Handlungsreisenden
%
%
%%%%%%%%%%%%%%%%%%%%%%%%%%%%%%%

\subsection{Problem des Handlungsreisenden}

\paragraph{POST /shortestRouteComputations} - Erstellung einer neue Routenberechnung\mbox{}\\
Erstellt eine neue Berechnung des Problem des Handlungsreisenden und speichert sie in die Datenbank.\\
\textbf{Status Codes:}\\
200 - OK: Berechnung wurde erstellt, zusätzlich erhält der Nutzer noch die ID der Berechnung.\\
400 - Bad Request: Die Validierung der Eingabedaten ist fehlgeschlagen.\\

\begin{lstlisting}[language=JSON, caption=Beispiel einer Eingabe für das Problem des Handlungsreisenden, label=lst:input_tsp]  
{
  "name": "<Berechnungsname>",
  "waypoints": [
    {
      "id": "<Wegpunkt ID (optional)>",
      "name": "<Wegpunkt Name>",
      "duration": <Aufenthaltszeit>,
      "wishedArrival": <Gewuenschte Ankunftszeit>
    }
  ],
  "configuration": {
    "startPoint": {
      "id": "<Wegpunkt ID (optional)>",
      "name": "<Wegpunkt Name>"
    },
    "startTime": <Startzeit>,
    "maxArrivalVariance": <Maximale Abweichung bei den Akunftszeiten>
  }
}
\end{lstlisting}

\paragraph{GET /algorithm/tspComputations/\{ID\}} - Abrufen der Eingabedaten einer Routenberechnung\mbox{}\\
Ruft die Eingabedaten einer bestimmten Berechnung des Problem des Handlungsreisenden ab. Die Daten werden in der Schnittstelle zuerst für den Algorithmus umgewandelt.\\
\textbf{Status Codes:}\\
200 - OK: Die Eingabedaten wurden gefunden und zurückgeben.\\
404 - Not Found: Die Eingabedaten wurden nicht gefunden.\\

\begin{lstlisting}[language=JSON, caption=Beispiel für Eingabedaten des Problem des Handlungsreisenden für den Algorithmus, label=lst:input_tsp_algo]  
{
  "id": "<Berechnung ID>",
  "waypoints": [
    {
      "id": "<Wegpunkt ID>",
      "name": "<Wegpunkt Name>",
      "duration": <Aufenthaltszeit>,
      "wishedArrival": <Gewuenschte Ankunftszeit>
    }
  ],
  "configuration": {
    "startPoint": {
      "id": "<Wegpunkt ID>",
      "name": "<Wegpunkt Name>"
    },
    "startTime": <Startzeit>,
    "maxArrivalVariance": <Maximale Abweichung bei den Akunftszeiten>
  }
}
\end{lstlisting}

\paragraph{POST /algorithm/tspComputations/\{ID\}/solutions} - Speichern einer Lösung für eine Routenberechnung\mbox{}\\
Speichert eine mögliche Lösung für eine Berechnung des Problem des Handlungsreisenden. Die Lösung wird zuerst für den Nutzer umgewandelt und danach noch validiert.\\
\textbf{Status Codes:}\\
200 - OK: Lösung wurde gespeichert.\\
400 - Bad Request: Die Validierung der Lösung zeigt Fehler auf, zusätzlich werden die Validierungsfehler mitgegeben.\\

\begin{lstlisting}[language=JSON, caption=Beispiel eines Resultates für das Problem des Handlungsreisenden aus Algorithmus-Sicht, label=lst:solution_tsp_algo]  
{
  "solutionType": "<Resultat Typ (PARTIAL | FINAL)>",
  "totalLength": <Totale Laenge>,
  "waypoints": [
    {
      "name": "<Wegpunkt Name>"
    }
  ]
}
\end{lstlisting}

\paragraph{GET /shortestRouteComputations/\{ID\}} - Abrufen des Status einer Routenberechnung\mbox{}\\
Ruft den Status und die Endresultate einer bestimmten Berechnung des Problem des Handlungsreisenden ab, die Daten werden direkt aus der Datenbank genommen, da sie dort in Nutzer-Sicht abgespeichert sind.\\
\textbf{Status Codes:}\\
200 - OK: Das Endresultat wurden gefunden und zurückgeben.\\
404 - Not Found: Keine Berechnung gefunden.\\

\begin{lstlisting}[language=JSON, caption=Beispiel eines Endresultates für das Problem des Handlungsreisenden, label=lst:solution_tsp]  
{
  "computationId": "<Berechnung ID>",
  "totalLength": <Totale Laenge>,
  "plannedArrival": <Geplante Endankunftszeit>,
  "waypoints": [
    {
      "id": "<Wegpunkt ID>",
      "name": "<Wegpunkt Name>",
      "duration": <Aufenthaltszeit>,
      "wishedArrival": <Gewuenschte Ankunftszeit>,
      "plannedArrival": <Geplante Ankunftszeit>
    }
  ]
}
\end{lstlisting}

%%%%%%%%%%%%%%%%%%%%%%%%%%%%%%%
%
%
%		Briefträgerproblem
%
%
%%%%%%%%%%%%%%%%%%%%%%%%%%%%%%%

\subsection{Briefträgerproblem}

\paragraph{POST /coverAllConnectionComputations} - Erstellung einer neue Routenberechnung\mbox{}\\
Erstellt eine neue Berechnung des Briefträgerproblem und speichert sie in die Datenbank.\\
\textbf{Status Codes:}\\
200 - OK: Berechnung wurde erstellt, zusätzlich erhält der Nutzer noch die ID der Berechnung.\\
400 - Bad Request: Die Validierung der Eingabedaten ist fehlgeschlagen.\\

\begin{lstlisting}[language=JSON, caption=Beispiel einer Eingabe für das Briefträgerproblem, label=lst:input_postman]  
{
  "name": "<Berechnungsname>",
  "items": [
    {
      "id": "<Wegpunkt ID (optional)>",
      "name": "<Wegpunkt Name>",
      "connections": [
        {
          "item": {<Wegpunkt>},
          "length": <Distanz>
        }
      ]
    }
  ],
  "startPoint": {
    "id": "<Wegpunkt ID (optional)>",
    "name": "<Wegpunkt Name>",
    "connections": [
      {
        "item": {<Wegpunkt>},
        "length": <Distanz>
      }
    ]
  }
}
\end{lstlisting}

\paragraph{GET /algorithm/postmanComputations/\{ID\}} - Abrufen der Eingabedaten einer Routenberechnung\mbox{}\\
Ruft die Eingabedaten einer bestimmten Berechnung des Briefträgerproblem ab. Die Daten werden in der Schnittstelle zuerst für den Algorithmus umgewandelt.\\
\textbf{Status Codes:}\\
200 - OK: Die Eingabedaten wurden gefunden und zurückgeben.\\
404 - Not Found: Die Eingabedaten wurden nicht gefunden.\\

\begin{lstlisting}[language=JSON, caption=Beispiel für Eingabedaten des Briefträgerproblems für den Algorithmus, label=lst:input_postman_algo]  
{
  "id": "<Berechnung ID>",
  "items": [
    {
      "id": "<Wegpunkt ID>",
      "name": "<Wegpunkt Name>",
      "connections": [
        {
          "item": {<Wegpunkt>},
          "length": <Distanz>
        }
      ]
    }
  ],
  "startPoint": {
    "id": "<Wegpunkt ID (optional)>",
    "name": "<Wegpunkt Name>",
    "connections": [
      {
        "item": {<Wegpunkt>},
        "length": <Distanz>
      }
    ]
  }
}
\end{lstlisting}

\paragraph{POST /algorithm/postmanComputations/\{ID\}/solutions} - Speichern einer Lösung für eine Routenberechnung\mbox{}\\
Speichert eine mögliche Lösung für eine Berechnung des Briefträgerproblems. Die Lösung wird zuerst für den Nutzer umgewandelt und danach noch validiert.\\
\textbf{Status Codes:}\\
200 - OK: Lösung wurde gespeichert.\\
400 - Bad Request: Die Validierung der Lösung zeigt Fehler auf, zusätzlich werden die Validierungsfehler mitgegeben.\\

\begin{lstlisting}[language=JSON, caption=Beispiel eines Resultates für das Briefträgerproblem aus Algorithmus-Sicht, label=lst:solution_postman_algo]  
{
  "solutionType": "<Resultat Typ (PARTIAL | FINAL)>",
  "items": [
    {
      "id": "<Wegpunkt ID>"
    }
  ]
}
\end{lstlisting}

\paragraph{GET /coverAllConnectionComputations/\{ID\}} - Abrufen des Status einer Routenberechnung\mbox{}\\
Ruft den Status und die Endresultate einer bestimmten Berechnung des Briefträgerproblems ab, die Daten werden direkt aus der Datenbank genommen, da sie dort in Nutzer-Sicht abgespeichert sind.\\
\textbf{Status Codes:}\\
200 - OK: Das Endresultat wurden gefunden und zurückgeben.\\
404 - Not Found: Keine Berechnung gefunden.\\

\begin{lstlisting}[language=JSON, caption=Beispiel eines Endresultates für das Briefträgerproblem, label=lst:solution_postman]  
{
  "computationId": "<Berechnung ID>",
  "items": [
    {
      "id": "<Wegpunkt ID>",
      "name": "<Wegpunkt Name>"
    }
  ],
  "totalLength": <Totale Laenge>
}
\end{lstlisting}

%%%%%%%%%%%%%%%%%%%%%%%%%%%%%%%
%
%
%		Stundenplan Erstellung
%
%
%%%%%%%%%%%%%%%%%%%%%%%%%%%%%%%

\subsection{Stundenplan Erstellung}

\paragraph{POST /timetableComputations} - Erstellung einer neue Stundenplan Berechnung\mbox{}\\
Erstellt eine neue Berechnung des Stundenplanproblems und speichert sie in die Datenbank.\\
\textbf{Status Codes:}\\
200 - OK: Berechnung wurde erstellt, zusätzlich erhält der Nutzer noch die ID der Berechnung.\\
400 - Bad Request: Die Validierung der Eingabedaten ist fehlgeschlagen.\\

\begin{lstlisting}[language=JSON, caption=Beispiel einer Eingabe für das Stundenplanproblem, label=lst:input_timetableScheduling]  
{
  "name": "<Berechnungsname>",
  "schoolClasses": [
    {
      "id": "<ID Klasse>",
      "name": "<Klassename>",
      "schoolSubjects": [
        {
          "id": "<ID Schulfach>"
        }
      ],
      "size": <Groesse der Klasse>
    }
  ],
  "teachers": [
    {
      "id": "<ID Lehrer>",
      "name": "<Name des Lehrers>",
      "skills": [
        {
          "id": "<ID Schulfach>"
        }
      ],
      "associations": [
        {
          "id": "<ID Klasse (optional)>"
        }
      ],
      "freeDays": [
        {
          "<Wochentag>": {
            "from": <Startzeit (optional)>,
            "to": <Endzeit (optional)>,
            "morning": <Am Morgen (optional)>,
            "afternoon": <Am Nachmittag (optional)>,
            "wholeDay": <Ganzer Tag (optional)>
          }
        }
      ]
    }
  ],
  "classRooms": [
    {
      "id": "<ID Klassenzimmer>",
      "name": "<Klassenzimmername>",
      "allowedSkills": [
        {
          "id": "<ID Schulfach (optional)>"
        }
      ],
      "blockedDays": [
        {
          "<Wochentag>": {
            "from": <Startzeit (optional)>,
            "to": <Endzeit (optional)>,
            "morning": <Am Morgen (optional)>,
            "afternoon": <Am Nachmittag (optional)>,
            "wholeDay": <Ganzer Tag (optional)>
          }
        }
      ],
      "capacity": <Fassungsvermoegen>
    }
  ],
  "schoolSubjects": [
    {
      "id": "<ID Schulfach>",
      "name": "<Schulfachname>",
      "explicitLocation": <Explizites Schulzimmer (optional)>
    }
  ],
  "configuration": {
    "breakTimeSliceSize": [<Pausenzeiten>],
    "dayTimeSlots": [
      {
        "<Wochentag>": {
          "from": <Startzeit (optional)>,
          "to": <Endzeit (optional)>,
          "defaultTimes": <Standard Werte (optional)>
        }
      }
    ],
    "lessonDuration": <Lektionsdauer>
  }
}
\end{lstlisting}

\paragraph{GET /algorithm/timetableComputations/\{ID\}} - Abrufen der Eingabedaten einer Stundenplan Berechnung\mbox{}\\
Ruft die Eingabedaten einer bestimmten Berechnung des Stundenplanproblems ab. Die Daten werden in der Schnittstelle zuerst für den Algorithmus umgewandelt.\\
\textbf{Status Codes:}\\
200 - OK: Die Eingabedaten wurden gefunden und zurückgeben.\\
404 - Not Found: Die Eingabedaten wurden nicht gefunden.\\

\begin{lstlisting}[language=JSON, caption=Beispiel für Eingabedaten des Stundenplanproblems für den Algorithmus, label=lst:input_timetableScheduling_algo]  
{
  "id": "<Berechnung ID>",
  "associations": [
    {
      "id": "<ID Klasse>",
      "name": "<Klassename>",
      "schoolSubjects": [
        {
          "id": "<ID Schulfach>"
        }
      ],
      "size": <Groesse der Klasse>
    }
  ],
  "responsibles": [
    {
      "id": "<ID Lehrer>",
      "name": "<Name des Lehrers>",
      "skills": [
        {
          "id": "<ID Schulfach>"
        }
      ],
      "associations": [
        {
          "id": "<ID Klasse>"
        }
      ],
      "freeDays": [
        {
          "<Wochentag>": {
            "from": <Startzeit>,
            "to": <Endzeit>,
            "morning": <Am Morgen>,
            "afternoon": <Am Nachmittag>,
            "wholeDay": <Ganzer Tag>
          }
        }
      ]
    }
  ],
  "places": [
    {
      "id": "<ID Klassenzimmer>",
      "name": "<Klassenzimmername>",
      "allowedSkills": [
        {
          "id": "<ID Schulfach>"
        }
      ],
      "blockedDays": [
        {
          "<Wochentag>": {
            "from": <Startzeit>,
            "to": <Endzeit>,
            "morning": <Am Morgen>,
            "afternoon": <Am Nachmittag>,
            "wholeDay": <Ganzer Tag>
          }
        }
      ],
      "capacity": <Fassungsvermoegen>
    }
  ],
  "skills": [
    {
      "id": "<ID Schulfach>",
      "name": "<Schulfachname>",
      "explicitLocation": <Explizites Schulzimmer>
    }
  ],
  "timeSlices": [
    {
      "number": <Zeitschlitz Nummer>
    }
  ]
}
\end{lstlisting}

\paragraph{POST /algorithm/timetableComputations/\{ID\}/solutions} - Speichern einer Lösung für eine Stundenplan Berechnung\mbox{}\\
Speichert eine mögliche Lösung für eine Berechnung des Stundenplanproblems. Die Lösung wird zuerst für den Nutzer umgewandelt und danach noch validiert.\\
\textbf{Status Codes:}\\
200 - OK: Lösung wurde gespeichert.\\
400 - Bad Request: Die Validierung der Lösung zeigt Fehler auf, zusätzlich werden die Validierungsfehler mitgegeben.\\

\begin{lstlisting}[language=JSON, caption=Beispiel eines Resultates für das Stundenplanproblem aus Algorithmus-Sicht, label=lst:solution_timetableScheduling_algo]  
{
  "solutionType": "<Resultat Typ (PARTIAL | FINAL)>",
  "timeSlices": [
    {
      "timeSlice": {
        "number": <Zeitschlitz Nummer>
      },
      "responsible": {
        "id": "<ID Lehrer>"
      },
      "association": {
        "id": "<ID Schulklasse>"
      },
      "skill": {
        "id": "<ID Schulfach>"
      },
      "place": {
        "id": "<ID Klassenzimmer>"
      }
    }
  ]
}
\end{lstlisting}

\paragraph{GET /timetableScheduleComputations/\{ID\}} - Abrufen des Status einer Stundenplan Berechnung\mbox{}\\
Ruft den Status und die Endresultate einer bestimmten Berechnung des Stundenplanproblems ab, die Daten werden direkt aus der Datenbank genommen, da sie dort in Nutzer-Sicht abgespeichert sind.\\
\textbf{Status Codes:}\\
200 - OK: Das Endresultat wurden gefunden und zurückgeben.\\
404 - Not Found: Keine Berechnung gefunden.\\

\begin{lstlisting}[language=JSON, caption=Beispiel eines Endresultates für das Stundenplanproblem, label=lst:solution_timetableScheduling]  
{
  "computationId": "<Berechnung ID>",
  "plan": [
    {
      "<Wochentag>": [
        {
          "timeSlice": {
            "from": <Startzeit>,
            "to": <Endzeit>
          },
          "teacher": {
            "id": "<ID Lehrer>",
            "name": "<Name des Lehrers>",
          },
          "schoolClass": {
            "id": "<ID Klasse>",
            "name": "<Klassenname>",
          },
          "schoolSubject": {
            "id": "<ID Schulfach>",
            "name": "<Schulfachname>",
          },
          "classRoom": {
            "id": "<ID Klassenzimmer>",
            "name": "<Klassenzimmername>",
          }
        }
      ]
    }
  ],
  "teacherStatistics": [
    {
      "name": "<Lehrername>",
      "statisticMap": [
        {
          "<Schulfach>": <Anzahl Stunden>
        }
      ]
    }
  ]
}
\end{lstlisting}

%%%%%%%%%%%%%%%%%%%%%%%%%%%%%%%
%
%
%		Spielplan Erstellung
%
%
%%%%%%%%%%%%%%%%%%%%%%%%%%%%%%%

\subsection{Spielplan Erstellung}

\paragraph{POST /matchScheduleComputations} - Erstellung einer neue Spielplan Berechnung\mbox{}\\
Erstellt eine neue Berechnung des Spielplanproblems und speichert sie in die Datenbank.\\
\textbf{Status Codes:}\\
200 - OK: Berechnung wurde erstellt, zusätzlich erhält der Nutzer noch die ID der Berechnung.\\
400 - Bad Request: Die Validierung der Eingabedaten ist fehlgeschlagen.\\

\begin{lstlisting}[language=JSON, caption=Beispiel einer Eingabe für das Spielplanproblem, label=lst:input_matchScheduling]  
{
  "name": "<Berechnungsname>",
  "teams": [
    {
      "id": "<ID Team>",
      "name": "<Teamname>",
      "category": {
          "id": "<ID Kategorie>"
      }
    }
  ],
  "referees": [
    {
      "id": "<ID Schiedsrichter>",
      "name": "<Name des Schiedsrichters>",
      "skills": [
        {
          "id": "<ID Kategorie>"
        }
      ],
      "associations": [
        {
          "id": "<ID Team (optional)>"
        }
      ]
    }
  ],
  "fields": [
    {
      "id": "<ID Spielfeld>",
      "name": "<Spielfeldname>",
      "allowedSkills": [
        {
          "id": "<ID Kategorie (optional)>"
        }
      ]
    }
  ],
  "categories": [
    {
      "id": "<ID Kategorie>",
      "name": "<Kategoriename>"
    }
  ],
  "configuration": {
    "breakTimeSliceSize": [<Pausenzeiten>],
    "timeSliceSize": <Spieldauer>
    "startPoint": <Startpunkt>
  }
}
\end{lstlisting}

\paragraph{GET /algorithm/matchScheduleComputations/\{ID\}} - Abrufen der Eingabedaten einer Spielplan Berechnung\mbox{}\\
Ruft die Eingabedaten einer bestimmten Berechnung des Spielplanproblems ab. Die Daten werden in der Schnittstelle zuerst für den Algorithmus umgewandelt.\\
\textbf{Status Codes:}\\
200 - OK: Die Eingabedaten wurden gefunden und zurückgeben.\\
404 - Not Found: Die Eingabedaten wurden nicht gefunden.\\

\begin{lstlisting}[language=JSON, caption=Beispiel für Eingabedaten des Spielplanproblems für den Algorithmus, label=lst:input_matchScheduling_algo]  
{
  "id": "<Berechnung ID>",
  "associations": [
    {
      "id": "<ID Team>",
      "name": "<Teamname>",
      "category": {
          "id": "<ID Kategorie>"
      }
    }
  ],
  "responsibles": [
    {
      "id": "<ID Schiedsrichter>",
      "name": "<Name des Schiedsrichters>",
      "skills": [
        {
          "id": "<ID Kategorie>"
        }
      ],
      "associations": [
        {
          "id": "<ID Team (optional)>"
        }
      ]
    }
  ],
  "places": [
    {
      "id": "<ID Spielfeld>",
      "name": "<Spielfeldname>",
      "allowedSkills": [
        {
          "id": "<ID Kategorie (optional)>"
        }
      ]
    }
  ],
  "skills": [
    {
      "id": "<ID Kategorie>",
      "name": "<Kategoriename>"
    }
  ]
}
\end{lstlisting}

\paragraph{POST /algorithm/matchScheduleComputations/\{ID\}/solutions} - Speichern einer Lösung für eine Spielplan Berechnung\mbox{}\\
Speichert eine mögliche Lösung für eine Berechnung des Spielplanproblems. Die Lösung wird zuerst für den Nutzer umgewandelt und danach noch validiert.\\
\textbf{Status Codes:}\\
200 - OK: Lösung wurde gespeichert.\\
400 - Bad Request: Die Validierung der Lösung zeigt Fehler auf, zusätzlich werden die Validierungsfehler mitgegeben.\\

\begin{lstlisting}[language=JSON, caption=Beispiel eines Resultates für das Spielplanproblem aus Algorithmus-Sicht, label=lst:solution_matchScheduling_algo]  
{
  "solutionType": "<Resultat Typ (PARTIAL | FINAL)>",
  "timeSlices": [
    {
      "timeSlice": {
        "number": <Zeitschlitz Nummer>
      },
      "responsible": {
        "id": "<ID Schiedsrichter>"
      },
      "association": {
        "team1": {
          "id": "<ID Team>"
        },
        "team2": {
          "id": "<ID Team>"
        }
      },
      "place": {
        "id": "<ID Spielfeld>"
      }
    }
  ]
}
\end{lstlisting}

\paragraph{GET /matchScheduleComputations/\{ID\}} - Abrufen des Status einer Spielplan Berechnung\mbox{}\\
Ruft den Status und die Endresultate einer bestimmten Berechnung des Spielplanproblems ab, die Daten werden direkt aus der Datenbank genommen, da sie dort in Nutzer-Sicht abgespeichert sind.\\
\textbf{Status Codes:}\\
200 - OK: Das Endresultat wurden gefunden und zurückgeben.\\
404 - Not Found: Keine Berechnung gefunden.\\

\begin{lstlisting}[language=JSON, caption=Beispiel eines Endresultates für das Spielplanproblem, label=lst:solution_matchScheduling]  
{
  "computationId": "<Berechnung ID>",
  "timeSlices": [
    {
      "timeSlice": {
        "from": <Startzeit>,
        "to": <Endzeit>
      },
      "referee": {
        "id": "<ID Schiedsrichter>",
        "name": "<Name des Schiedsrichters>",
      },
      "teams": {
        "team1": {
          "id": "<ID Team>",
          "name": "<Teamname>",
        },
        "team2": {
          "id": "<ID Team>",
          "name": "<Teamname>",
        }
      },
      "categorie": {
        "id": "<ID Kategorie>",
        "name": "<Kategoriename>",
      },
      "field": {
        "id": "<ID Klassenzimmer>",
        "name": "<Klassenzimmername>",
      }
    }
  ],
  "refereeStatistics": [
    {
      "name": "<Schiedsrichtername>",
      "statisticMap": [
        {
          "<Kategoriename>": <Anzahl Spiele>
        }
      ]
    }
  ],
  "teamStatistics": [
    {
      "name": "<Kategoriename>",
      "statisticMap": [
        {
          "<Teamname>": <Anzahl Spiele>
        }
      ]
    }
  ]
}
\end{lstlisting}

\clearpage
\printglossary[type=\acronymtype] % prints just the list of acronyms
 \addcontentsline{toc}{chapter}{Akronyme}
 
\printglossary
 \addcontentsline{toc}{chapter}{Glossar}

\clearpage
 \bibliography{bibtex/literatur}
 \addcontentsline{toc}{chapter}{Literaturverzeichnis}

\clearpage
 \listoffigures
 \addcontentsline{toc}{chapter}{Abbildungsverzeichnis}

\clearpage
 \lstlistoflistings
 \addcontentsline{toc}{chapter}{Listingsverzeichnis}

\clearpage
 \listoftables
 \addcontentsline{toc}{chapter}{Tabellenverzeichnis}


\end{document}
