%%%%%%%%%%%%%%%%%%%%%%%%%%%%%%%%%%%%%%%%%%%%%%%%%%%%%%%%%%%%%%%%%
%
% Project     : Bachelorarbeit
% Title       : Machbarkeitsanalyse für eine ressourcenorientierte Schnittstelle zur Verarbeitung grundlegender Probleme der Informatik
% File        : header.tex Rev. 01
% Date        : 01.03.2015
% Author      : Raffael Santschi
%
%%%%%%%%%%%%%%%%%%%%%%%%%%%%%%%%%%%%%%%%%%%%%%%%%%%%%%%%%%%%%%%%%

\usepackage{xparse}
\usepackage[nopostdot,nonumberlist,acronym]{glossaries}
\DeclareDocumentCommand{\newdualentry}{O{} O{} m m m m }{%
  \newglossaryentry{gls-#3}{name={#5},text={#5\glsadd{#3}},
    description={#6},#1
  }
  \newacronym[#2]{#3}{#4}{\gls{gls-#3}}
}
\makeglossaries
%\renewcommand*{\glstextformat}[1]{\textsf{#1}}
% Customize the format of the first use.  See the manual for details if
% you want to include more information here such as the definition.
\defglsdisplayfirst[\glsdefaulttype]{\textit{#1}}

\newdualentry{sql}    % label
  {SQL}               % abbreviation
  {Structured Query Language}  % long form
  {Abfrage- und Definitionssprache von Datenbanken.}
\newdualentry{blob}    % label
  {BLOB}               % abbreviation
  {Binary Large Object}  % long form
  {Datentyp von Datenbanken für grosse, nicht weiter strukturierte, binäre Objekte (z.B. Bilder oder PDF-Dateien).}
\newdualentry{rdbms}    % label
  {RDBMS}               % abbreviation
  {relationales Datenbankmanagementsystem}  % long form
  {Das Grundkonzept dieser Datenbank ist die Relation.}
\newdualentry{ordbms}    % label
  {ORDBMS}               % abbreviation
  {objektrelationales Datenbankmanagementsystem}  % long form
  {Erweitert das Konzept der \gls{rdbms} mit Funktionen des objektorientirten Stiles.}
\newdualentry{oodbms}    % label
  {OODBMS}               % abbreviation
  {objektorientiertes Datenbanksystem}  % long form
  {Das Konzept dieser Datenbanken zielt eine eine bessere und nähere Zusammenarbeit mit objektorientierten Programmiersprachen}
\newdualentry{api}    % label
  {API}               % abbreviation
  {Application Programming Interface}
  {Application Programming Interace ist eine Schnittstelle, über welche anderen Applikationen Leistungen beziehen können.}
\newdualentry{rest}    % label
  {REST}               % abbreviation
  {Representational State Transfer}
  {Programmierparadigma für die Implementation von Webservices. Es bietet eine Kommunikationsschnittstelle für Webanwendung und 
wird vorallem für Syste-System-Kommunikation verwendet. REST ist Zustandslos und liefert somit bei einem Aufruf einer URL genau 
jedes Mal den selben Inhalt zurück.}
\newdualentry{json}    % label
  {JSON}               % abbreviation
  {JavaScript Object Notation}  % long form
  {Format zum Austausch von strukturierterten Daten, bekannt für seine gute Lesbarkeit und wenig Overhead.}
\newdualentry{xml}    % label
  {XML}               % abbreviation
  {Extensible Markup Language}  % long form
  {Markup Sprache zum Austausch strukturierter Daten.}
\newdualentry{ide}    % label
  {IDE}               % abbreviation
  {Integrated Development Environment}  % long form
  {Eine integrierte Entwicklerumgebung (englisch Integrated Development Environment) beinhalten einen Texteditor, 
Compiler (falls benötigt), Debugger, Formatierungsfunktionen und zum Teil die Möglichkeit zur Erstellung von grafischen 
Benutzeroberflächen.}
\newdualentry{orm}    % label
  {ORM}               % abbreviation
  {Object-relational mapping}  % long form
  {Objektrelationale Abbildung (englisch object-relational mapping) ist eine Technik der Softwareentwicklung, mit der ein in einer 
objektorientierten Programmiersprache geschriebenes Anwendungsprogramm seine Objekte in einer relationalen Datenbank ablegen 
kann. Dem Programm erscheint die Datenbank dann als objektorientierte Datenbank, was die Programmierung erleichtert. 
\cite{wiki_orm}}
\newdualentry{oql}    % label
  {OQL}               % abbreviation
  {Object Query Language}  % long form
  {Object Query Language ist stark an SQL angelehnt, wobei nicht mit den Spalten der Tabelle sondern mit den Attributen des Objekts 
Abfragen erstellt werden.}
\newdualentry{odl}    % label
  {ODL}               % abbreviation
  {Object Definition Language}  % long form
  {Definitionssprache, welche Ähnlichkeiten mit der Definition von Objekten in objektorientierten Programmiersprachen hat.}
\newdualentry{acid}    % label
  {ACID}               % abbreviation
  {Atomicity, Consistency, Isolation, Durability}  % long form
  {Häufig erwünschte Eigenschaften von Datenbanken, welche bei relationalen Datenbanken durch Transaktionen realisiert werden.}
\newdualentry{base}    % label
  {BASE}               % abbreviation
  {Basically Available, Soft state, Eventual consistency}  % long form
  {Eine etwas entschärfte Variante von \gls{acid}, bei welcher es mehr um die Verfügbarkeit und die Schnelligkeit geht, was mit einer 
weichen Konsistenz erreicht wird.}

\newglossaryentry{spring}{%
  name={Spring},
  description={Weitverbreitetes Java-Framework zur Erstellung von Enterprise-Applikationen.}
}
\newglossaryentry{slack}{%
  name={Slack},
  description={Chat Applikation mit vielen Integrationen und facettenreichen Webservices.}
}
\newglossaryentry{yaml}{%
  name={YAML},
  description={Markup Language mit wenig Overhead und in einer menschenlesbarer Form. Der Name ist ein recursives Akronym für "`YAML Ain't Markup Language"' (früher "`Yet Another Markup Language"').}
}
\newglossaryentry{git}{%
  name={git},
  description={Tool zur verteilten Versionsverwaltung. Ehemals für die Verwaltung des Linux Kernels entwickelt, besitzt es heute grosse Popularität in der gesammten Softwareentwicklung und ein ausgeprägtes Ökosystem.}
}
\newglossaryentry{basisfaktor}{%
  name={Basisfaktor},
  description={Basisfaktoren (unterbewusste Anforderungen) muss das System in jedem Fall vollständig erfüllen, sonst stellt sich beim 
\gls{stakeholder} massive Unzufriedenheit ein. \cite{req_eng_book}}
}
\newglossaryentry{entity}{%
  name={Entity},
  description={Entity (auch Entität) ist ein eindeutig zu bestimmendes Daten-Objekt.}
}
\newglossaryentry{sonar}{%
  name={Sonar},
  description={Sonar ist ein statisches Code-Analyse Tool.}
}
\newglossaryentry{stakeholder}{%
  name={Stakeholder},
  description={Stakeholder sind für den Requirement Engineer wichtige Quellen zur Identifikation möglicher Anforderungen des 
Systems.\cite{req_eng_book} Ein Stakeholder ist eine Person, die in irgendeiner Weise vom Projekt betroffen ist, jedoch nicht 
notwendigerweise direkt Einfluss auf den Projektverlauf haben muss.}
}
\newglossaryentry{poll_principle}{%
  name={Poll-Prinzip},
  description={Beim Poll-Prinzip fragt der Nutzer nach, ob eine Aktion beendet ist. Bei lange Berechnungen kann dies zu sehr vielen 
unnötigen Requests führen, bei diesem Anwendungsfall wäre das \gls{push_principle} sinnvoller.}
}
\newglossaryentry{push_principle}{%
  name={Push-Prinzip},
  description={Beim Push-Prinzip wird der Nutzer vom System aktiv benachrichtigt, wenn eine Aktion beendet ist.}
}
\newglossaryentry{webhook}{%
  name={WebHook},
  description={Ein nicht standardisiertes Verfahren, um ständiges \glslink{poll_principle}{Polling} zu umgehen. Dabei wird nach einem gewissen Event ein 
POST-Request an die angegebene URL geschickt.}
}
\newglossaryentry{eulerkreis}{%
  name={Eulerkreis},
  description={Ein ungerichteter Graph, bei welchen kein Knoten eine ungerade Anzahl anliegender Kanten hat.}
}
\newglossaryentry{aussagenlogik}{%
  name={Aussagenlogik},
  description={Ein Teilgebiet der Logik, in welchem es um Aussagen und deren Verknüpfung geht. Für eine Aussage der Aussagenlogik 
kann genau bestimmt werden, ob sie wahr oder falsch ist.}
}
\newglossaryentry{semi_structured_data}{%
  name={semistrukturierte Daten},
  description={Daten, welche keiner allgemeine Struktur unterliegen. \gls{xml} ist eine sehr verbreitete Notation dafür.}
}
\newglossaryentry{domainlanguage}{%
  name={Domänensprache},
  description={Sprache für Begriffe in einer spezifische Umgebung bzw. einer gewissen Nutzergruppe.}
}