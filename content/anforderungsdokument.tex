%%%%%%%%%%%%%%%%%%%%%%%%%%%%%%%%%%%%%%%%%%%%%%%%%%%%%%%%%%%%%%%%%
%
% Project     : Bachelorarbeit
% Title       : Machbarkeitsanalyse für eine ressourcenorientierte Schnittstelle zur Verarbeitung grundlegender Probleme der Informatik
% File        : anforderungsdokument.tex Rev. 01
% Date        : 01.03.2015
% Author      : Raffael Santschi
%
%%%%%%%%%%%%%%%%%%%%%%%%%%%%%%%%%%%%%%%%%%%%%%%%%%%%%%%%%%%%%%%%%

\chapter{Anforderungsdokument \resultAssignment{[R3]}}\label{chap.anforderungsdokument}

Das Anforderungsdokument legt die Basis für die Implementation. Es ist für den Verlauf des Projekts wichtig, dass zu Beginn die Anforderungen aufgestellt werden. Bei der 
Erstellung des Anforderungsdokuments werden verschiedene Betrachtungsweisen aufgezeigt und die Anforderungen an das System in verschiedenen Detailstufen angeschaut.


\section{Übersicht}\label{anf_uebersicht}

In diesem Abschnitt  wird die System- und Kontextabgrenzung dargelegt, die Systemumgebung beschrieben, die getroffenen Annahmen festgehalten und die verschiedenen \gls{stakeholder} mit 
ihren Erwartungen aufgelistet.

\subsection{System- und Kontextabgrenzung}\label{systemabgrenzung}
Der Systemkontext umfasst alle Aspekte, die für die Anforderungen des geplanten Systems relevant sind und nicht im Rahmen der Entwicklung dieses System gestaltet werden können.
\cite{req_eng_book} 

\begin{figure}[h]
\centering
\includegraphics[scale=0.8]{images/visio/systemkontext.png}
\caption[Systemkontext]{Systemkontext \selfmade{}}
\label{fig:systemkontext}
\end{figure}

Der Systemkontext (siehe \autoref{fig:systemkontext}) zeigt, dass das System relevante Schnittstellen zu den Nutzern und zum Verarbeitungssystem hat. Das System muss Daten 
für das Verarbeitungssystem zur Verfügung stellen und auch solche annehmen. Zudem wird das System von den Nutzern, Optimierungsproblemen und den dazugehörigen Algorithmen 
beeinflusst.

\FloatBarrier
\subsection{Systemumgebung}\label{systemumgebung}
\autoref{fig:systemumgebung} zeigt die Systemumgebung, welche die Ausgangslage für das Projekt definiert. Am Anfang des Projekts war bekannt, dass Nutzer und ein Verarbeitungssystem 
Dienste des zu erstellenden Systems beziehen werden. Die genaue Ausprägung dieser Dienste werden in diesem Kapitel behandelt.

\begin{figure}[h]
\centering
\includegraphics[scale=0.8]{images/visio/systemumgebung.png}
\caption[Systemumgebung]{Systemumgebung \selfmade{}}
\label{fig:systemumgebung}
\end{figure}

\FloatBarrier

\subsection{Stakeholder}\label{stakeholder}
Zum Erfassen aller Nutzergruppen, die Einfluss auf das Projekt haben können, dient die \gls{stakeholder}-Analyse. Zudem ermöglicht sie die Erfassung aller Gruppen, die potenziell 
Anforderungen an das Projekt stellen. In \autoref{table:stakeholder} wurden die \gls{stakeholder} dieses Projekts zusammengetragen und ihre Erwartungen, ihre 
Einstellung und ihr Einfluss gegenüber dem Projekt festgehalten. Da es sich um eine Machbarkeitsanalyse handelt, befinden sich nur der Auftraggeber, ein einzelner potenzieller Kunde und der 
Entwickler in der Auflistung.

\begin{table}[ht]
\centering
  \begin{tabular}{ p{5cm} | p{5cm} | p{1.5cm} | p{1.5cm} }
	\hline
	\rowcolor{darkgray}
	\textbf{Name}					&	\textbf{Erwartung}	&	\textbf{Einstellung} 	&	\textbf{Einfluss}	\\ \hline
	\rowcolor{gray}
								&				&	-Positiv \mbox{-Neutral} \mbox{-Negativ} 	&	-Hoch \mbox{-Mittel} \mbox{-Niedrig} \\ \hline
	\textbf{Phil Hofmann} (Vorsteher der Geschäftsführung der 200ok GmbH)						
								&	Der Auftraggeber in diesem Projekt erwartet von der Machbarkeitsanalyse Informationen für ein mögliches Projekt zur 
									Umsetzung der Gesamtidee.
												& 	Positiv		&	Hoch		\\ \hline
	\textbf{Potenzieller Kunde}
								&	Er wünscht sich eine einfache Abwicklung für seine Probleme, er möchte sich nicht mit Algorithmen und 
									theoretischer Informatik beschäftigen.
												& 	Positiv		&	Mittel		\\ \hline
	\textbf{Entwickler}
								&	Er hofft, dass sich die Schnittstelle wie gewünscht umsetzen lässt und aus der Machbarkeitsanalyse ein weiteres Projekt entsteht.
												& 	Positiv		&	Mittel		\\ \hline												
  \end{tabular}
   \caption{Liste der \gls{stakeholder}}\label{table:stakeholder}
\end{table}

\newpage
\section{Anforderungen}\label{sec.anfoderungen}
Zum Erfassen der Anforderungen an das System wurden zuerst verschiedene Use Cases definiert, mit deren Hilfe anschliessend der Anforderungskatalog erstellt werden konnte.

\subsection{Use Cases}\label{use_cases}
Das Use Case Diagramm (siehe Abbildung \ref{fig:use_case}) zeigt einen Akteur, ein System und sechs Use Cases, welche für diese Arbeit relevant sind.
\begin{figure}[h]
\includegraphics{images/anforderungen/use_cases.png}
\caption[Use-Case Diagramm]{Use-Case Diagramm \selfmade{}}
\label{fig:use_case}
\end{figure}

Alle Use Cases wurden anhand der Vorlage in \autoref{table:use_case_template} spezifiziert. Diese Vorlage basiert auf Angaben von \cite{req_eng_book}.

\begin{table}[ht]
\centering
  \begin{tabular}{ l | p{10cm} }
	\hline
	\rowcolor{darkgray}
	\textbf{Attribute}&	\textbf{Beschreibung}\\ \hline
	\rowcolor{gray}
	\textbf{Bezeichner}&	\textbf{Eindeutiger Bezeichner}\\ \hline
	\textbf{Name}		&	Eindeutiger Name des Use Case\\ \hline
	\textbf{Beschreibung}	&	Komprimierte Beschreibung\\ \hline
	\textbf{Auslösendes Ereignis} &	Ereignis, das den Use Case auslöst.\\ \hline
	\textbf{Akteure}		&	Auflistung der Akteure, die mit dem Use Case in Beziehung stehen.\\ \hline
	\textbf{Vorbedingung}	&	Liste notwendiger Voraussetzungen, die erfüllt sein müssen, bevor die Ausführung des Use Case beginnen kann.\\ \hline
	\textbf{Nachbedingung}	&	Liste von Zuständen, in denen sich das System unmittelbar nach der Ausführung des Hauptszenarios befindet.\\ \hline
	\textbf{Ergebnis}		&	Beschreibung der Ausgaben, die während der Ausführung des Use Case erzeugt werden.\\ \hline
	\textbf{Hauptszenario}	&	Beschreibung des Hauptszenarios des Use Case\\ \hline
	\textbf{Alternativszenarien}	&	Beschreibung von Alternativszenarien des Use Case oder Angabe der auslösenden Ereignisse. 
					Hier gelten oftmals andere Nachbedingungen.\\ \hline
  \end{tabular}
   \caption{Vorlage für Use Case Spezifikation}\label{table:use_case_template}
\end{table}

\begin{table}[ht]
\centering
  \begin{tabular}{ l | p{10cm} }
	\hline
	\rowcolor{gray}
	\textbf{Bezeichner}	&	\textbf{UC-1}\\ \hline
	\textbf{Name}		&	Lösung beauftragen\\ \hline
	\textbf{Beschreibung}	&	Ein Nutzer möchte eine Lösung eines Problem mit spezifischen Parametern beauftragen.\\ \hline
	\textbf{Auslösendes Ereignis} &	Nutzer möchte ein Problem lösen.\\ \hline
	\textbf{Akteure}		&	Nutzer\\ \hline
	\textbf{Vorbedingung}	&	Das System bietet zur Lösung dieses Problems eine Schnittstelle.\\ \hline
	\textbf{Nachbedingung}	&	Das System hat die nötigen Informationen für die Lösung des Problems und der Nutzer erhält eine ID, mit welcher er den Status bzw. das Resultat 
						abfragen kann.\\ \hline
	\textbf{Ergebnis}		&	Erfassung der Informationen für die Lösung des Problems\\ \hline
	\textbf{Hauptszenario}	&	\begin{enumerate}
					\item Der Nutzer ruft die Funktion für das zu lösende Problem mit den Parametern auf.
					\item Das System speichert die Parameter für die weitere Verarbeitung.
					\item Der Nutzer erhält eine ID für das Abrufen des Status bzw. des Resultats.
					\end{enumerate}
					\\ \hline
	\textbf{Alternativszenarien}	&	\begin{enumerate}
					\item[3a] Der Nutzer erhält eine Fehlermeldung, wenn das Problem nicht korrekt erfasst werden konnte.
					\end{enumerate}
					\\ \hline
  \end{tabular}
   \caption{Use Case UC-1: Lösung beauftragen}\label{table:use_case_1}
\end{table}

\begin{table}[ht]
\centering
  \begin{tabular}{ l | p{10cm} }
	\hline
	\rowcolor{gray}
	\textbf{Bezeichner}	&	\textbf{UC-2}\\ \hline
	\textbf{Name}			&	Berechnung starten\\ \hline
	\textbf{Beschreibung}	&	Das System startet die Berechnung beim Verarbeitungssystem.\\ \hline
	\textbf{Auslösendes Ereignis} &	Nutzer möchte ein Problem lösen.\\ \hline
	\textbf{Akteure}		&	Verarbeitungssystem\\ \hline
	\textbf{Vorbedingung}	&	Die Informationen für die Lösung des Problems sind erfasst.\\ \hline
	\textbf{Nachbedingung}	&	Das Verarbeitungssystem beginnt mit der Berechnung.\\ \hline
	\textbf{Ergebnis}		&	Starten der Berechnung\\ \hline
	\textbf{Hauptszenario}	&	\begin{enumerate}
					\item Das System startet die Berechnung. 
					\item Das System übergibt eine ID für das Abrufen der abgelegten Daten.
					\end{enumerate}
					\\ \hline
	\textbf{Alternativszenarien}	&	\begin{enumerate}
					\item[2a] Das System speichert die Fehlermeldung, falls das Starten der Berechnung fehlschlägt.
					\end{enumerate}
					\\ \hline
  \end{tabular}
   \caption{Use Case UC-2: Berechnung starten}\label{table:use_case_2}
\end{table}

\begin{table}[ht]
\centering
  \begin{tabular}{ l | p{10cm} }
	\hline
	\rowcolor{gray}
	\textbf{Bezeichner}	&	\textbf{UC-3}\\ \hline
	\textbf{Name}			&	Eingabe Parameter abholen\\ \hline
	\textbf{Beschreibung}	&	Das Verarbeitungssystem benötigt für die Lösung des Problems die Eingabeparameter.\\ \hline
	\textbf{Auslösendes Ereignis}&	Das Verarbeitungssystem startet eine neue Berechnung.\\ \hline
	\textbf{Akteure}		&	Verarbeitungssystem\\ \hline
	\textbf{Vorbedingung}	&	Das Verarbeitungssystem wurde angestossen, das Problem zu lösen.\\ \hline
	\textbf{Nachbedingung}	&	Das Verarbeitungssystem hat die Eingabeparameter erhalten.\\ \hline
	\textbf{Ergebnis}		&	Erhalt von Eingabeparametern\\ \hline
	\textbf{Hauptszenario}	&	\begin{enumerate}
					\item Das Verarbeitungssystem fordert die Eingabeparameter für das zu lösende Problem an.
					\item Das System leitet die Eingabeparameter weiter.
					\item Das Verarbeitungssystem erhält die Eingabeparameter.
					\end{enumerate}
					\\ \hline
	\textbf{Alternativszenarien}	&	\begin{enumerate}
					\item[2a] Das System liefert eine Fehlermeldung zurück, falls keine Eingabeparameter vorhanden sind.
					\end{enumerate}
					\\ \hline
  \end{tabular}
   \caption{Use Case UC-3: Eingabe Parameter abholen}\label{table:use_case_3}
\end{table}


\begin{table}[ht]
\centering
  \begin{tabular}{ l | p{10cm} }
	\hline
	\rowcolor{gray}
	\textbf{Bezeichner}	&	\textbf{UC-4}\\ \hline
	\textbf{Name}			&	Status abfragen\\ \hline
	\textbf{Beschreibung}	&	Der Nutzer kann den Status einer Berechnung abfragen, da die Verarbeitung einige Zeit benötigen könnte.\\ \hline
	\textbf{Auslösendes Ereignis}&	Der Nutzer möchte den Status der Berechnung wissen.\\ \hline
	\textbf{Akteure}		&	Nutzer\\ \hline
	\textbf{Vorbedingung}	&	Der Nutzer hat bereits eine Berechnung beauftragt und kennt die ID.\\ \hline
	\textbf{Nachbedingung}	&	Der Nutzer kennt den Status der Berechnung.\\ \hline
	\textbf{Ergebnis}		&	Kenntnis des Status\\ \hline
	\textbf{Hauptszenario}	&	\begin{enumerate}
					\item Der Nutzer fragt den Status einer Berechnung ab.
					\item Das System fragt den Status in der Datenbank ab.
					\item Das System sendet den Status zurück.
					\item Der Nutzer erhält den Status.
					\end{enumerate}
					\\ \hline
	\textbf{Alternativszenarien}	&	\begin{enumerate}
					\item[3a] Das System sendet eine Fehlermeldung zurück, falls der Status nicht ermittelt werden kann.
					\end{enumerate}
					\\ \hline
  \end{tabular}
   \caption{Use Case UC-4: Status abfragen}\label{table:use_case_4}
\end{table}

\begin{table}[ht]
\centering
  \begin{tabular}{ l | p{10cm} }
	\hline
	\rowcolor{gray}
	\textbf{Bezeichner}	&	\textbf{UC-5}\\ \hline
	\textbf{Name}			&	Resultat speichern\\ \hline
	\textbf{Beschreibung}	&	Um das Resultat nicht zu verlieren, muss das Ergebnis nach der Berechnung gespeichert werden.\\ \hline
	\textbf{Auslösendes Ereignis}&	Das Verarbeitungssystem möchte das Resultat speichern.\\ \hline
	\textbf{Akteure}		&	Verarbeitungssystem\\ \hline
	\textbf{Vorbedingung}	&	Das Verarbeitungssystem hat ein Resultat berechnet.\\ \hline
	\textbf{Nachbedingung}	&	Das Resultat ist gespeichert.\\ \hline
	\textbf{Ergebnis}		&	Speicherung des Resultats\\ \hline
	\textbf{Hauptszenario}	&	\begin{enumerate}
					\item Das Verarbeitungssystem schickt das Resultat der Berechnung an das System.
					\item Das System erhält das Resultat.
					\item Das System speichert das Resultat.
					\item Das System bestätigt das Erhalten des Resultats.
					\item Das Verarbeitungssystem erhält die Bestätigung.
					\end{enumerate}
					\\ \hline
	\textbf{Alternativszenarien}	&	\begin{enumerate}
					\item[4a] Das System liefert eine Fehlermeldung zurück, wenn das Resultat nicht korrekt gespeichert werden konnte.
					\end{enumerate}
					\\ \hline
  \end{tabular}
   \caption{Use Case UC-5: Resultat speichern}\label{table:use_case_5}
\end{table}

\begin{table}[ht]
\centering
  \begin{tabular}{ l | p{10cm} }
	\hline
	\rowcolor{gray}
	\textbf{Bezeichner}	&	\textbf{UC-6}\\ \hline
	\textbf{Name}			&	Resultat abholen\\ \hline
	\textbf{Beschreibung}	&	Der Nutzer holt das Resultat zu einem bestimmten Zeitpunkt ab.\\ \hline
	\textbf{Auslösendes Ereignis}&	Der Nutzer möchte das Resultat der Berechnung abholen.\\ \hline
	\textbf{Akteure}		&	Nutzer\\ \hline
	\textbf{Vorbedingung}	&	Der Nutzer hat bereits eine Berechnung beauftragt und kennt die ID.\\ \hline
	\textbf{Nachbedingung}	&	Der Nutzer hat das Resultat erhalten.\\ \hline
	\textbf{Ergebnis}		&	Erhalt des Resultats\\ \hline
	\textbf{Hauptszenario}	&	\begin{enumerate}
					\item Der Nutzer fragt das Resultat der Berechnung ab.
					\item Das System sucht das Resultat der Berechnung.
					\item Das System sendet das Resultat.
					\item Der Nutzer erhält das Resultat.
					\end{enumerate}
					\\ \hline
	\textbf{Alternativszenarien}	&	\begin{enumerate}
					\item[3a] Das System sendet eine Fehlermeldung, wenn kein Resultat vorhanden ist.
					\end{enumerate}
					\\ \hline
  \end{tabular}
   \caption{Use Case UC-6: Resultat abholen}\label{table:use_case_6}
\end{table}

\newpage
\FloatBarrier
\subsection{Anforderungen}\label{anforderungen}
Alle Anforderungen wurden anhand der folgenden Vorlage (siehe Tabelle \ref{table:req_template}) erfasst. Diese Vorlage basiert auf Angaben von \cite{req_eng_book} und wurde um 
eigene Attribute erweitert.

\begin{table}[ht]
\centering
  \begin{tabular}{ l | p{8cm} }
	\hline
	\rowcolor{gray}
	\textbf{Bezeichner}&	\textbf{Eindeutiger Identifikator}\\ \hline
	\textbf{Priorität} 		&	Must, Should, Nice to have\\ \hline
	\textbf{Anforderungstyp}	&	Funktionale Anforderung, Qualitätsanforderung, Randbedingung\\ \hline
	\textbf{Name} 			&	Eindeutiger, charakterisierender Name\\ \hline
	\textbf{Use Case} 		&	Referenz zum zugehörigen Use Case\\ \hline
	\textbf{Beschreibung} 	&	Beschreibung der Anforderung\\ \hline
	\textbf{Begründung} 		&	Bedeutung der Anforderung für das geplante System\\ \hline
	\textbf{Akzeptanz Kriterium}	&	Messbare Abnahmekriterien\\ \hline
	\textbf{Abhängigkeiten} 	&	Referenz zu anderen Anforderungen\\ \hline
  \end{tabular}
   \caption{Vorlage für Anforderungen}\label{table:req_template}
\end{table}


Die Beschreibung der Anforderungen wurden zusätzlich mit der Satzschablone in \autoref{fig:satzschablone}) aus \cite{req_eng_book} erstellt. Dies hat den Vorteil, dass die 
Anforderungen normiert und exakt sind. Die Abstufungen "`muss"' und "`sollte"' werden verwendet, um die Wichtigkeit der Anforderungen auszudrücken.
\begin{figure}[h]
\includegraphics[scale=0.95]{images/anforderungen/satzschablone.png}
\caption[Satzschablone]{Satzschablone (Grafik entnommen aus \cite{req_eng_book})}
\label{fig:satzschablone}
\end{figure}


\newpage
\FloatBarrier
\subsubsection{Funktionale Anforderungen}\label{func_anforderungen}
Mit den funktionalen Anforderungen wird festgelegt, was die Schnittstelle tun sollte.

\begin{table}[ht]
\centering
  \begin{tabular}{ l | p{8cm} }
	\hline
	\rowcolor{gray}
	\textbf{Bezeichner}&	\textbf{RE-F1}\\ \hline
	\textbf{Priorität} 		&	Must\\ \hline
	\textbf{Anforderungstyp}	&	Funktionale Anforderung\\ \hline
	\textbf{Name} 			&	Bereitstellung der Schnittstellen für Probleme mit hoher Laufzeitkomplexität\\ \hline
	\textbf{Use Case} 		&	\nameref{table:use_case_1}\\ \hline
	\textbf{Beschreibung} 	&	Das System muss dem Nutzer die Möglichkeit bieten, die Lösung verschiedener Probleme zu beauftragen.\\ \hline
	\textbf{Begründung} 		&	Die Schnittstellen ist die Anlaufstelle des Nutzers. Er beauftragt das System, eine Berechnung zu starten.\\ \hline
	\textbf{Akzeptanz Kriterium}	&	\begin{enumerate}
					\item Der Nutzer kann eine Schnittstelle für die bereitgestellten Berechnungsfunktionen ansprechen.
					\end{enumerate}
					\\ \hline
	\textbf{Abhängigkeiten} 	&	-\\ \hline
  \end{tabular}
   \caption{Anforderung RF-F1}\label{table:req_1}
\end{table}

\begin{table}[ht]
\centering
  \begin{tabular}{ l | p{8cm} }
	\hline
	\rowcolor{gray}
	\textbf{Bezeichner}&	\textbf{RE-F2}\\ \hline
	\textbf{Priorität} 		&	Must\\ \hline
	\textbf{Anforderungstyp}	&	Funktionale Anforderung\\ \hline
	\textbf{Name} 			&	Speicherung der Eingabeparameter\\ \hline
	\textbf{Use Case} 		&	\nameref{table:use_case_1}\\ \hline
	\textbf{Beschreibung} 	&	Falls eine Berechnung in Auftrag gegeben wurde, muss das System fähig sein, die Eingabeparameter abzuspeichern.\\ \hline
	\textbf{Begründung} 		&	Die Berechnung wird von einem anderen System ausgeführt. Damit diese auf die Parameter zugreifen können, müssen die Daten persistiert werden.\\ \hline
	\textbf{Akzeptanz Kriterium}	&	\begin{enumerate}
					\item Die Parameter sind persistent.
					\item Es gibt eine Fehlermeldung, falls bei der Speicherung etwas fehlschlägt oder die Eingabeparameter nicht gültig sind.
					\end{enumerate}
					\\ \hline
	\textbf{Abhängigkeiten} 	&	\\ \hline
  \end{tabular}
   \caption{Anforderung RF-F2}\label{table:req_3}
\end{table}

\begin{table}[ht]
\centering
  \begin{tabular}{ l | p{8cm} }
	\hline
	\rowcolor{gray}
	\textbf{Bezeichner}&	\textbf{RE-F3}\\ \hline
	\textbf{Priorität} 		&	Must\\ \hline
	\textbf{Anforderungstyp}	&	Funktionale Anforderung\\ \hline
	\textbf{Name} 			&	Rückgabe einer ID bei Beauftragung\\ \hline
	\textbf{Use Case} 		&	\nameref{table:use_case_1}\\ \hline
	\textbf{Beschreibung} 	&	Falls eine Berechnung in Auftrag gegeben wurde, muss das System dem Ersteller eine ID zurückliefern.\\ \hline
	\textbf{Begründung} 		&	Die ID hilft dem Nutzer den Status der Berechnung abzuholen und wird am Schluss für das Resultat benötigt.\\ \hline
	\textbf{Akzeptanz Kriterium}	&	\begin{enumerate}
					\item Der Nutzer erhält nach dem Start einer Berechnung eine ID.
					\end{enumerate}
					\\ \hline
	\textbf{Abhängigkeiten} 	&	\nameref{table:req_3}\\ \hline
  \end{tabular}
   \caption{Anforderung RF-F3}\label{table:req_2}
\end{table}

\begin{table}[ht]
\centering
  \begin{tabular}{ l | p{8cm} }
	\hline
	\rowcolor{gray}
	\textbf{Bezeichner}&	\textbf{RE-F4}\\ \hline
	\textbf{Priorität} 		&	Must\\ \hline
	\textbf{Anforderungstyp}	&	Funktionale Anforderung\\ \hline
	\textbf{Name} 			&	Start der Berechnung\\ \hline
	\textbf{Use Case} 		&	\nameref{table:use_case_2}\\ \hline
	\textbf{Beschreibung} 	&	Falls eine Berechnung in Auftrag gegeben wurde, muss das System fähig sein, die Berechnung beim Verarbeitungssystem zu starten.\\ \hline
	\textbf{Begründung} 		&	Der Nutzer kennt das Verarbeitungssystem nicht, das System muss dem Verarbeitungssystem den Start-Befehl geben.\\ \hline
	\textbf{Akzeptanz Kriterium}	&	\begin{enumerate}
					\item Der Befehl für den Start wird versendet und die ID dabei übergeben.
					\item Die Fehlermeldung bei einem Fehlversuch wird gespeichert.
					\end{enumerate}
					\\ \hline
	\textbf{Abhängigkeiten} 	&	\nameref{table:req_3}\\ \hline
  \end{tabular}
   \caption{Anforderung RF-F4}\label{table:req_4}
\end{table}

\begin{table}[ht]
\centering
  \begin{tabular}{ l | p{8cm} }
	\hline
	\rowcolor{gray}
	\textbf{Bezeichner}&	\textbf{RE-F5}\\ \hline
	\textbf{Priorität} 		&	Must\\ \hline
	\textbf{Anforderungstyp}	&	Funktionale Anforderung\\ \hline
	\textbf{Name} 			&	Abfrage der Eingabeparameter\\ \hline
	\textbf{Use Case} 		&	\nameref{table:use_case_3}\\ \hline
	\textbf{Beschreibung} 	&	Falls eine Berechnung in Auftrag gegeben wurde, muss das System dem Verarbeitungssystem die Möglichkeit bieten, die Eingabeparameter abzufragen.\\ \hline
	\textbf{Begründung}		&	Damit das Verarbeitungssystem die Berechnung durchführen kann, braucht es die Eingabeparameter.\\ \hline
	\textbf{Akzeptanz Kriterium}	&	\begin{enumerate}
					\item Das Verarbeitungssystem erhält die Eingabeparameter.
					\item Das Verarbeitungssystem erhält eine Fehlermeldung, falls keine Eingabeparameter vorhanden sind.
					\end{enumerate}
					\\ \hline
	\textbf{Abhängigkeiten} 	&	\nameref{table:req_3}\\ \hline
  \end{tabular}
   \caption{Anforderung RF-F5}\label{table:req_5}
\end{table}

\begin{table}[ht]
\centering
  \begin{tabular}{ l | p{8cm} }
	\hline
	\rowcolor{gray}
	\textbf{Bezeichner}&	\textbf{RE-F6}\\ \hline
	\textbf{Priorität} 		&	Should\\ \hline
	\textbf{Anforderungstyp}	&	Funktionale Anforderung\\ \hline
	\textbf{Name} 			&	Abfrage des Status\\ \hline
	\textbf{Use Case} 		&	\nameref{table:use_case_4}\\ \hline
	\textbf{Beschreibung} 	&	Falls eine Berechnung in Auftrag gegeben wurde, sollte das System dem Nutzer die Möglichkeit bieten, den Status der Berechnung abzufragen.\\ \hline
	\textbf{Begründung} 		&	Da die Verarbeitung asynchron läuft, weiss der Benutzer nicht, wann seine Berechnung fertig ist.\\ \hline
	\textbf{Akzeptanz Kriterium}	&	\begin{enumerate}
					\item Der Nutzer erhält einen Status seiner Berechnung.
					\end{enumerate}
					\\ \hline
	\textbf{Abhängigkeiten} 	&	\nameref{table:req_2}\\ \hline
  \end{tabular}
   \caption{Anforderung RF-F6}\label{table:req_6}
\end{table}

\begin{table}[ht]
\centering
  \begin{tabular}{ l | p{8cm} }
	\hline
	\rowcolor{gray}
	\textbf{Bezeichner}&	\textbf{RE-F7}\\ \hline
	\textbf{Priorität} 		&	Nice to have\\ \hline
	\textbf{Anforderungstyp}	&	Funktionale Anforderung\\ \hline
	\textbf{Name} 			&	Registrierung eines Dienstes zur Benachrichtigung für Statusänderungen\\ \hline
	\textbf{Use Case} 		&	\nameref{table:use_case_4}\\ \hline
	\textbf{Beschreibung} 	&	Falls eine Berechnung in Auftrag gegeben und dazu ein Dienst zur Benachrichtigung eingetragen wurde, sollte das System den Nutzer über eine 
						Änderung des Status mittels dieses Dienstes informieren.\\ \hline
	\textbf{Begründung} 		&	Da die Verarbeitung lange dauern könnte, weiss der Benutzer nicht, wann seine Berechnung fertig ist. Um ein ständiges Pollen zu vermeiden, 
							können Dienst zur Benachrichtigung (zum Beispiel WebHooks) verwenden werden.\\ \hline
	\textbf{Akzeptanz Kriterium}	&	\begin{enumerate}
					\item Der Nutzer wird über die Änderung des Status auf dem eingetragen Dienst informiert.
					\end{enumerate}
					\\ \hline
	\textbf{Abhängigkeiten} 	&	\nameref{table:req_2}\\ \hline
  \end{tabular}
   \caption{Anforderung RF-F7}\label{table:req_7}
\end{table}

\begin{table}[ht]
\centering
  \begin{tabular}{ l | p{8cm} }
	\hline
	\rowcolor{gray}
	\textbf{Bezeichner}&	\textbf{RE-F8}\\ \hline
	\textbf{Priorität} 		&	Must\\ \hline
	\textbf{Anforderungstyp}	&	Funktionale Anforderung\\ \hline
	\textbf{Name} 			&	Speicherung des Resultats\\ \hline
	\textbf{Use Case} 		&	\nameref{table:use_case_5}\\ \hline
	\textbf{Beschreibung} 	&	Nach der Berechnung muss das System dem Verarbeitungssystem die Möglichkeit bieten, das Resultat abspeichern zu können.\\ \hline
	\textbf{Begründung} 		&	Das Resultat muss, bis der Nutzer es abholt, zwischengespeichert werden.\\ \hline
	\textbf{Akzeptanz Kriterium}	&	\begin{enumerate}
					\item Das Verarbeitungssystem kann das Resultat abspeichern.
					\item Das Verarbeitungssystem erhält eine Fehlermeldung, falls das Speichern fehlgeschlagen ist.
					\end{enumerate}
					\\ \hline
	\textbf{Abhängigkeiten} 	&	\nameref{table:req_4}\\ \hline
  \end{tabular}
   \caption{Anforderung RF-F8}\label{table:req_8}
\end{table}

\begin{table}[ht]
\centering
  \begin{tabular}{ l | p{8cm} }
	\hline
	\rowcolor{gray}
	\textbf{Bezeichner}&	\textbf{RE-F9}\\ \hline
	\textbf{Priorität} 		&	Must\\ \hline
	\textbf{Anforderungstyp}	&	Funktionale Anforderung\\ \hline
	\textbf{Name} 			&	Abfrage des Resultats\\ \hline
	\textbf{Use Case} 		&	\nameref{table:use_case_6}\\ \hline
	\textbf{Beschreibung} 	&	Das System muss dem Nutzer die Möglichkeit bieten, das Resultat der Berechnung abzufragen.\\ \hline
	\textbf{Begründung} 		&	Der Nutzer möchte das Resultat der Berechnung wissen.\\ \hline
	\textbf{Akzeptanz Kriterium}	&	\begin{enumerate}
					\item Der Nutzer erhält das Resultat der Berechnung.
					\item Der Nutzer erhält eine entsprechende Fehlermeldung, wenn beim Bereitstellen des Resultats ein Fehler aufgetreten ist.
					\end{enumerate}
					\\ \hline
	\textbf{Abhängigkeiten} 	&	\nameref{table:req_2}\\ \hline
  \end{tabular}
   \caption{Anforderung RF-F9}\label{table:req_9}
\end{table}

\newpage
\FloatBarrier
\subsubsection{Qualitätsanforderung}\label{non_func_anforderungen}
Die erfassten Qualitätsanforderungen definieren, was für Eigenschaften die Schnittstelle haben sollte.

\begin{table}[ht]
\centering
  \begin{tabular}{ l | p{8cm} }
	\hline
	\rowcolor{gray}
	\textbf{Bezeichner}&	\textbf{RE-NF1}\\ \hline
	\textbf{Priorität} 		&	Should\\ \hline
	\textbf{Anforderungstyp}	&	Qualitätsanforderung\\ \hline
	\textbf{Name} 			&	Prozess-agnostische Schnittstelle\\ \hline
	\textbf{Use Case} 		&	\nameref{table:use_case_1}\\ \hline
	\textbf{Beschreibung} 	&	Das System sollte fähig sein, die Lösung eines Problems so bereitzustellen, dass kein Wissen über den Verarbeitungsprozess erforderlich ist.\\ \hline
	\textbf{Begründung} 		&	Der Verarbeitungsprozess kann spezifisch und von Problem zu Problem unterschiedlich sein, der Nutzer sollte eine möglichst einfache 
							Schnittstelle dafür haben.\\ \hline
	\textbf{Akzeptanz Kriterium}	&	\begin{enumerate}
					\item Das Interface kann verwendet werden, ohne dass das Verarbeitungssystem bekannt ist.
					\end{enumerate}
					\\ \hline
	\textbf{Abhängigkeiten} 	&	-\\ \hline
  \end{tabular}
   \caption{Qualitätsanforderung RF-NF1}\label{table:req_nf_1}
\end{table}

\begin{table}[ht]
\centering
  \begin{tabular}{ l | p{8cm} }
	\hline
	\rowcolor{gray}
	\textbf{Bezeichner}&	\textbf{RE-NF2}\\ \hline
	\textbf{Priorität} 		&	Should\\ \hline
	\textbf{Anforderungstyp}	&	Qualitätsanforderung\\ \hline
	\textbf{Name} 			&	Entgegennahme generischer Eingabeparameter\\ \hline
	\textbf{Use Case} 		&	\nameref{table:use_case_1}\\ \hline
	\textbf{Beschreibung} 	&	Das System sollte in der Lage sein, unterschiedliche Ausprägungen von Eingabeparametern entgegenzunehmen.\\ \hline
	\textbf{Begründung} 		&	Da es bei den Problemen unterschiedliche Ausprägungen gibt, ist auf eine generische Deserialisierung der Eingabeparameter hinzuarbeiten.\\ \hline
	\textbf{Akzeptanz Kriterium}	&	\begin{enumerate}
					\item Unterschiedliche Ausprägungen eines Problems benutzen das gleiche API.
					\end{enumerate}
					\\ \hline
	\textbf{Abhängigkeiten} 	&	-\\ \hline
  \end{tabular}
   \caption{Qualitätsanforderung RF-NF2}\label{table:req_nf_2}
\end{table}

\begin{table}[ht]
\centering
  \begin{tabular}{ l | p{8cm} }
	\hline
	\rowcolor{gray}
	\textbf{Bezeichner}&	\textbf{RE-NF3}\\ \hline
	\textbf{Priorität} 		&	Should\\ \hline
	\textbf{Anforderungstyp}	&	Qualitätsanforderung\\ \hline
	\textbf{Name} 			&	Speicherung generischer Eingabeparameter\\ \hline
	\textbf{Use Case} 		&	\nameref{table:use_case_1}\\ \hline
	\textbf{Beschreibung} 	&	Das System sollte Eingabeparameter einheitlich abspeichern.\\ \hline
	\textbf{Begründung} 		&	Da es viele unterschiedliche Probleme gibt, ist eine generische Persistierung anzustreben.\\ \hline
	\textbf{Akzeptanz Kriterium}	&	\begin{enumerate}
					\item Unterschiedliche Probleme haben kein abweichendes Persistierungsschema.
					\end{enumerate}
					\\ \hline
	\textbf{Abhängigkeiten} 	&	-\\ \hline
  \end{tabular}
   \caption{Qualitätsanforderung RF-NF3}\label{table:req_nf_3}
\end{table}

\begin{table}[ht]
\centering
  \begin{tabular}{ l | p{8cm} }
	\hline
	\rowcolor{gray}
	\textbf{Bezeichner}&	\textbf{RE-NF4}\\ \hline
	\textbf{Priorität} 		&	Should\\ \hline
	\textbf{Anforderungstyp}	&	Qualitätsanforderung\\ \hline
	\textbf{Name} 			&	Erweiterbare Schnittstelle\\ \hline
	\textbf{Use Case} 		&	-\\ \hline
	\textbf{Beschreibung} 	&	Das System sollte einfach erweiterbar sein.\\ \hline
	\textbf{Begründung} 		&	Falls ein Nachfrage für ein anderes Problem besteht, wäre es gut, wenn die Schnittstelle mit wenig Aufwand erweitert werden könnte.\\ \hline
	\textbf{Akzeptanz Kriterium}	&	\begin{enumerate}
					\item Die Schnittstelle kann mit wenig Aufwand erweitert werden.
					\end{enumerate}
					\\ \hline
	\textbf{Abhängigkeiten} 	&	-\\ \hline
  \end{tabular}
   \caption{Qualitätsanforderung RF-NF4}\label{table:req_nf_4}
\end{table}

\FloatBarrier
\clearpage
\newpage

\subsection{Zusammenfassung der Anforderungen}\label{toc_anfoderungen}
Die Priorität der einzelnen Anforderungen ist wichtig, falls nicht alle Anforderungen umgesetzt werden können. Die Priorität wurde zusammen mit den \glslink{stakeholder}{Stakeholdern} 
festgelegt und in \autoref{table:req_priorities} zur besseren Übersicht zusammengetragen. Falls es nicht möglich gewesen wäre, alle Anforderungen in der Zeit umzusetzen, wären sie anhand 
dieser Listen zurückgestellt worden.

\begin{table}[ht]
\centering
  \begin{tabular}{ l | p{9cm}| l }
	\hline
	\rowcolor{gray}
	\textbf{Bezeichner}	& \textbf{Name}	&	\textbf{Priorität}\\ \hline
	RE-F1 			&  Bereitstellung der Schnittstellen für Probleme mit hoher Laufzeitkomplexität	& Must\\ \hline
	RE-F2 			&  Speicherung der Eingabeparameter	& Must\\ \hline
	RE-F3 			&  Rückgabe einer ID bei Beauftragung	& Must\\ \hline
	RE-F4 			&  Start der Berechnung	& Must\\ \hline
	RE-F5 			&  Abfrage der Eingabeparameter	& Must\\ \hline
	RE-F6 			&  Abfrage des Status	& Should\\ \hline
	RE-F7 			&  Registrierung eines Dienstes zur Benachrichtigung für Statusänderungen	& Nice to have\\ \hline
	RE-F8 			&  Speicherung des Resultats	& Must\\ \hline
	RE-F9 			&  Abfrage des Resultats	& Must\\ \hline
	RE-NF1 			&  Prozess-agnostische Schnittstelle & Should\\ \hline
	RE-NF2 			&  Entgegennahme generischer Eingabeparameter & Should\\ \hline
	RE-NF3 			&  Speicherung generischer Eingabeparameter & Should\\ \hline
	RE-NF4 			&  Erweiterbare Schnittstelle & Should\\ \hline	
  \end{tabular}
   \caption{Priorität der Anforderungen}\label{table:req_priorities}
\end{table}

\subsection{Annahmen}\label{annahmen}
Zwischen dem Endnutzer und dem zu erstellenden System existiert noch eine Sicherheitsschicht, welcher nicht Teil dieser Arbeit ist. Später wird dieses Projekt allenfalls um diese Sicherheitsschicht 
erweitert oder eine übergeordnete Schnittstelle erstellt, welche diese anspricht. 

Je nach Implementierung der Algorithmen ist das Ansteuern und die Aufbereitung der Daten unterschiedlich. Eine Referenzimplementierung zu jedem Problem zu finden, stellte sich als schwierig 
heraus. Deshalb wurden die Ein- und Ausgabe-Schemata der Algorithmen aus der Literaturrecherche abgeleitet.