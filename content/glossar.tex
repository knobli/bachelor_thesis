%%%%%%%%%%%%%%%%%%%%%%%%%%%%%%%%%%%%%%%%%%%%%%%%%%%%%%%%%%%%%%%%%
%
% Project     : Bachelorarbeit
% Title       : Machbarkeitsanalyse für eine ressourcenorientierte Schnittstelle zur Verarbeitung grundlegender Probleme der Informatik
% File        : glossar.tex Rev. 01
% Date        : 01.03.2015
% Author      : Raffael Santschi
%
%%%%%%%%%%%%%%%%%%%%%%%%%%%%%%%%%%%%%%%%%%%%%%%%%%%%%%%%%%%%%%%%%

\chapter*{Glossar}\label{glossar}
 \addcontentsline{toc}{chapter}{Glossar}

In diesem Abschnitt werden Abkürzungen und Begriffe kurz erklärt.

\begin{longtable}{>{\raggedright}m{3cm}m{11cm}}

\caption[Glossar]{\label{tbl:Abbr}Glossar}\\ 
\toprule
\textbf{Begriff}&\textbf{Bedeutung}\\ \midrule\addlinespace
\endfirsthead
\caption*{\textbf{Tabelle~\ref{tbl:Abbr} (Fortsetzung):} Glossar}\\ \toprule
\textbf{Abkürzung}&\textbf{Bedeutung}\\ \midrule\addlinespace
\endhead

\bottomrule\multicolumn{2}{>{\small\raggedleft\arraybackslash}r}{\slshape Fortsetzung auf der nächsten Seite}\\
\endfoot
\bottomrule
\endlastfoot		

	\textbf{API}&
	(siehe \glossarmark{Application Programming Interface})\\ \addlinespace

	\textbf{Application Programming Interface}&
	Application Programming Interace (\glossarmark{API}) ist eine Schnittstelle, über welche anderen Applikationen Leistungen beziehen kann. In diesem Projekt zum Beispiel die Informationen und Funktionen, welche vom Backend zur Verfügung gestellt werden.\\ \addlinespace

	\textbf{Basisfaktor}&
	Basisfaktoren (unterbewusste Anforderungen) muss das System in jedem Fall vollständig erfüllen, sonst stellt sich beim \glossarmark{Stakeholder} massive Unzufriedenheit ein. \cite{req_eng_book}\\ \addlinespace	

	\textbf{Begeisterungsfaktor}&
	Begeisterungsfaktoren (unbewusste Anforderungen) sind Merkmale eines Systems, deren Wert ein \glossarmark{Stakeholder} erst erkennt, wenn er sie selbst ausprobieren kann oder sie vom Requirements Engineer vorgeschlagen werden.\cite{req_eng_book}\\ \addlinespace	

	\textbf{Entity}&
	Entity (auch Entität) ist ein eindeutig zu bestimmendes Daten-Objekt \\ \addlinespace	

	\textbf{Integrated Development Environment}&
	Eine integrierte Entwicklerumgebung (englisch Integrated Development Environment) beinhalten einen Texteditor, Compiler (falls dieser benötigt wird), Debugger, Formatierungsfunktionen und in dem Kontext der App Entwicklung die Möglichkeit der Erstellung von grafischen Benutzeroberflächen.\\ \addlinespace

	\textbf{IDE}&
	(siehe \glossarmark{Integrated Development Environment})\\ \addlinespace

	\textbf{Object-relational mapping}&
	Objektrelationale Abbildung (englisch object-relational mapping, \glossarmark{ORM}) ist eine Technik der Softwareentwicklung, mit der ein in einer objektorientierten Programmiersprache geschriebenes Anwendungsprogramm seine Objekte in einer relationalen Datenbank ablegen kann. Dem Programm erscheint die Datenbank dann als objektorientierte Datenbank, was die Programmierung erleichtert. \cite{wiki_orm}\\ \addlinespace	

	\textbf{Object Query Language}&
	Object Query Language (\glossarmark{OQL}) ist stark an SQL angelehnt, wobei man nicht mit den Spalten der Tabelle, sondern mit den Attributen des Objekts Abfragen erstellt.\\ \addlinespace	

	\textbf{OQL}&
	(siehe \glossarmark{Object Query Language})\\ \addlinespace

	\textbf{ORM}&
	(siehe \glossarmark{Object-relational mapping})\\ \addlinespace

	\textbf{Projektstrukturplan}&
	Ein Synonym für Work Breakdown Structure (WBS): Eine in der Regel an den Liefergegenständen orientierte Anordnung von Projektelementen, die den Gesamtinhalt und -umfang des Projekts strukturiert und definiert.\cite{proj_mgmt_book}\\ \addlinespace	

	\textbf{Refactoring}&
	Eine Veränderung der internen Struktur der Software, welche sie verständlicher und wartbarer macht, jedoch ohne eine Änderung des ursprünglichen Verhaltens zu bewirken.\cite{feathers2004working}\\ \addlinespace		

	\textbf{Representational State Transfer}&
	Representational State Transfer (\glossarmark{REST}) ist ein mögliche Kommunikationsschnittstelle für Webanwendungen und wird vorallem für System-System-Kommunikation verwendet. Nach \glossarmark{REST} liefert eine Web-Seite genau einen Seiteninhalt zurück und das auch bei mehrmaligen Aufrufen.\\ \addlinespace

	\textbf{REST}&
	(siehe \glossarmark{Representational State Transfer})\\ \addlinespace

	\textbf{RESTful}&
	(siehe \glossarmark{Representational State Transfer})\\ \addlinespace

	\textbf{Sonar}&
	Sonar ist ein statisches Code-Analyse Tool.\\ \addlinespace

	\textbf{Stakeholder}&
	 Stakeholder sind für den Requirement Engineer wichtige Quellen zur Identifikation möglicher Anforderungen des Systems.\cite{req_eng_book} Ein Stakeholder ist eine Person, die in irgendeiner Weise vom Projekt betroffen ist, jedoch nicht notwendigerweise direkt Einfluss auf den Projektverlauf haben muss.\\ \addlinespace			

\end{longtable}

\todo{AddWebHooks, asynchron, Poll/Push-Prinzip, Eulerischenkreis, Aussagenlogik, ACID (RDBMS)}