%%%%%%%%%%%%%%%%%%%%%%%%%%%%%%%%%%%%%%%%%%%%%%%%%%%%%%%%%%%%%%%%%
%
% Project     : Bachelorarbeit
% Title       : Machbarkeitsanalyse für eine ressourcenorientierte Schnittstelle zur Verarbeitung grundlegender Probleme der Informatik
% File        : abstract.tex Rev. 01
% Date        : 01.03.2015
% Author      : Raffael Santschi
%
%%%%%%%%%%%%%%%%%%%%%%%%%%%%%%%%%%%%%%%%%%%%%%%%%%%%%%%%%%%%%%%%%

\thispagestyle{empty}


\newpage
\thispagestyle{empty}
\chapter*{Abstract}\label{abstract}
Bei vielen Problemen in der Informatik kann eine exakte Lösung nicht innerhalb nützlicher Frist berechnet werden. Stattdessen werden Approximierungsalgorithmen verwenden, die sich von Problem 
zu Problem unterscheiden. Ziel dieser Arbeit ist eine Machbarkeitanalyse für eine generische Schnittstelle zur Lösung solcher Probleme. Die Schnittstelle sollte keinerlei Kenntnisse der theoretischen 
Informatik oder der jeweiligen Lösungsansätze der theoretischen Probleme voraussetzen. Dazu wurden fünf Probleme mit hoher Laufzeitkomplexität ausgewählt, welche nicht nur von rein 
wissenschaftlichem Interesse sind. Die Problemfelder wurden auf ihre Ein- und Ausgabeparameter analysiert und die dazugehörigen Algorithmen betrachtet. Im späteren Verlauf der Arbeit wurde 
ein sechstes Problem dazugenommen, welches eine leichte Abwandlung eines bereits ausgewählten Problems ist. Damit konnte geprüft werden, ob sich die Implementierung der beiden Probleme 
generischer umsetzen lässt als bei den anderen Problemen.\\

Beim Erstellen des Konzepts wurden die Gemeinsamkeiten des Berechnungsablaufs betrachtet. Um mehr Freiheiten bei der Implementierung zu haben und eine grosse Benutzerfreundlichkeit zu 
garantieren, wurde die Nutzer- und Algorithmus-Domäne voneinander entkoppelt. Dies bot die Möglichkeit, jeweils eine andere Domänensprache zu verwenden. Zwischen den beiden Domänen 
kamen pre- und post-Aktionen zum Einsatz, welche die Datenaufbereitung für den Algorithmus beziehungsweise den Nutzer durchführten.\\

Vor der Umsetzung wurde eine Nutzwertanalyse zur Auswahl eines geeigneten Datenbanksystems durchgeführt. Nach einer Vorselektierung standen eine relationale, eine objektorientierte und 
eine dokumentorientierte Datenbank zur Auswahl. Beim Vergleich wies das dokumentorientierte Datenbanksystem mit seiner Flexibilität einen grossen Vorteil auf. Diese Flexibilität ermöglichte eine 
schnelle und unkomplizierte Implementierung und bietet dies auch für kommende Erweiterungen.\\

Als Prototyp wurde ein REST API implementiert, welches die Funktionalität zur Berechnung der sechs vorher definierten Probleme bereitstellt. Hinter einem dieser Probleme wurde ein 
Algorithmus implementiert, womit der ganze Prozess getestet werden konnte. Bei den anderen Problemen wurde anhand der Recherche Ein- und Ausgabeschemata der Algorithmen definiert.\\

Das Konzept für die Schnittstelle konnte für alle sechs Probleme angewandt und der Ablauf konnte generisch gehalten werden. Das Auswahlverfahren der Probleme hat sich bewährt, die Probleme 
zeigen unterschiedliche Ausprägungen in ihren Parametern und ihrer Lösungsweise. Einige Probleme können mit dem gleichen generischen Algorithmus gelöst werden. Zwei der Probleme, welche 
beide aus dem Bereich "`Netzwerk Design"' stammen, benötigen sehr unterschiedliche Herangehensweisen. Die Schnittstelle bietet genug Flexibilität, um ganz unterschiedliche Probleme zu behandeln und 
verschiedene Algorithmen anzusteuern. Die Machbarkeitsstudie ist als erfolgreich zu betrachten.