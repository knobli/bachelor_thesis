%%%%%%%%%%%%%%%%%%%%%%%%%%%%%%%%%%%%%%%%%%%%%%%%%%%%%%%%%%%%%%%%%
%
% Project     : Bachelorarbeit
% Title       : Machbarkeitsanalyse für eine ressourcenorientierte Schnittstelle zur Verarbeitung grundlegender Probleme der Informatik
% File        : abstract.tex Rev. 01
% Date        : 01.03.2015
% Author      : Raffael Santschi
%
%%%%%%%%%%%%%%%%%%%%%%%%%%%%%%%%%%%%%%%%%%%%%%%%%%%%%%%%%%%%%%%%%

\thispagestyle{empty}


\newpage
\thispagestyle{empty}
\chapter*{Abstract}\label{abstract}
Ziel dieser Arbeit war es zu analysieren, ob eine generische Schnittstelle für grundlegende Probleme der Informatik möglich ist. Die Schnittstelle sollte keinerlei Kenntnisse 
der theoretischen Informatik oder der jeweiligen Probleme voraussetzen. Dazu wurden fünf Probleme mit hoher Laufzeitkomplexität aus 'Computers and Intractability: A Guide to the Theory of 
NP-Completeness' von Micheal Garey und David S. Johnson ausgewählt. Die Problemfelder wurden auf ihre Ein- und Ausgabeparameter analysiert und die dazugehörigen 
Algorithmen wurden betrachtet. Im späteren Verlauf der Arbeit wurde noch ein sechstes Problem dazugenommen, welches sehr ähnlich zu einem bereits ausgewählten Problem war. Dies 
um zu schauen, wie sich die beiden Probleme bei der Implementierung im Gegensatz zu den anderen Problemen verhalten.\\

Beim Erstellen des Konzept wurden die Gemeinsamkeiten des Berechnungsablaufs betrachtet. Um mehr Freiheiten zu haben und Benutzerfreundlichkeit zu garantieren, wurde die Nutzer- und 
Algorithmus-Welt voneinander entkoppelt. Dies bot die Möglichkeit jeweils eine andere Domänensprache zu verwenden. Zwischen den beiden Welten kamen pre- und post-Aktionen zum Einsatz, 
welche die Datenaufbereitung für den Algorithmus bzw. den Nutzer durchführten.\\

Vor der Umsetzung wurde eine Nutzwertanalyse zur Auswahl eines geeigneten Datenbanksystems durchgeführt. Nach der Vorselektierung standen die relationalen, objektorientierten und 
dokumentorientierten Datenbanken zur Auswahl. Beim Vergleich wies das dokumentorientierte Datenbanksystem einige Vorzüge auf, wie seine Flexibilität, welche eine schnelle und unkomplizierte 
Erweiterung bzw. Anpassung ermöglicht.\\

Als Prototyp wurde ein REST API implementiert, welches die Funktionalität bereitstellt, Berechnung von den sechs verschiedenen Problemen zu erstellen. Hinter einem dieser Probleme wurde ein 
Algorithmus angebunden, womit der ganze Prozess durchgespielt werden konnte. Bei den anderen Problemen wurde anhand der Recherche Ein- und Ausgabeschemata der Algorithmen definiert. 
Zusätzlich wurde zu jedem Problem ein Validator geschrieben, welcher überprüft, ob ein Resultat gültig ist oder nicht.\\

Das Konzept für die Schnittstelle konnte für alle sechs Probleme angewandt und der Ablauf generisch gehalten werden. Die Machbarkeitsstudie ist als erfolgreich zu betrachten. Das 
Auswahlverfahren der Probleme hat sich bewährt, die Probleme zeigten unterschiedliche Ausprägungen. Einige Probleme könnten mit dem gleichen generische Algorithmus gelöst werden. Zwei 
andere Probleme, welche beide aus dem gleichen Bereich stammen, benötigen sehr unterschiedliche Herangehensweisen. Die Schnittstelle bietet genug Flexibilität um ganz unterschiedliche 
Probleme zu behandeln und verschiedene Algorithmen anzusteuern.