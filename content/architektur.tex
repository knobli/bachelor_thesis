%%%%%%%%%%%%%%%%%%%%%%%%%%%%%%%%%%%%%%%%%%%%%%%%%%%%%%%%%%%%%%%%%
%
% Project     : Bachelorarbeit
% Title       : Machbarkeitsanalyse für eine ressourcenorientierte Schnittstelle zur Verarbeitung grundlegender Probleme der Informatik
% File        : architektur.tex Rev. 01
% Date        : 01.03.2015
% Author      : Raffael Santschi
%
%%%%%%%%%%%%%%%%%%%%%%%%%%%%%%%%%%%%%%%%%%%%%%%%%%%%%%%%%%%%%%%%%

\chapter{Konzept der Schnittstelle \resultAssignment{[R4]}}\label{chap.architektur}
In diesem Kapitel wird auf die Strukur und das Konzept der Schnittstelle eingegangen. Dieses Konzept legt die Grundlage für die Umsetzung und ist somit entscheidend für die Lösung der 
Aufgabe.

\section{Übersicht}\label{architektur_uebersicht}
Die Systemumgebung (siehe Abbildung \ref{fig:system_scope}) hat zwei Berührungspunkte mit der Aussenwelt. Der eine ist zum Nutzer, der andere zu einem Verarbeitungssystem. 
Um die Daten zu speichern, wird eine Datenbank benötigt. Dadurch wird ein asynchroner Ablauf und ein mehrfaches Abfragen der Daten ermöglicht.
\begin{figure}[h]
\centering
\includegraphics[scale=0.8]{images/visio/Systemscope.png}
\caption[Systemübersicht]{Systemübersicht \selfmade{}}
\label{fig:system_scope}
\end{figure}

\section{Konzept}\label{arch_backend}
Das Konzept sollte einfach und flexibel sein, damit das System möglichst schnell erweitert werden kann. Beim Betrachten der Probleme und der Abläufe wurde bemerkt, dass die Vorgänge 
Ähnlichkeiten aufweisen. Die Daten werden jeweils angenommen und für einen bestimmten Algorithmus aufbereitet. Das Verarbeitungssystem schickt das Resultat zurück und dieses wird dann 
wieder umgewandelt, so dass es für den Nutzer brauchbar ist. Es finden zwei Umwandlungen statt. Die Umwandlungen an sich sind von Problem zu Problem verschieden, es gibt jedoch 
auch dort gewisse Ähnlichkeiten. Für den Aufbau des System wurde das \gls{ddd} gewählt, da es sich in der Entwicklungswelt etabliert hat (siehe \cite{evans2004domain} 
\cite{soft_arch_book}). In \autoref{fig:architektur} wird der Aufbau des Systems gezeigt. Die  \gls{domainlanguage} des Nutzers wird mit Hilfe des Controllers, Entities und der Translators auf 
die \gls{domainlanguage} der Algorithmen abgebildet.

\begin{figure}[h]
\centering
\includegraphics[scale=0.8]{images/visio/architektur_db.png}
\caption[Architekturaufbau des Systems]{Architekturaufbau des Systems \selfmade{}}
\label{fig:architektur}
\end{figure}
 
\FloatBarrier
\subsection{REST API}
Für die Schnittstelle wird ein \gls{rest} \gls{api} erstellt, welches die nötigen Funktionen bietet. Das \gls{api} funktioniert nach dem \gls{rest} De-facto-Standard 
(siehe \cite{masse2011rest}). Falls Fehler auftreten, werden \glspl{http_statuscode} verwendet, um diese an den Aufrufer weiterzugeben.

\subsection{Business Logik}
Die Business Logik besteht aus Translator, Service, Entity und Validator von den jeweiligen Problemen. 
\paragraph{Translator}
Es gibt jeweils ein `ComputationTranslator' für die Umwandlung vom Nutzer hin zum Algorithmus, in diesem Translator werden die Daten für den Algorithmus aufbereitet. Daneben gibt es einen 
`SolutionTranslator' für die Umwandlung vom Algorithmus zurück in das System, bei welchem das Resultat in ein menschenlesbares Resultat umgewandelt wird. In den Translators steckt die 
ganze Logik. Es fliesst ein, wie der Algorithmus die Daten für die Verarbeitung benötigt und wie das Resultat zurück kommt. Es wäre zum Beispiel möglich, die Daten in Kombinationen für einen 
evolutionären Algorithmus umzuwandeln. Damit würde theoretisch ein generischer evolutionärer Algorithmus für alle Probleme ausreichen, jedoch ist das nach dem No-free-Lunch Theorem 
\cite{no_free_lunch} unmöglich. Ebenfalls möglich wäre eine Umwandlung des Stundenplanproblems auf ein Knotenfärbungsproblem und somit könnte der gleiche Algorithmus angesprochen 
werden (vergleiche \cite{timetabling_abdullah}).  Zusätzlich kann im Translator das Resultat mit zusätzlichen Informationen, zum Beispiel Statistiken, angereichert werden. Die Möglichkeiten mit 
dem Konzept der Translators ist sehr vielfältig.
\paragraph{Service}
Der Service ist zentraler Dreh- und Angelpunkt, er spricht über das Repository die Datenbank an, startet die Umwandlungen und Validierungen und gibt die sequentielle Abfolge vor. Er kann 
sehr generisch gehalten werden und benötigt keine problemspezifische Methoden. Zu den Services gehören auch die Solver-Klassen, welche für das Starten der jeweiligen Algorithmen zuständig 
sind.
\paragraph{Entity}
Die Entitäten repräsentieren die Datenstruktur der Algorithmen, der Nutzereingaben oder der Resultate. Mit den Entitäten wird gerechnet und ihr Zustand wird in der Datenbank persistiert. 
Für die Erstellung der Entitäten wurde für jedes Problem analysiert, wie die Parameter am besten eingegeben werden und wie diese dann vom Algorithmus gebraucht werden. Weiter 
wurde ein nützliches Format für die verschiedenen Resultate ermittelt.
\paragraph{Validator}
Der Validator entscheidet, ob eine Lösung gültig ist oder nicht. Die Validatoren wurden erstellt, um bei der Entwicklung der Algorithmen die Resultate überprüfen zu können und um manuell 
erstellte Lösungen zu testen. Grundsätzlich könnte ein Algorithmus geschrieben werden, welcher sich nur auf die Antwort des Service verlässt und so lange neue Kombinationen versucht, 
bis die Schnittstelle keine Fehler mehr findet. Wie bereits im \autoref{cat_theo_inf} erklärt, kann jedes \gls{np}-vollständige Problem in \glslink{polynomialzeit}{polynomialer} Zeit validiert 
werden. Weiter werden im Validator Angaben überprüft, welche der Algorithmus eventuell nicht berücksichtigen konnte. Dies hilft dem Nutzer die Qualität des Resultats abzuschätzen.

\subsection{Persistence API}
Die Abstraktion der Datenbank wird mittels eines Persistence \glspl{api}, welches mit der Datenbank interagiert, realisiert. Dieses \gls{api} ist für das Laden und Speichern der Daten 
verantwortlich und bietet die Möglichkeit, spezifische Abfragen auszuführen. Diese Funktionalitäten werden in Entity-spezifischen Repositories zur Verfügung gestellt.

\subsection{Datenbank}
Jedes Problem hat seine eigene Ausprägung von Computation und Solution, welche abgespeichert werden müssen. Die Datenbank sollte, wenn möglich, eine ähnliche Flexibilität wie 
das Programm aufweisen. Zum überwiegenden Teil sind nur Einfüge-Operationen notwendig, nur selten wird ein Status einer Berechnung geändert. Pro Berechnung wird jeweils nur ein 
Algorithmus angesetzt, demnach gibt es keine gleichzeitigen Zugriffe auf die Ressourcen. Dies wäre jedoch auch kein Problem, da die Resultate separat gespeichert und nicht verändert 
werden. Die Vorgänge benötigen somit nicht zwingend eine transaktionale Unterstützung seitens der Datenbank. Die Daten werden immer in der Nutzer-Sicht gespeichert, damit jederzeit 
Algorithmus unabhängig auf die Daten zugreifen werden kann.

\newpage
\subsection{Ablauf}
In \autoref{fig:workflow} wird der Ablauf des ganzen Vorganges und die Interaktion mit dem Nutzer und dem Verarbeitungssystem verdeutlicht. Die Translators, welche in dem 
Prototyp verwendet werden, sind nur eine Möglichkeit, welche dieses Konzept bietet. Generell baut dieses Konzept auf eine pre- und post-Aktion vor bzw. nach dem Starten des Algorithmus 
auf.

\begin{figure}[h]
\centering
\includegraphics[scale=0.605]{images/visio/workflow.png}
\caption[Flussdiagramm des Arbeitsablaufs]{Flussdiagramm des Arbeitsablaufs \selfmade{}}
\label{fig:workflow}
\end{figure}

\FloatBarrier
\newpage

Die \autoref{fig:sequenz_diagramm_start} zeigt das Sequenzdiagramm für den Start einer Berechnung. Alle Komponenten mit '\{Problem\}' sind spezifische Problem-Komponenten, 
die anderen sind generische. Bei Speichern einer Berechnung wird der Status 'CREATED' gesetzt. Nach dem Speichern wird über die `Solver'-Komponente asynchron das Verarbeitungssystem 
gestartet und der Status auf 'STARTED' gesetzt. Das Verarbeitungssystem holt sich die benötigten Informationen. Der Controller lädt die Parameter vom Service, dieser wiederum lädt es vom 
Repository und wandelt es für den Algorithmus um.

\begin{figure}[h]
\centering
\includegraphics[scale=0.74]{images/visio/sequenz_diagramm_start.png}
\caption[Start einer Berechnung]{Start einer Berechnung \selfmade{}}
\label{fig:sequenz_diagramm_start}
\end{figure}

\newpage

Die \autoref{fig:sequenz_diagramm_result} zeigt das Sequenzdiagramm für das Abspeichern eines Resultats einer beliebigen Berechnung. Bei einem Speicheraufruf werden zuerst die 
Berechnungsinformationen geladen, danach wird die Lösung vom Algorithmus mit Hilfe der Eingabeparameter transferiert und vom Validator validiert. Nun wird anhand des Resultattyps der 
Status der Berechnung geändert. Das Resultat wird mit dem Ergebnis der Validierung in die Datenbank gespeichert. Wenn der Nutzer den Status einer Berechnung abruft, fragt der Controller 
über den Service den Status ab. Der Service lädt die Berechnung, fragt die vorhandenen Resultate ab, fügt diese zu einem Status zusammen und schickt den Status an den Controller zurück.

\begin{figure}[h]
\centering
\includegraphics[scale=0.72]{images/visio/sequenz_diagramm_result.png}
\caption[Abspeichern des Resultats einer Berechnung]{Abspeichern des Resultats einer Berechnung \selfmade{}}
\label{fig:sequenz_diagramm_result}
\end{figure}

\section{Datenbank Varianten}\label{db_varianten}
Es gibt verschiedene Datenbanktypen mit verschiedenen Vor- und Nachteilen. In diesem Abschnitt werden vier verschiedene Typen miteinander verglichen und der Beste für diesen 
Anwendungszweck ausgewählt. Um die Eigenheiten der Datenbanken hervorzuheben, wird bei jeder Art das gleiche Beispiel mit der spezifischen Definitions-- und Abfragesprache gemacht.

\subsection{CAP-Theorem}\label{cap_theorem}
Das \gls{cap_theorem} ist im Jahr 2000 aus einer Vermutung von Eric Brewer entstanden \cite{cap_brewer}. Das Theorem behauptet, dass die Werte Konsistenz (C), Verfügbarkeit (A) und 
Partitionstoleranz (P) ein Dreieck bilden (siehe \autoref{fig:cap}) und dass ein verteiltes System nur jeweils zwei dieser Eigenschaften gleichzeitig erfüllen kann.

\begin{figure}[h]
\centering
\includegraphics[scale=0.8]{images/visio/cap.png}
\caption[CAP-Theorem]{CAP-Theorem \selfmade{}}
\label{fig:cap}
\end{figure}

\subsubsection{Beispiele}

\begin{itemize}
	\item \textbf{AP}: Domain Name System (DNS), NoSQL Datenbanken
	\item \textbf{CA}: Relationale Datenbanken
\end{itemize}

\subsection{Relationales Datenbanksystem}\label{rdbms}
Ein \gls{rdbms} basiert auf Transaktionen und dem \gls{acid} Prinzip (vergleiche \cite{limited2010introduction}). 
\begin{itemize}
	\item \textbf{Atomicity}: Eine Reihe von Befehlen wird entweder ganz oder gar nicht ausgeführt.
	\item \textbf{Consistency}: Der Datenzustand ist nach jeder Veränderung wieder konsistent, wenn es auch vorher der Fall war.
	\item \textbf{Isolation}: Unterschiedliche Befehlsketten beeinflussen sich nicht gegenseitig.
	\item \textbf{Durability}: Vollzogene Änderungen sind dauerhaft, auch bei einem Systemfehler.
\end{itemize}
Beziehungen zwischen Tabellen werden mit Fremdschlüsseln definiert, welche auf den Hauptschlüssel der referenzierten Tabelle zeigt. \autoref{lst:table_definition_rdbms} zeigt die
Tabelle 'Person', welche eine Verbindung zur Tabelle 'Address' hat.

\begin{lstlisting}[language=SQL, caption=Tabellendefinition in relationalem Datenbanksystem, label=lst:table_definition_rdbms]  
    CREATE TABLE ADDRESS(
             ID       INTEGER(11),
             Street   VARCHAR(30),
             Zip      INTEGER(6),
             City     VARCHAR(12));

    CREATE TABLE PERSON(
               ID          INTEGER(11),
               Name        VARCHAR(50),
               FK_Address  INTEGER(11));
\end{lstlisting}

Die Abfrage der Strasse bei einer relationaler Datenbank mittels \gls{sql} ist in \autoref{lst:select_street_rdbms} dargestellt.

\begin{lstlisting}[language=SQL, caption=Abfrage in relationalem Datenbanksystem, label=lst:select_street_rdbms]  
    SELECT ADDRESS.Street
    FROM PERSON 
    JOIN ADDRESS on PERSON.FK_Address = ADDRESS.ID;
    WHERE PERSON.Name = 'Thomas';
\end{lstlisting}

\subsection{Objektrelationales Datenbanksystem}\label{ordbms}
\glslink{ordbms}{Objektrelationale Datenbankmanagementsysteme (ORDBMS)} wurden entwickelt, um das \gls{rdbms} mit den Funktionen des objektorientierten Paradigmas zu erweitern.
Die folgende Beschreibung der Eigenschaften ist aus \cite{limited2010introduction} abgeleitet. In objektrelationalen Datenbanksystemen werden 
komplexe Datentypen unterstützt, so können zusätzliche Typen definiert und bei einer Tabellendefinition verwendet werden.  
In \autoref{lst:type_definition_ordbms} ist eine Typendefinition von 'AddressType' zu sehen, welche wiederum in \autoref{lst:use_type_definition_ordbms} 
in der Tabelle 'Person' verwendet wird.

\begin{lstlisting}[language=SQL, caption=Typendefinition im objektrelationalen Datenbanksystem, label=lst:type_definition_ordbms]  
    CREATE TYPE AddressType  AS
             (Street   VARCHAR(30),
              Zip      INTEGER(6),
              City     VARCHAR(12));
\end{lstlisting}

\begin{lstlisting}[language=SQL, caption=Verwendung von Typendefinition im objektrelationalen Datenbanksystem, label=lst:use_type_definition_ordbms]  
    CREATE TABLE PERSON(
               ID       VARCHAR(20),
               Name     VARCHAR(50),
               Address  AddressType);
\end{lstlisting}

Eine Abfrage der Strasse einer bestimmten Person wird mittels der erweiterten \gls{sql}-Syntax, wie in \autoref{lst:select_street_ordbms} dargestellt, abgefragt.

\begin{lstlisting}[language=SQL, caption=Abfrage in objektrelationalem Datenbanksystem, label=lst:select_street_ordbms]  
    SELECT Address.Street
    FROM PERSON;
    WHERE Name = 'Thomas';
\end{lstlisting}

Zusätzlich können auch Arrays von Typen definiert werden, in \autoref{lst:use_array_ordbms} ist die Definition der Tabelle 'Person' mit 
bis zu fünf E-Mail-Adressen gezeigt.

\begin{lstlisting}[language=SQL, caption=Verwendung von Array in objektrelationalem Datenbanksystem, label=lst:use_array_ordbms]  
    CREATE TABLE PERSON(
               ID       VARCHAR(20),
               Name     VARCHAR(50),
               Mail     VARCHAR(40) ARRAY[5]);
\end{lstlisting}

Weiter bieten die Systeme die Möglichkeiten, Vererbungshierarchien und Methoden auf Typen und Tabellen zu definieren. Der Anwendungsbereich 
liegt bei Applikationen, welche viele kurzlebige Transaktionen mit komplexen Objekten durchführen.

\subsection{Objekorientiertes Datenbanksystem}\label{object_db}
Ein \gls{oodbms} hat als Ziel eine bessere und nähere Zusammenarbeit mit objektorientierten Sprachen. 
Die Informationen über \gls{oodbms} sind aus \cite{limited2010introduction} und \cite{oodbms_details} entnommen. Der grosse Unterschied zu anderen Datenbanksystemen
ist die Persistierung, bei \gls{oodbms} wird bereits beim Erstellen eines neuen persistenten Objekts im Code ein Pointer auf das Objekt in der Datenbank zurückgegeben. 
Das objektrelationale Mapping, welches bei \gls{rdbms} und \glslink{ordbms}{ORDBMS} notwendig ist, entfällt. 
Weiter unterstützen die Systeme Versionierung von Objekten, die Datenbank kann somit mit verschiedenen Versionen der Objekten gleichzeitig umgehen. 
Dies kann bei einer geplanten Umstrukturierung sehr hilfreich sein, wenn die neue Version erst für Tests verwendet wird und die bestehenden Objekte erst nach dem 
Release migriert werden sollen. \autoref{lst:table_definition_oodbms} zeigt, wie das Objekt 'Person' in \gls{odl} definiert wird.

\begin{lstlisting}[language=C++, caption=Objektdefinition in objektorientierem Datenbanksystem, label=lst:table_definition_oodbms]  
    class PERSON
    {
          attribute string Id;
          attribute string Name;
          attribute struct Address
              { string Street,
                short Zip,
                string City} address;
    }
\end{lstlisting}

Für die Abfrage der Strasse einer Person wird ein Statement wie in \autoref{lst:select_street_oodbms} verwendet. Diese Abfragesprache nennt
sich \gls{oql}. Die Attribute der Objekte und Structs können mit einem Punkt separiert verkettet werden.

\begin{lstlisting}[language=SQL, caption=Abfrage in objektorientierem Datenbanksystem, label=lst:select_street_oodbms]  
    SELECT p.Address.Street
    FROM persons p
    WHERE p.Name = 'Thomas';
\end{lstlisting}

Die objektorientierten Datenbanksysteme bieten die aus den \gls{rdbms} bekannten Abfragefunktionen (Avg, Max, Min, Distinct). 
Zusätzlich können bei diesen Systemen Vererbungshierarchien und Methoden in Objekten definiert werden. Sie werden bei Applikationen verwendet,
welche viele, lange Transaktionen mit komplexen Objekten haben. Lade- und Speichervorgänge sind bei \gls{oodbms} sehr komplex, deshalb 
sollten kurze Transaktionen, in welchen sehr vielen komplexe Objekten geladen werden, vermieden werden.

\subsection{NoSQL Datenbanksystem}\label{no_sql_db}
NoSQL Datenbanksysteme folgen nicht dem \gls{acid} Prinzip. Sie sind entwickelt worden, um den erforderlichen Geschwindigkeiten und Grössen von Anwendungen in Firmen wie Google, 
Facebook, Yahoo und Twitter zu genügen. Die Systeme sind nicht relational aufgebaut und verwenden oft eine andere Abfragesprache als \gls{sql} (siehe \cite{vaish2013getting}). 
Die meisten NoSQL Datenbanksystem setzen auf das \gls{base}-Prinzip von Eric Brewer und sind im Bezug auf das \gls{cap_theorem} je nach Implementation in AP oder CP einzugliedern.

\begin{myQuote}{Gaurav Vaish \cite{vaish2013getting}}
"`\textbf{Basic availability}: Each request is guaranteed a response—successful or failed execution.\\
\textbf{Soft state}: The state of the system may change over time, at times without any input (for eventual consistency).\\
\textbf{Eventual consistency}: The database may be momentarily inconsistent but will be consistent eventually."'
\end{myQuote}

Nahezu alle NoSQL Datenbanksysteme sind schemalos, dass heisst, die Tabellen müssen nicht definiert werden. Die Abfragen gestalten sich einfach, JOIN Befehle entfallen 
komplett. Durch ihren Aufbau sind sie grundsätzlich performanter als anderen Datenbanken \cite{nosql_vs_sql}, das geht jedoch meist auf die Kosten der Konsistenz. Deshalb ist es wichtig, 
dass einem die Anforderungen bei der Auswahl einer Datenbank klar sind.

\subsubsection{NoSQL Datenbanktypen}\label{no_sql_db_subgroups}
NoSQL ist eine Übergruppe von unterschiedlichen Datenbanktypen. In diesem Abschnitt wird auf drei verschiedene NoSQL Datenbanktypen eingegangen, die Eigenschaften der jeweiligen
 Typen sind aus \cite{vaish2013getting} entnommen.

\paragraph{Document}
Bei dokumentorientierten Datenbanken werden \gls{semi_structured_data} abgespeichert, diese verwenden meistens die Dateiformate \gls{json}, 
\gls{xml} oder \gls{yaml}. Die meisten Datenbanksysteme dieser Art haben kein definiertes Schema für Einträge oder 
nur ein teilweise definiertes Schema. Es ist somit möglich ganz unterschiedliche Elemente in eine Collection, so werden Tabellen in diesem Kontext genannt, zu speichern. Diese Erklärung 
erinnert an \gls{blob} in relationalen Datenbanken, jedoch können hier Indexes auf bestimmte Attribute gesetzt und innerhalb der Elemente effizient gesucht 
werden.

In \autoref{lst:person_mongodb} wird ein Eintrag gezeigt, welches in der Collection 'Persons' gespeichert ist. Dieses Beispiel basiert auf einer MongoDB Instanz, die Definition 
für die Collection entfällt komplett.

\begin{lstlisting}[language=JSON, caption=Personen Element in JSON Format, label=lst:person_mongodb]  
    {
       _id: 1,
       name: "Thomas",
       address: {
                 street: "Bahnhofstrasse 45",
                 zip: 8315,
                 city: "Lindau"}
    }
\end{lstlisting}

\autoref{lst:select_mongodb} zeigt die Abfrage der Strasse bei MongoDB, welche das Resultat aus \autoref{lst:select_result_mongodb} liefert. Bei MongoDB wird bei einer Abfrage standardmässig 
immer das ganze Element zurückgegeben. Falls nur gewisse Attribute abgefragt werden sollen, können diese in dem Projektionsabschnitt definiert werden. Die ID des Elements wird, falls 
sie nicht explizit ausgeschlossen wird, immer mitselektiert.

\begin{lstlisting}[language=SQL, caption=Abfrage in MongoDB, label=lst:select_mongodb]  
    db.persons.find({name: "Thomas"}, {_id: 0, "address.street": 1})
\end{lstlisting}

\begin{lstlisting}[language=JSON, caption=Resultat der Abfrage in MongoDB, label=lst:select_result_mongodb]  
    {
       address: {
                 street: "Bahnhofstrasse 45"}
    }
\end{lstlisting}

\paragraph{Column-oriented}
Anders als bei relationalen Datenbanken, welche zeilenorientiert arbeiten, werden die Daten in diesen Systemen spaltenorientiert gespeichert. Eine serialisierte Lohndatenbank würde beim 
zeilenorientierten Konzept wie in \autoref{lst:ser_rowbased_db} und beim spaltenorientierten Ansatz wie in \autoref{lst:ser_columnbased_db} aussehen.

\begin{lstlisting}[language=SQL, caption=Serialisierung zeilenorientierte Datenbank, label=lst:ser_rowbased_db]  
    1,Thomas,100000
    2,Patrick,120000
    3,Andreas,200000
\end{lstlisting}

\begin{lstlisting}[language=SQL, caption=Serialisierung spaltenorientierte Datenbank, label=lst:ser_columnbased_db]  
    1,2,3
    Thomas,Patrick,Andreas
    100000,120000,200000
\end{lstlisting}

Die Definitions- und Abfragesprache ist gleich wie bei relationalen Datenbanken, deshalb wird hier auf Beispiele verzichtet. Der Vorteil 
einer spaltenorientierten Datenbank liegt in der Performance bei \glspl{aggregationsfunktion}. Weiter führt dieser Datenbanktyp Veränderungen, 
welche alle Werte einer Spalte betreffen, effizient durch. Dafür ist das Einfügen einer neuen Zeile oder das Auslesen einer ganzen Zeile aufwändiger als bei dem zeilenorientierten Ansatz.

\paragraph{Graph}
In Graphdatenbanken sind die abgespeicherten Objekte die Knoten und die Beziehung der Objekte bilden die Kanten. Die Kanten können zusätzlich mit Eigenschaften versehen werden. Dieser 
Datenbanktyp kann für komplexe Netzwerke verwendet werden. Der Vorteil liegt bei der Handhabung der Beziehungen. Das Herausfinden der kürzesten Route zwischen zwei Elementen über 
ihre definierten Beziehungen stellt für diesen Datenbanktyp keine Herausforderung dar.

Die Daten werden je nach Implementation auch in Dokumenten abgespeichert, somit entfällt auch hier eine Definition des Schemas. In \autoref{lst:select_neo4j} ist das Beispiel 
für das Abfragen der Strasse einer Person in Neo4j zu sehen.

\begin{lstlisting}[language=SQL, caption=Abfrage in Neo4j, label=lst:select_neo4j]  
       MATCH (person:Person)
       WHERE person.name = "Thomas"
       RETURN person.address.street;
\end{lstlisting}

\subsection{Vorselektierung}\label{preselection}
Um nicht alle sechs vorgestellten Datenbanksysteme in der Nutzwertanalyse miteinander vergleichen zu müssen, wurde eine Vorselektierung aufgrund der Ist-Analyse gemacht. 
Objektrelationale Datenbanken unterscheiden sich vor allem bei der Definition des Datenbankschemas, deshalb wurden sie mit den relationalen Datenbanken zusammen genommen. Falls die 
relationalen Datenbanken bei der Nutzwertanalyse als Gewinner hervorgehen, müsste noch bestimmt werden, welcher der beiden Typen verwendet werden soll. Die Anforderungen lassen auf 
keinen Anwendungsfall für eine spaltenorientierte Datenbank schliessen. In den einzelnen Problemen spielen Beziehungen zwischen Objekten zwar oft eine Rolle, doch das würde eher in den 
Bereich des Algorithmus fallen. Aus diesem Grund werden Graphdatenbanken ebenfalls ausgeschlossen. Somit bleiben noch die relationale, die objektorientierte und die dokumentorientierte 
Datenbank für die Nutzwertanalyse übrig.

Da es innerhalb der ausgewählten Datenbanksystemen sehr viele verschiedene Produkte gibt, wurde von jedem Typ eines ausgewählt. Bei den relationalen Datenbanken fiel die 
Entscheidung auf MySQL\cite{mysql}, da dieses kostenlos verfügbar ist und bereits Erfahrung vorhanden war. Die ausgewählte objektorientierte Datenbank ist Objectivity/DB \cite{objectivitydb}, 
welche auch beim Teilchenbeschleuniger in CERN verwendet wird \cite{cern_objectivity}. Da MongoDB \cite{mongodb} sehr verbreitet und eine gute Integration in die Java Frameworks hat 
\cite{spring_data_mongodb} \cite{hibernate_mongodb}, wurde dieses Produkt unter den dokumentorientierten Datenbanken ausgewählt.

\newpage
\section{Nutzwertanalyse}\label{architektur_nutzwertanalyse}

\subsection{Bewertungskriterien}\label{architektur_bewertungspunkte}

In der Nutzwertanalyse werden folgende Punkte betrachtet, nach dem angegebenen Schema bewertet und dann gewichtet. 
Die Kriterien sind grösstenteils aus dem \gls{cap_theorem} abgeleitet.

\paragraph{Aufwand}
\begin{itemize}
	\item \textbf{Beschreibung}: Wie gross ist der geschätzte Aufwand für die Einarbeitung und Implementation mit diesem Datenbanktyp?
	\item \textbf{Bewertung}: 1: sehr hoch, 10: sehr niedrig
	\item \textbf{Gewichtung}: 5 (Die Zeit für dieses Projekt ist beschränkt und die Entscheidung könnte zu einem Risiko werden.)
\end{itemize}

\paragraph{Flexibilität}
\begin{itemize}
	\item \textbf{Beschreibung}: Wie flexibel ist diese Variante in Bezug auf Erweiterungen/Vererbung?
	\item \textbf{Bewertung}: 1: sehr spezifisch, 10: sehr flexibel
	\item \textbf{Gewichtung}: 5 (Die Anforderungen geben vor, dass das System einfach zu erweitern sein sollte.)
\end{itemize}

\paragraph{Konsistenz}
\begin{itemize}
	\item \textbf{Beschreibung}: Wie gut ist die Datenbank in Bezug auf Konsistenz?
	\item \textbf{Bewertung}: 1: gar nicht 10: sehr gut
	\item \textbf{Gewichtung}: 2 (Die Daten haben untereinander so gut wie keine Abhängigkeiten, es werden fast nur Einfüge-Operationen benützt.)
\end{itemize}

\paragraph{Verfügbarkeit}
\begin{itemize}
	\item \textbf{Beschreibung}: Wie hoch ist die Verfügbarkeit der Datenbank-Operationen?
	\item \textbf{Bewertung}: 1: niedrig, 10: sehr hoch
	\item \textbf{Gewichtung}: 4 (Die Schnittstelle wird für möglichst viele gleichzeitige Requests ausgelegt, eine Warteschlaufe von Operationen ist nicht gewünscht.)
\end{itemize}

\paragraph{Partitionstoleranz}
\begin{itemize}
	\item \textbf{Beschreibung}: Kann die Datenbank verteilt betrieben werden?
	\item \textbf{Bewertung}: 1: nur eingeschränkt, 10: sehr einfach
	\item \textbf{Gewichtung}: 4 (Mit zunehmender Grösse wird die Datenbank sehr wahrscheinlich verteilt betrieben.)
\end{itemize}

\newpage
\subsection{Bewertung}\label{architektur_bewertung}
Anhand der zuvor definierten Kriterien wurde eine Bewertung vorgenommen. Diese Einschätzung ist nicht generell gültig, sondern bezieht sich nur auf dieses Projekt.

\begin{table}[ht]
\centering
  \begin{tabular}{>{\columncolor{darkgray}} l | p{3.5cm} | p{3.5cm} | p{3.5cm}}
	\hline
	\rowcolor{darkgray}
	\textbf{Kriterium}		&	\textbf{Relationale Datenbank} 	&	\textbf{Objektorientierte Datenbank}	&	\textbf{Dokumentorientierte Datenbank}	\\ \hline
	Produkt		&	MySQL					&	Objectivity/DB				&	MongoDB	\\ \hline
	\rowcolor{gray}
	Aufwand		&	7 (35)		&	5 (25)		&	7 (35)		\\ \hline
	Begründung		&	Ist bekannt und das Einrichten ist relativ schnell durchgeführt.	
				&	Keine Erfahrung mit objektorientierten Datenbanken, Konzept muss zuerst verstanden werden.		
				&	Dokumentorientierte Datenbank noch nie benutzt, jedoch bereits Erfahrung mit dem verwendeten Speicherformat \gls{json}.\\ \hline
	\rowcolor{gray}
	Flexibilität		&	4 (20)		&	3 (15)	&	10 (50)		\\ \hline
	Begründung		&	Jede Anpassung an den Entities Attributen erfordert auch eine Änderung des Schemas.
				&	Änderung am Schema erforderlich, was zur Migration aller Daten führt.
				&	Sehr flexibel, bei Änderungen der Entities oder neuen Problemen keine Anpassung nötig.\\ \hline
	\rowcolor{gray}
	Konsistenz		&	10 (20)	&	10 (20)&	5 (10)		\\ \hline
	Begründung		&	Konsistenz ist einer der Hauptgründe für relationale Datenbanksysteme.	
				&	Basiert auch auf dem \gls{acid} Prinzip.			
				&	Die Stärke liegt nicht in der Konsistenz.\\ \hline
	\rowcolor{gray}
	Verfügbarkeit	&	6 (24)		&	6 (24) &	9 (36)		\\ \hline
	Begründung		&	Bei Transaktionen sind zeitweise Teile der Datenbank gesperrt und nicht verfügbar.	
				&	Transaktionen haben einen negativen Einfluss auf die Verfügbarkeit.					
				&	Keine Sperrzeiten von Einträgen, somit keine Verzögerungen von Aktionen.	\\ \hline
	\rowcolor{gray}
	Partitionstoleranz	&	7 (28)		&	9 (36) &	9 (36)		\\ \hline
	Begründung		&	Benötigt aufwändige Konfiguration um die Datenbank verteilt zu betreiben.
				&	Ist für einen verteilten Betrieb ausgelegt.	
				&	Kann gut und einfach verteilt betrieben werden.		\\ \hline \hline
	\rowcolor{gray}
	\textbf{Total (gewichtet)}	&	\textbf{34 (122)}	&	\textbf{33 (123)} &	\textbf{40 (167)}	\\ \hline
  \end{tabular}
   \caption{Nutzwertanalyse - Datenbank Varianten}\label{table:bewertungskriterien}
\end{table}

\FloatBarrier
\subsection{Fazit}\label{architektur_fazit}
Das Resultat der Nutzwertanalyse hat ergeben, dass für dieses Projekt am besten eine dokumentorientierte Datenbanklösung verwendet wird.
Die Schnittstelle benötigt keine Transaktionen und profitiert von einer hohen Verfügbarkeit und Partitionstoleranz. Die Flexibilität
von dokumentorientierten Datenbanksystemen ist ideal für dieses Projekt, da sie eine schnelle und unkomplizierte Erweiterung bzw. Anpassung ermöglicht. Zusätzlich sollten 
auch die verschiedenen Ausprägungen der einzelnen Problemen gut unterstützt werden.

\newpage

\section{Datendiagramm des Datenspeichers}\label{datendiagramm_datenspeicher}
Da sich bei der Nutzerwertanalyse ergeben hat, dass eine dokumentorientierte Datenbank am besten geeignet ist, entfällt ein Datendiagramm der Tabellen, wie es von relationalen Datenbanken 
bekannt ist. Die Collections benutzen kein vordefiniertes Schema, sondern die verwendeten Klassen legen das Schema fest. Aus diesem Grund reicht ein Klassendiagramm, um zu wissen, wie 
die Daten abgespeichert werden.

In \autoref{fig:problems_diagramm} ist die Klassen-Hierarchie der Berechnungstypen dargestellt und in \autoref{fig:solutions_diagramm} werden die verschiedenen 
Resultatklassen gezeigt.


\begin{figure}[h]
\centering
\includegraphics[scale=0.62]{images/computations_diagramm.png}
\caption[Klassendiagramm der verschiedenen Problemklassen]{Klassendiagramm der verschiedenen Problemklassen \selfmade{}}
\label{fig:problems_diagramm}
\end{figure}

\begin{figure}[h]
\centering
\includegraphics[scale=0.6]{images/solutions_diagramm.png}
\caption[Klassendiagramm der verschiedenen Resultatklassen]{Klassendiagramm der verschiedenen Resultatklassen \selfmade{}}
\label{fig:solutions_diagramm}
\end{figure}
